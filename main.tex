% !TeX spellcheck = sv_SE
%http://www.cs.put.poznan.pl/ksiek/latexmath.html
%https://en.wikibooks.org/wiki/LaTeX/Advanced_Mathematics
%http://www.maths.lth.se/matematiklth/personal/magnusa/kurser/endim-ht2015/B1/kurspmB1ht15.pdf

\documentclass[a4paper]{article} 
\usepackage[T1]{fontenc} 
\usepackage[utf8]{inputenc} 
\usepackage[swedish]{babel} 
\usepackage[fleqn]{amsmath}
\usepackage{amssymb}
\usepackage{cancel}
\usepackage{graphicx}
\usepackage{enumitem}
\usepackage{systeme}
\usepackage{environ}
\usepackage[top=1in, bottom=1.25in, left=1.0in, right=1.25in]{geometry}

\setlength{\parindent}{0em}
\setlength{\parskip}{1em}

\newenvironment{task}[1]
{
	\begin{description}[align=right]
		\item [#1]~~~~
}{		%input
	\end{description}
}

\newenvironment{linsys}[1]
{
	\let\oldarraycolsep\arraycolsep
	\setlength\arraycolsep{0pt}
	\left\{\begin{array}{#1}
}{
	\end{array} \right.
	\setlength\arraycolsep{\oldarraycolsep}
}

\newenvironment{detmat}
{
	\left|\begin{matrix}
}{
	\end{matrix}\right|
}

\newenvironment{mat}
{
	\left(\begin{matrix}
	}{
	\end{matrix}\right)
}

\newcommand{\adj}{\text{adj}}
\newcommand{\abs}[1]{\left|#1\right|}

\newcommand{\vek}[1]{\overrightarrow{#1}}

\newcommand\varpm{\mathbin{\vcenter{\hbox{%
				\oalign{\hfil$\scriptstyle+$\hfil\cr
					\noalign{\kern-.5ex}					
					$\scriptscriptstyle({-})$\cr}%
			}}}}

\newcommand{\taskref}[1]{\textbf{#1}}

\newcommand{\chapter}[2]{\section*{Kapitel #1: #2}}

\newcommand{\ans}{\textbf{Svar: }}
\newcommand{\mproof}{\text{~~~~V.S.V}}
\newcommand{\proof}{~~~~V.S.V}

\DeclareMathOperator{\lra}{\Leftrightarrow}
\DeclareMathOperator{\ra}{\Rightarrow}
\DeclareMathOperator{\sign}{sign}


\let\*\relax
\DeclareMathOperator{\*}{\cdot}

\let\Re\relax
\let\Im\relax
\DeclareMathOperator{\Re}{Re}
\DeclareMathOperator{\Im}{Im}

\title{Tillämpad matematik - Linjära system\\ FMAF10} 
\author{Emil Wihlander\\ dat15ewi@student.lu.se} 

\begin{document} 
\maketitle
\pagebreak
\chapter{1}{Svängningar och komplexa tal}
\begin{task}{1.1 a)}
Allmänna funktionen för odämpad harmonisk svängning är $u(t)=A\sin(\omega t + \alpha)$ där $\omega$ är vinkelfrekvensen.
\[u(t)=3\sin(2t-5) \ra \omega=2\]
\[T=\frac{2\pi}{\omega} \ra T = \frac{2\pi}{2} = \pi\]
\[f=\frac{1}{T} \ra f = \frac{1}{\pi}\]
\ans vinkelfrekvens: 2, period: $\pi$, frekvens: $\frac{1}{\pi}$
\end{task}

\begin{task}{b)}
	Allmänna funktionen för odämpad harmonisk svängning är $u(t)=A\sin(\omega t + \alpha)$ där $\omega$ är vinkelfrekvensen.
	\[u(t)=50\sin(100\pi t+1) \ra \omega=100\pi\]
	\[T=\frac{2\pi}{\omega} \ra T = \frac{2\pi}{100\pi} = \frac{1}{50}\]
	\[f=\frac{1}{T} \ra f = 50\]
	\ans vinkelfrekvens: $100\pi$, period: $\frac{1}{50}$, frekvens: 50
\end{task}

\begin{task}{1.2 a)}
\end{task}

\begin{task}{b)}
\end{task}

\begin{task}{c)}
\end{task}

\begin{task}{d)}
\end{task}

\begin{task}{e)}
\end{task}

\begin{task}{f)}
\end{task}

\begin{task}{1.3}
	Använd regeln $\sin(\alpha+\beta)=\sin\alpha\cos\beta+\cos\alpha\sin\beta$ från formelbladet.
	\begin{align*}
	u(t)= &
	6\sin(3t+\frac{\pi}{4})=
	6(\sin(3t)\cos(\frac{\pi}{4})+\cos(3t)\sin(\frac{\pi}{4}))= \\ =
	& 6\frac{1}{\sqrt{2}}\sin(3t)+6\frac{1}{\sqrt{2}}\cos(3t)=
	3\sqrt{2}\cos(3t)+3\sqrt{2}\sin(3t)
	\end{align*}
	\ans $a=b=3\sqrt{2}, \omega=3 \ra 3\sqrt{2}\cos(3t)+3\sqrt{2}\sin(3t)$
\end{task}

\pagebreak
\begin{task}{1.4 a)}
	låt $u(t)=A\sin(\omega t + \alpha)=A\sin\alpha\cos(\omega t)+A\cos\alpha\sin(\omega t)=\sqrt{3}\cos(\omega t)-\sin(\omega t)$ där $A$ är amplituden och $\alpha$ är fasförskjutningen.
	\begin{align*}
		\begin{linsys}{rr}
		A\sin\alpha=&\sqrt{3} \\
		A\cos\alpha=&-1
		\end{linsys} \lra &
		\sqrt{(A\sin\alpha)^2+(A\cos\alpha)^2}=\sqrt{(\sqrt{3})^2+(-1)^2} \lra \\ \lra
		& \sqrt{A^2}\sqrt{\sin\alpha^2+\cos\alpha^2}=\sqrt{4} \ra
		A\sqrt{1}=2 \lra
		A=2
	\end{align*}
	\begin{align*}
		\tan\alpha= &
		\frac{\sin\alpha}{\cos\alpha}=
		\frac{A\sin\alpha}{A\cos\alpha}=
		\frac{\sqrt{3}}{-1} \ra \\ \ra
		\alpha= &
		\arctan\frac{\sqrt{3}}{1}+\frac{\pi}{2}=
		\frac{\pi}{6}+\frac{\pi}{2}=
		\frac{2\pi}{3}
	\end{align*}
	eller:
	\begin{align*}
		u(t)= &
		\sqrt{3}\cos(\omega t)-\sin(\omega t)=
		2(\frac{\sqrt{3}}{2}\cos(\omega t)-\frac{1}{2}\sin(\omega t))= \\ =
		& 2(\sin\frac{2\pi}{3}\cos(\omega t)+\cos\frac{2\pi}{3}\sin(\omega t))=
		\sin(\omega t + \frac{2\pi}{3})
	\end{align*}
	\ans Amplitud: $2$ och fasförskjutning: $\frac{2\pi}{3}$
\end{task}
\pagebreak
\chapter{2}{Steg och impulsfunktioner}
\begin{task}{2.1 a)}
\end{task}

\begin{task}{b)}
\end{task}

\begin{task}{c)}
\end{task}

\begin{task}{d)}
\end{task}

\begin{task}{e)}
\end{task}

\begin{task}{2.2}
\end{task}

\begin{task}{2.3 a)}
	\[\theta(t-1)\theta(3-t)\]
	eller
	\[\theta(t-1)-\theta(t-3)\]
\end{task}

\begin{task}{b)}
	Funktionen som syns är $-0.5t+1.5$, stegfunktioner som skärmar in $]1,3[$ är (se \taskref{a)}) $\theta(t-1)-\theta(t-3)$ vilket medför:
	
	\ans $(-0.5t+1.5)(\theta(t-1)-\theta(t-3))$
\end{task}

\begin{task}{2.4 a)}
	Funktionen i intervallet $]0,1[$ är $t$. Stegfunktion: $\theta(t)-\theta(t-1)$.
	
	Funktionen i intervallet $]1,2[$ är $1$. Stegfunktion: $\theta(t-1)-\theta(t-2)$.
	
	Funktionen i intervallet $]2,3[$ är $3-t$. Stegfunktion: $\theta(t-2)-\theta(t-3)$ vilket ger:
	
	\ans $t(\theta(t)-\theta(t-1))+\theta(t-1)-\theta(t-2)+(3-t)(\theta(t-2)-\theta(t-3))$
\end{task}

\begin{task}{b)}
	Funktionen i intervallet $]0,1[$ är $t$. Stegfunktion: $\theta(t)-\theta(t-1)$.
	
	Funktionen i intervallet $]1,2[$ är $t-1$. Stegfunktion: $\theta(t-1)-\theta(t-2)$ vilket ger:
	
	\ans $t(\theta(t)-\theta(t-1))+(t-1)(\theta(t-1)-\theta(t-2))$
\end{task}

\pagebreak
\begin{task}{2.5}
	\[p_b(t)=\frac{1}{b}(\theta(t)-\theta(t-b))\]
	Om stegfunktioner finns som en faktor i en integral kan dessa ersätta integrationsgränserna eftersom de evaluerar till noll utanför intervallet.
	\[\int_{-\infty}^{+\infty}\! (\theta(t-a)-\theta(t-b))t \, dt =
	\int_{a}^{b}\! t \, dt\]
	Lös med hjälp av ovanstående samband:
	\begin{align*}
	\int_{-\infty}^{+\infty}\! p_b(t)e^{-st} \, dt =
	&\int_{-\infty}^{+\infty}\!\frac{1}{b}(\theta(t)-\theta(t-b))e^{-st} \, dt =
	\int_{0}^{b}\!\frac{1}{b}e^{-st} \, dt = \\ =
	&\left[-\frac{1}{sb}e^{-st}\right]_{0}^{b}=
	-\frac{e^{-sb}}{sb}-\left(-\frac{1}{sb}\right)=
	\frac{1-e^{-sb}}{sb}\cond{s\neq0}
	\end{align*}
	Om $s=0$:
	\[\int_{-\infty}^{+\infty}\! p_b(t)\*1 \, dt =1 \qquad\text{enligt def., se boken}\]'
	\ans $\int_{-\infty}^{+\infty}\! p_b(t)e^{-st} \, dt =\frac{1}{sb}(1-e^{-sb})\cond{s\neq0}$ och $1\cond{s=0}$
\end{task}

\begin{task}{2.6}
	Räknelag (se boken s. 21):
	\[\int_{-\infty}^{+\infty}\! \delta(t-a)f(t) \, dt =f(a)\cond{\text{om } f \text{ är kontinuerlig i }t=a}\]
	Eftersom $e^{-st}$ är kontinuerlig för alla $t$ använd räknelagen:
	\[\int_{-\infty}^{+\infty}\! \delta(t-a)e^{-st} \, dt =
	e^{-sa}\]
	\ans $e^{-sa}$
\end{task}

\begin{task}{2.7}
	Räknelag (se boken s. 21):
	\[\frac{d}{dt}(\theta(t-a))=\delta(t-a)\]
	Använd räknelagen:
	\[\frac{d}{dt}p_b=
	\frac{1}{b}\frac{d}{dt}(\theta(t)-\theta(t-b))=
	\frac{1}{b}\left(\frac{d}{dt}\theta(t)-\frac{d}{dt}\theta(t-b)\right)=
	\frac{1}{b}\left(\delta(t)-\delta(t-b)\right)\]
	\ans $\frac{1}{b}\left(\delta(t)-\delta(t-b)\right)$
\end{task}

\begin{task}{2.8 a)}
	Räknelag (se boken s. 21):
	\[f(t)\delta(t)=f(0)\delta(t)\cond{\text{om } f \text{ är kontinuerlig i }t=0}\]
	Låt $f(t)=t$, eftersom $t$ är kontinuerlig använd räknelagen:
	\[t\delta(t)=
	f(t)\delta(t)=
	f(0)\delta(t)=
	0\*\delta(t)=
	0\]
	\ans $0$
\end{task}

\begin{task}{b)}
	Räknelag (se boken s. 21):
	\[f(t)\delta(t-a)=f(a)\delta(t-a)\cond{\text{om } f \text{ är kontinuerlig i }t=a}\]
	Låt $f(t)=t$, eftersom $t$ är kontinuerlig använd räknelagen:
	\[t\delta(t-1)=
	f(t)\delta(t-1)=
	f(1)\delta(t-1)=
	1\*\delta(t-1)=
	\delta(t-1)\]
	\ans $\delta(t-1)$
\end{task}

\begin{task}{c)}
	Räknelag (se boken s. 21):
	\[f(t)\delta(t-a)=f(a)\delta(t-a)\cond{\text{om } f \text{ är kontinuerlig i }t=a}\]
	Låt $f(t)=e^{-t}$, eftersom $e^{-t}$ är kontinuerlig använd räknelagen:
	\[e^{-t}\delta(t-2)=
	f(t)\delta(t-2)=
	f(2)\delta(t-2)=
	e^{-2}\delta(t-1)\]
	\ans $e^{-2}\delta(t-1)$
\end{task}

\begin{task}{d)}
	Räknelag (se boken s. 21):
	\[f(t)\delta(t-a)=f(a)\delta(t-a)\cond{\text{om } f \text{ är kontinuerlig i }t=a}\]
	Låt $f(t)=\sin t$, eftersom $\sin t$ är kontinuerlig använd räknelagen:
	\[\sin t\delta(t-\pi)=
	f(t)\delta(t-\pi)=
	f(\pi)\delta(t-\pi)=
	0\*\delta(t-\pi)=
	0\]
	\ans $0$
\end{task}

\begin{task}{2.9}
	Använd sats 2.1 (s. 22):
	\[f(t)= t^2(\theta(t)-\theta(t-1))+(2-t)(\theta(t-1)-\theta(t-2))\]
	Eftersom funktionen saknar språng är ($\frac{d}{dt}t^2=2t$ och $\frac{d}{dt}(2-t)=-1$):
	\[f'(t)=f_p'(t)=2t(\theta(t)-\theta(t-1))-(\theta(t-1)-\theta(t-2))\]
	Eftersom $f'(t)$ har språng i $t=1$ och $t=2$ måste storleken på dessa beräknas (högra funktionen minus den vänstra):
	\[t=1 \ra (-1)-2\*1=-3\]
	\[t=2 \ra 0-(-1)=1\]
	$f'(t)$ är deriverbar i alla punkter utom $t=\{0,1,2\}$, $t=0$ saknar dock språng.
	\[f''(t)=f_p''(t)+b_1\delta(t-a_1)+b_2\delta(t-a_2) \qquad\text{där}\qquad a_1=1,\;b_1=-3,\;a_2=2,\;b_2=1\]
	$\frac{d}{dt}2t=2$ och $\frac{d}{dt}(-1)=0$:
	\begin{align*}
	f''(t)=
	&2(\theta(t)-\theta(t-1))+0\*(\theta(t-1)-\theta(t-2))-3\delta(t-1)+1\*\delta(t-2)= \\ =
	&2(\theta(t)-\theta(t-1))-3\delta(t-1)+\delta(t-2)
	\end{align*}
	\ans 
	\[f'(t)=2t(\theta(t)-\theta(t-1))-(\theta(t-1)-\theta(t-2))\]
	\[f''(t)=2(\theta(t)-\theta(t-1))-3\delta(t-1)+\delta(t-2)\]
\end{task}

\begin{task}{2.10 a)}
	Sinus med amplitud 2 och vinkelfrekvensen 2, samt från 0 till $\pi/2$:
	\[f(t)=2\sin 2t(\theta(t)-\theta(t-\pi/2))\]
\end{task}

\begin{task}{b)}
	Använd sats 2.1 (s. 22):
	\[f(t)= 2\sin 2t(\theta(t)-\theta(t-\pi/2))\]
	Eftersom funktionen saknar språng är ($\frac{d}{dt}2\sin 2t=4\cos2t$):
	\[f'(t)=f_p'(t)=4\cos2t(\theta(t)-\theta(t-\pi/2))\]
	Eftersom $f'(t)$ har språng i $t=0$ och $t=\pi/2$ måste storleken på dessa beräknas (högra funktionen minus den vänstra):
	\[t=0 \ra 4\cos(2\*0)-0=4\]
	\[t=\pi/2 \ra 0-4\cos(2\*\pi/2)=4\]
	$f'(t)$ är deriverbar i alla punkter utom $t=\{0,\pi/2\}$.
	\[f''(t)=f_p''(t)+b_1\delta(t-a_1)+b_2\delta(t-a_2) \qquad\text{där}\qquad a_1=0,\;b_1=4,\;a_2=\pi/2,\;b_2=4\]
	$\frac{d}{dt}4\cos2t=-8\sin2t$:
	\begin{align*}
	f''(t)=
	&-8\sin2t(\theta(t)-\theta(t-\pi/2))+4\delta(t)+4\delta(t-\pi/2)
	\end{align*}
	\ans 
	\[f'(t)=4\cos2t(\theta(t)-\theta(t-\pi/2))\]
	\[f''(t)=-8\sin2t(\theta(t)-\theta(t-\pi/2))+4\delta(t)+4\delta(t-\pi/2)\]
\end{task}

\begin{task}{2.11}
	Beskriv $\abs{x}$ med hjälp av stegfunktioner:
	\[f(x) = \abs{x} = -x(1-\theta(x))+x\theta(x)\]
	Eftersom funktionen saknar språng är ($\frac{d}{dx}x=1$):
	\[f'(x)=f_p'(x) = -1\*(1-\theta(x))+1\*\theta(x) = -1+\theta(x)+\theta(x) = 2\theta(x)-1\]
	Eftersom $f'(t)$ har språng i $t=0$ måste storleken på denna beräknas (högra funktionen minus den vänstra):
	\[x=0 \ra (2-1)-(-1)=2\]
	$f'(t)$ är deriverbar i alla punkter utom $t=0$.
	\[f''(x)=f_p''(x)+b\delta(x-a)\qquad\text{där}\qquad a=0,\;b=2\]
	$\frac{d}{dt}1=0$:
	\[f''(x) = -0\*(1-\theta(x))+0\*\theta(x)+2\delta(x-0)=2\delta(x)\]
	\ans 
	\[f'(t)=2\theta(x)-1\]
	\[f''(t)=2\delta(x)\]
\end{task}

\begin{task}{2.12}
	Använd sambandet på s. 17:
	\begin{align*}
	v(t)=
	&\int_{-\infty}^{t} \! \tau^a\theta(\tau) \, d\tau =
	\begin{linsys}{ll}
		0,& t\le0 \\
		\displaystyle\int_0^t\! \tau^a \, d\tau,\quad & t>0
	\end{linsys}=
	\left(\int_0^t\! \tau^a \, d\tau\right)\theta(t)= \\ =
	&\left(\left[\frac{\tau^{a+1}}{a+1}\right]_0^t\right)\theta(t)=
	\left(\frac{t^{a+1}}{a+1}-\frac{0^{a+1}}{a+1}\right)\theta(t)=
	\frac{t^{a+1}}{a+1}\theta(t)\cond{a>-1}
	\end{align*}
	\ans $\frac{t^{a+1}}{a+1}\theta(t)\cond{a>-1}$
\end{task}
\pagebreak
\chapter{3}{Laplacetransformationer}

\begin{task}{3.1 a)}
	$f(t)=e^{-2t}\theta(t)$
	
	Se definitionen av Lapacetransformen i boken.
	\begin{align*}
	\laplace f(s)=
	&\int_{-\infty}^{+\infty}\! e^{-st}f(t) \, dt=
	\int_{-\infty}^{+\infty}\! e^{-st}e^{-2t}\theta(t) \, dt=
	\int_{0}^{+\infty}\! e^{-(s+2)t} \, dt=
	\left[-\frac{e^{-(s+2)t}}{s+2}\right]_0^{+\infty}= \\ =
	&\lim\limits_{T\rightarrow\infty}\frac{1}{s+2}(1-e^{-(s+2)T})
	\end{align*}
	Om $s>-2$ gäller att $e^{-(s+2)T} \rightarrow 0$ när $T \rightarrow \infty$ vilket medför:
	\[\laplace f(s)=\frac{1}{s+2}\cond{s>-2}\]
	Om $s=-2$:
	\[\laplace f(s)=
	\int_{0}^{+\infty}\! 1 \, dt=
	\left[t\right]_0^{+\infty}\rightarrow\infty\]
	Om $s<-2$ gäller att $e^{-(s+2)T} \rightarrow \infty$ när $T \rightarrow \infty$ vilket medför:
	\[\laplace f(s)\rightarrow-\infty\]
	Detta medför att $\laplace f(s)$ endast är konvergent när $s>-2$ och därmed är Laplacetransformen för $e^{-2t}\theta(t)$ endast definierad i det intervallet.
	
	Låt nu $s$ vara ett komplext tal, $s=a+bi$:
	\[\abs{e^{-(a+bi+2)t}}=
	\abs{e^{-(a+2)t}}\underbrace{\abs{e^{-ibt}}}_{=1}=
	e^{-(a+2)t}\]
	Här ser vi att $e^{-(s+2)t}\rightarrow 0$ då $t\rightarrow \infty$ om $a = \Re s > -2$ vilket utvidgar Laplacetransformen att inkludera hela planet $\Re s > -2$.
	
	\ans $\laplace f(s)= \dfrac{1}{s+2}\cond{\Re s > -2}$
\end{task}

\begin{task}{3.1 b)}
	$f(t)=\theta(t)-\theta(t-1)$
	
	Se definitionen av Lapacetransformen i boken.
	\begin{align*}
	\laplace f(s)=
	&\int_{-\infty}^{+\infty}\! e^{-st}f(t) \, dt=
	\int_{-\infty}^{+\infty}\! e^{-st}(\theta(t)-\theta(t-1)) \, dt=
	\int_{0}^{1}\! e^{-st} \, dt=
	\left[-\frac{e^{-st}}{s}\right]_0^{1}= \\ =
	&-\frac{e^{-s}}{s}+\frac{1}{s}=
	\frac{1-e^{-s}}{s}\cond{s\neq0}
	\end{align*}
	Om $s=0$ gäller att $e^{-st} = 1$ vilket medför:
	\[\laplace f(0)=
	\int_{0}^{1}\! 1 \, dt=
	\left[t\right]_0^1=
	1-0=1\]
	
	\ans $\laplace f(s)= \dfrac{1}{s}(1-e^{-s})\cond{s\neq0}$ och $\laplace f(0)= 1$
\end{task}
\pagebreak
\chapter{4}{Den inversa Laplacetransformen}

\begin{task}{4.1}
	multiplicera faktorerna.
	\begin{align*}
	F(s)=
	&\frac{(s-1)(s+1)}{(s+2)^2(s+1+2i)(s+1-2i)}=
	\frac{s^2-1}{(s^2+2s+4)((s+1)^2+4)}= \\ =
	&\frac{s^2-1}{(s^2+4s+4)(s^2+2s+5)}=
	\frac{s^2-1}{s^4+6s^3+17s^2+28s+20}
	\end{align*}
\end{task}

\begin{task}{4.2 a)}
	Använd tabellen:
	\[\laplace f(s)=\frac{1}{s}\llra \theta(t)=f(t)\]
	\ans $f(t)=\theta(t)$
\end{task}

\begin{task}{b)}
	Faktorisera och använd tabellen:
	\[\laplace f(s)=\frac{1}{s^3}=\frac{1}{2}\*\frac{2}{s^{2+1}}\llra \frac{1}{2}t^2\theta(t)=f(t)\]
	\ans $f(t)=\dfrac{1}{2}t^2\theta(t)$
\end{task}

\begin{task}{c)}
	Faktorisera och använd tabellen:
	\begin{align*}
	\laplace f(s)=
	&\frac{s^4+6s^3-10s^2+1}{s^5}=
	\frac{1}{s}+6\frac{1}{s^2}-10\frac{1}{s^3}+\frac{1}{s^5}=
	\frac{1}{s}+6\frac{1}{s^2}-5\frac{2!}{s^3}+\frac{1}{4!}\*\frac{4!}{s^5}\llra \\ \llra
	&\theta(t)+6t\theta(t)-5t^2\theta(t)+\frac{1}{24}t^4\theta(t)=
	(1+6t-5t^2+\frac{1}{24}t^4)\theta(t)=
	f(t)
	\end{align*}
	\ans $f(t)=(1+6t-5t^2+\frac{1}{24}t^4)\theta(t)$
\end{task}

\begin{task}{4.2 a)}
	Faktorisera och använd tabellen:
	\[\laplace f(s)=\frac{2}{s+3}=2\frac{0!}{(s-(-3))^{0+1}}\llra 2e^{-3t}\theta(t)=f(t)\]
	\ans $f(t)=2e^{-3t}\theta(t)$
\end{task}

\begin{task}{b)}
	Faktorisera och använd tabellen:
	\begin{align*}
	\laplace f(s)=
	&\frac{1}{s+3}-\frac{2}{(s+3)^2}+\frac{1}{(s+3)^3}=
	\frac{1}{s+3}-2\frac{1}{(s+3)^2}+\frac{1}{2}\*\frac{2}{(s+3)^3}\llra \\ \llra
	&e^{-3t}\theta(t)-2te^{-3t}\theta(t)+\frac{1}{2}t^2e^{-3t}\theta(t)=
	(1-2t+\frac{1}{2}t^2)e^{-3t}\theta(t)=
	f(t)
	\end{align*}
	\ans $f(t)=(1-2t+\frac{1}{2}t^2)e^{-3t}\theta(t)$
\end{task}

\begin{task}{c)}
	Faktorisera och använd tabellen:
	\begin{align*}
	\laplace f(s)=
	&\frac{s+5}{(s+3)^2}=
	\frac{s+3+2}{(s+3)^2}=
	\frac{s+3}{(s+3)^2}+\frac{2}{(s+3)^2}=
	\frac{1}{s+3}+2\frac{1}{(s+3)^2}\llra \\ \llra
	&e^{-3t}\theta(t)+2te^{-3t}\theta(t)=
	(1+2t)e^{-3t}\theta(t)=
	f(t)
	\end{align*}
	\ans $f(t)=(1+2t)e^{-3t}\theta(t)$
\end{task}

\begin{task}{4.4 a)}
	Faktorisera och använd tabellen:
	\begin{align*}
	\laplace f(s)=
	&\frac{s}{s^2+6s+8}=
	\frac{s}{(s+4)(s+2)}=
	\frac{2}{s+4}-\frac{1}{s+2}=
	2\frac{1}{s+4}-\frac{1}{s+2}\llra  \\ \llra
	&2e^{-4t}\theta(t)-e^{-2t}\theta(t)=
	(2e^{-4t}-e^{-2t})\theta(t)=
	f(t)
	\end{align*}
	\ans $f(t)=(2e^{-4t}-e^{-2t})\theta(t)$
\end{task}

\begin{task}{b)}
	Kvadratkomplettera, faktorisera och använd tabellen:
	\begin{align*}
	\laplace f(s)=
	&\frac{s}{s^2+6s+10}=
	\frac{s}{(s+3)^2+1}=
	\frac{s+3}{(s+3)^2+1}-3\frac{1}{(s+3)^2+1}
	\end{align*}
	Använd dämpningsregeln:
	\[\frac{s+3}{(s+3)^2+1}\llra e^{-3t}\cos(t)\theta(t)\]
	\[\frac{1}{(s+3)^2+1}\llra e^{-3t}\sin(t)\theta(t)\]
	\[f(t)=e^{-3t}\cos(t)\theta(t)-3e^{-3t}\sin(t)\theta(t)=(\cos t-3\sin t)e^{-3t}\theta(t)\]
	\ans $f(t)=(\cos t-3\sin t)e^{-3t}\theta(t)$
\end{task}

\begin{task}{4.5 a)}
	Använd tabellen:
	\begin{align*}
	\laplace f(s)=
	&\frac{1}{s^2+16}=
	\frac{1}{4}\*\frac{4}{s^2+4^4} \llra 
	\frac{1}{4}\sin(4t)\theta(t)=f(t)
	\end{align*}
	\ans $f(t)=\frac{1}{4}\sin(4t)\theta(t)$
\end{task}

\begin{task}{b)}
	Använd tabellen:
	\begin{align*}
	\laplace f(s)=
	&\frac{s}{s^2+16}=
	\frac{s}{s^2+4^4} \llra 
	\cos(4t)\theta(t)=f(t)
	\end{align*}
	\ans $f(t)=\cos(4t)\theta(t)$
\end{task}

\begin{task}{c)}
	Kvadratkomplettera och använd dämpningsregeln och tabellen:
	\begin{align*}
	\laplace f(s)=
	&\frac{1}{s^2+4s+8}=
	\frac{1}{(s+2)^2+2^2}=
	\frac{1}{2}\*\frac{2}{(s+2)^2+2^2} \llra \\ \llra
	&\frac{1}{2}e^{-2t}\sin(2t)\theta(t)=
	f(t)
	\end{align*}
	\ans $f(t)=\frac{1}{2}e^{-2t}\sin(2t)\theta(t)$
\end{task}

\begin{task}{d)}
	Kvadratkomplettera, faktorisera och använd dämpningsregeln och tabellen:
	\begin{align*}
	\laplace f(s)=
	&\frac{s}{s^2+4s+8}=
	\frac{s}{(s+2)^2+2^2}=
	\frac{s+2}{(s+2)^2+2^2}-\frac{2}{(s+2)^2+2^2} \llra \\ \llra
	&e^{-2t}\cos(2t)\theta(t)-e^{-2t}\sin(2t)\theta(t)=
	(\cos(2t)-\sin(2t))e^{-2t}\theta(t)=
	f(t)
	\end{align*}
	\ans $f(t)=(\cos(2t)-\sin(2t))e^{-2t}\theta(t)$
\end{task}
\pagebreak
\chapter{5}{Lösning av differentialekvationer genom Laplacetransformation}

\begin{task}{5.1}
	Låt $Y(s)=\laplace(\theta y)(s)$ vilket medför att:
	\[\laplace(\theta y')(s)=s\laplace(\theta y)(s)-y(0)=sY(s)-1\]
	\[\laplace(\theta y'')(s)=s\laplace(\theta y')(s)-y'(0)=s(sY(s)-1)-2=s^2Y(s)-s-2\]
	Multiplicera ekvationen med $\theta(t)$ och hitta Laplacetransformparet:
	\[y''(t)\theta(t)+2y'(t)\theta(t)+5y(t)\theta(t)=e^{-t}\theta(t) \llra
	s^2Y(s)-s-2+2(sY(s)-1)+5Y(s)=\frac{1}{s+1}\]
	\begin{align*}
	&s^2Y(s)-s-2+2(sY(s)-1)+5Y(s)=\frac{1}{s+1} \lra
	(s^2+2s+5)Y(s)=\frac{1}{s+1}+s+4 \lra \\ \lra
	&Y(s)=
	\frac{1+s(s+1)+4(s+1)}{(s+1)(s^2+2s+5)}=
	\frac{s^2+5s+5}{(s+1)((s+1)^2+4)}=
	\frac{k_1}{s+1}+\frac{k_2s+k_3}{(s+1)^2+4}
	\end{align*}
	Identifiera variablerna:
	\begin{align*}
	&k_1((s+1)^2+4)+(k_2s+k_3)(s+1)=s^2+5s+5 \lra \\ \lra
	&k_1(s^2+2s+5)+k_2(s^2+s)+k_3(s+1)=s^2+5s+5 \lra \\ \lra
	&\begin{cases}
	k_1+k_2=1 \\
	2k_1+k_2+k_3=5 \\
	5k_1+k_3=5
	\end{cases} \lra
	\begin{cases}
	k_1=\frac{1}{4} \\
	k_2=\frac{3}{4} \\
	k_3=\frac{15}{4}
	\end{cases}
	\end{align*}
	\begin{align*}
	Y(s)=
	&\frac{1}{4}\left(\frac{1}{s+1}+\frac{3s+15}{(s+1)^2+4}\right)=
	\frac{1}{4}\left(\frac{1}{s+1}+3\frac{s+1}{(s+1)^2+2^2}+6\frac{2}{(s+1)^2+2^2}\right) \llra \\ \llra
	&\frac{1}{4}\left(e^{-t}\theta(t)+3e^{-t}\cos(2t)\theta(t)+6e^{-t}\sin(2t)\theta(t)\right)=
	\frac{e^{-t}}{4}\left(1+3\cos(2t)+6\sin(2t)\right)\theta(t)
	\end{align*}
	\[y(t)\theta(t)=\frac{e^{-t}}{4}\left(1+3\cos(2t)+6\sin(2t)\right)\theta(t)\]
	\[y(t)=\frac{e^{-t}}{4}\left(1+3\cos(2t)+6\sin(2t)\right)\]
	Eftersom $y(0)=\frac{e^0}{4}(1+3\cos(0)+6\sin(0))=1$ är funktionen, utöver $t>0$, definierad för $t=0$.
	
	\ans $y(t)=\frac{e^{-t}}{4}\left(1+3\cos(2t)+6\sin(2t)\right)\cond{t \ge 0}$
\end{task}

\begin{task}{5.2}
	Låt $Y(s)=\laplace(\theta y)(s)$ vilket medför att:
	\[\laplace(\theta y')(s)=s\laplace(\theta y)(s)-y(0)=sY(s)\]
	\[\laplace(\theta y'')(s)=s\laplace(\theta y')(s)-y'(0)=s^2Y(s)-1\]
	Multiplicera vänsterledet med $\theta(t)$ och hitta Laplacetransformparet:
	\[y''(t)\theta(t)-2y'(t)\theta(t)+2y(t)\theta(t) \llra
	s^2Y(s)-s-2+2(sY(s)-1)+5Y(s)=\frac{1}{s+1}\]
	\begin{align*}
	&s^2Y(s)-1-2sY(s)+2Y(s)=0 \lra
	(s^2-2s+2)Y(s)=1 \lra \\ \lra
	&Y(s)=
	\frac{1}{s^2-2s+2}=
	\frac{1}{(s-1)^2+1} \llra
	e^{t}\sin(t)\theta(t)
	\end{align*}
	\[y(t)\theta(t)=e^{t}\sin(t)\theta(t)\]
	\[y(t)=e^{t}\sin(t)\]
	Eftersom $y(0)=e^{0}\sin(0)=0$ är funktionen, utöver $t>0$, definierad för $t=0$.
	
	\ans $y(t)=e^{t}\sin(t)\cond{t \ge 0}$
\end{task}

\begin{task}{5.3}
	Låt $Y(s)=\laplace(\theta y)(s)$ vilket medför att:
	\[\laplace(\theta y')(s)=s\laplace(\theta y)(s)-y(0)=sY(s)\]
	\[\laplace(\theta y'')(s)=s\laplace(\theta y')(s)-y'(0)=s^2Y(s)\]
	Multiplicera ekvationen med $\theta(t)$ och hitta Laplacetransformparet:
	\[y''(t)\theta(t)+2y'(t)\theta(t)+y(t)\theta(t)=e^{-2t}\theta(t) \llra
	s^2Y(s)+2sY(s)+Y(s)=\frac{1}{s+2}\]
	\begin{align*}
	&s^2Y(s)+2sY(s)+Y(s)=\frac{1}{s+2} \lra
	(s^2+2s+1)Y(s)=\frac{1}{s+2} \lra \\ \lra
	&Y(s)=
	\frac{1}{(s+2)(s^2+2s+1)}=
	\frac{1}{(s+2)(s+1)^2}=
	\frac{k_1}{s+2}+\frac{k_2}{s+1}+\frac{k_3}{(s+1)^2}
	\end{align*}
	Identifiera variablerna:
	\begin{align*}
	&k_1(s+1)^2+k_2(s+2)(s+1)+k_3(s+2)=1 \lra \\ \lra
	&k_1(s^2+2s+1)+k_2(s^2+3s+2)+k_3(s+2)=1 \lra \\ \lra
	&\begin{cases}
	k_1+k_2=0 \\
	2k_1+3k_2+k_3=0 \\
	k_1+2k_2+2k_3=1
	\end{cases} \lra
	\begin{cases}
	k_1=1 \\
	k_2=-1 \\
	k_3=1
	\end{cases}
	\end{align*}
	\begin{align*}
	Y(s)=
	\frac{1}{s+2}-\frac{1}{s+1}+\frac{1}{(s+1)^2}\llra
	&e^{-2t}\theta(t)-e^{-t}\theta(t)+e^{-t}t\theta(t)= \\ =
	&(e^{-2t}+e^{-t}(t-1))\theta(t)
	\end{align*}
	\[y(t)\theta(t)=(e^{-2t}+e^{-t}(t-1))\theta(t)\]
	\[y(t)=e^{-2t}+e^{-t}(t-1)\]
	Eftersom $y(0)=e^{0}+e^{0}(0-1)=0$ är funktionen, utöver $t>0$, definierad för $t=0$.
	
	\ans $y(t)=e^{-2t}+e^{-t}(t-1)\cond{t \ge 0}$
\end{task}

\pagebreak
\begin{task}{5.4}
	Låt $Y_1(s)=\laplace(\theta y_1)(s)$ och $Y_2(s)=\laplace(\theta y_2)(s)$ vilket medför att:
	\[\laplace(\theta y_1')(s)=s\laplace(\theta y_1)(s)-y_1(0)=sY_1(s)-1\]
	\[\laplace(\theta y_2')(s)=s\laplace(\theta y_2)(s)-y_2(0)=sY_2(s)\]
	Multiplicera ekvationssystemet med $\theta(t)$:
	\[
	\begin{cases}
	y_1\theta(t)-2y_2'\theta=2\theta \\
	y_1'\theta(t)+2y_2=-2t\theta
	\end{cases}\]
	Laplacetransformera båda ekvationerna:
	\[
	\begin{cases}
	Y_1(s)-2sY_2(s)=\frac{2}{s} \\
	sY_1(s)-1+2Y_2(s)=-\frac{2}{s^2}
	\end{cases} \lra
	\begin{linsys}{rrr}
	 Y_1(s)-&2sY_2(s)=&\frac{2}{s} \\
	sY_1(s)+& 2Y_2(s)=&\frac{s^2-2}{s^2}
	\end{linsys}\]
	Använd Cramers regel:
	\begin{align*}
	\Delta(s)=\begin{detmat}
	1 & -2s \\
	s & 2
	\end{detmat}=
	2-(-2s)s=2(s^2+1)
	\end{align*}
	\begin{align*}
	Y_1(s)=
	\frac{1}{\Delta(s)}\begin{detmat}
	\frac{2}{s} & -2s \\
	\frac{s^2-2}{s^2} & 2
	\end{detmat}=
	\frac{1}{2(s^2+1)}\left(\frac{4}{s}-\frac{(-2s)(s^2-2)}{s^2}\right)=
	\frac{s}{s^2+1}
	\end{align*}
	\begin{align*}
	Y_2(s)=
	\frac{1}{\Delta(s)}\begin{detmat}
	1 &\frac{2}{s} \\
	s &\frac{s^2-2}{s^2}
	\end{detmat}=
	\frac{1}{2(s^2+1)}\left(\frac{s^2-2}{s^2}-2\right)=
	\frac{1}{2}\*\frac{1}{s^2+1}-\frac{1}{s^2}
	\end{align*}
	\[\begin{cases}
	Y_1(s)=\frac{s}{s^2+1} \\
	Y_2(s)=\frac{1}{2}\*\frac{1}{s^2+1}-\frac{1}{s^2}
	\end{cases}\llra
	\begin{cases}
	y_1\theta=\cos(t)\theta(t) \\
	y_2\theta=\frac{1}{2}\sin(t)\theta(t)-t\theta(t)
	\end{cases}\]
	Eftersom $\theta(t)=1$ endast när $t>0$ måste det läggas till som villkor:
	\[\begin{cases}
	y_1=\cos(t) \\
	y_2=\frac{1}{2}\sin(t)-t
	\end{cases}\cond{t > 0}\]
	Eftersom $y_1(0)=\cos(0)=1$ och $y_2(0)=\frac{1}{2}\sin(0)-0=0$ är de de också definierade för $t=0$.
	
	\ans 
	$\begin{cases}
	y_1=\cos(t) \\
	y_2=\frac{1}{2}\sin(t)-t
	\end{cases}\cond{t \ge 0}$
\end{task}

\begin{task}{5.5}
	Eftersom differentialekvationens högerled är $\delta(t)$ låt $Y(s)=\laplace y(s)$ ($\delta(t)\theta(t)$ saknar betydelse)
	\[sY(s)=\laplace y'(s),\qquad s^2Y(s)=\laplace y''(s)\]
	Laplacetransformera ekvationen:
	\begin{align*}
	&s^2Y(s)+2sY(s)+2Y(s)=1 \lra
	(s^2+2s+2)Y(s)=1\lra \\ \lra
	&Y(s)=\frac{1}{s^2+2s+2}=
	\frac{1}{(s+1)^2+1}\llra
	e^{-t}\sin(t)\theta(t)
	\end{align*}
	Eftersom $\theta(t)$ är en faktor är $y$ kausal.
	
	\ans $y(t)=e^{-t}\sin(t)\theta(t)$
\end{task}

\begin{task}{5.6}
	För att använda Laplacetransformation med begynnelsevärden måste $t > 0$. Så börja med att hitta lösningen för det intervallet och utvidga lösningen efter.
	
	Låt $Y(s)=\laplace(\theta y)(s)$.
	\[\laplace(\theta y')(s)=s\laplace(\theta y)(s)-y(0)=sY(s)\]
	\[\laplace(\theta y'')(s)=s\laplace(\theta y')(s)-y'(0)=s^2Y(s)\]
	\[\laplace(\theta y^{(3)})(s)=s\laplace(\theta y'')(s)-y''(0)=s^3Y(s)\]
	\[\laplace(\theta y^{(4)})(s)=s\laplace(\theta y^{(3)})(s)-y^{(3)}(0)=s^4Y(s)-1\]
	Multiplicera ekvationen med $\theta(t)$ och Laplacetransformera den:
	\begin{align*}
	&s^4Y(s)-1=Y(s) \lra
	(s^4-1)Y(s)=1 \lra \\ \lra
	Y(s)=&\frac{1}{s^4-1}=
	\frac{1}{(s^2+1)(s+1)(s-1)}=
	\frac{k_1s+k_2}{s^2+1}+\frac{k_3}{s+1}+\frac{k_4}{s-1}
	\end{align*}
	Identifiera variablerna:
	\begin{align*}
	&(k_1s+k_2)(s+1)(s-1)+k_3(s^2+1)(s-1)+k_4(s^2+1)(s+1)=1 \lra \\ \lra
	&k_1(s^3-s)+k_2(s^2-1)+k_3(s^3-s^2+s-1)+k_4(s^3+s^2+s+1)=1 \lra \\ \lra
	&\begin{cases}
	k_1+k_3+k_4=0 \\
	k_2-k_3+k_4=0 \\
	-k_1+k_3+k_4=0 \\
	-k_2-k_3+k_4=1
	\end{cases} \lra
	\begin{cases}
	k_1=0 \\
	k_2=-\frac{1}{2} \\
	k_3=-\frac{1}{4} \\
	k_4=\frac{1}{4}
	\end{cases}
	\end{align*}
	\begin{align*}
	Y(s)=
	&-\frac{1}{2}\*\frac{1}{s^2+1}-\frac{1}{4}\frac{1}{s+1}+\frac{1}{4}\*\frac{1}{s-1}=
	\frac{1}{4}\left(\frac{1}{s-1}-\frac{1}{s+1}-2\frac{1}{s^2+1}\right) \llra \\ \llra
	&\frac{1}{4}(e^{t}\theta(t)-e^{-t}\theta(t)-\sin(t)\theta(t))=
	\frac{1}{4}(e^{t}-e^{-t}-2\sin(t))\theta(t)
	\end{align*}
	\[y(t)\theta(t)=\frac{1}{4}(e^{t}-e^{-t}-2\sin(t))\theta(t)\]
	\[y(t)=\frac{1}{4}(e^{t}-e^{-t}-2\sin(t))\cond{t>0}\]
	$y$ deriveras 3 gånger och då visar det sig att $y(0)=y'(0)=y''(0)=0$, $y^{(3)}(0)=1$ vilket innebär att begynnelsevillkoren är uppfyllda.
	
	Eftersom termerna är definierade för $t<0$ och deriveras på samma sätt för negativa värden kan intervallet för $t$ utvidgas till $t\in\mathbb{R}$.
	
	\ans $y(t)=\frac{1}{4}(e^{t}-e^{-t}-2\sin(t))\cond{t\in\mathbb{R}}$
\end{task}

\pagebreak
\begin{task}{5.7}
	Låt $Y(s)=\laplace(\theta y)(s)$.
	\[\laplace(\theta y')(s)=s\laplace(\theta y)(s)-y(0)=sY(s)\]
	\[\laplace(\theta y'')(s)=s\laplace(\theta y')(s)-y'(0)=s^2Y(s)\]
	\[\laplace(\theta y''')(s)=s\laplace(\theta y'')(s)-y''(0)=s^3Y(s)\]
	Multiplicera ekvationen med $\theta(t)$ och Laplacetransformera den:
	\begin{align*}
	&s^3Y(s)+3s^2Y(s)+3sY(s)+Y(s)=\frac{1}{(s+1)^2} \lra
	(s^3+3s^2+3s+1)Y(s)=\frac{1}{(s+1)^2} \lra \\ \lra
	&Y(s)=\frac{1}{(s+1)^2(s^3+3s^2+3s+1)}=
	\frac{1}{(s+1)^5}=
	\frac{1}{24}\*\frac{4!}{(s+1)^5} \llra
	\frac{1}{24}e^{-t}t^4\theta(t)
	\end{align*}
	\[y(t)\theta(t)=\frac{1}{24}e^{-t}t^4\theta(t)\]
	\[y(t)=\frac{1}{24}e^{-t}t^4\cond{t>0}\]
	$y$ deriveras 2 gånger och då visar det sig att $y(0)=y'(0)=y''(0)=0$ vilket innebär att begynnelsevillkoren är uppfyllda, funktionen är alltså definierad även för $t=0$.
	
	\ans $y(t)=\frac{1}{24}e^{-t}t^4\cond{t\ge0}$
\end{task}

\pagebreak
\begin{task}{5.8}
	Låt $Y(s)=\laplace(\theta y)(s)$.
	\[\laplace(\theta y')(s)=s\laplace(\theta y)(s)-y(0)=sY(s)\]
	\[\laplace(\theta y'')(s)=s\laplace(\theta y')(s)-y'(0)=s^2Y(s)\]
	Multiplicera ekvationen med $\theta(t)$ (notera $\theta(t-\alpha)\theta(t-\beta)=\theta(t-\alpha)\cond{\alpha \ge \beta}$):
	\[y''\theta(t)+3y'\theta(t)+2y\theta(t)=\theta(t)\theta(t)-\theta(t-1)\theta(t) \lra
	y''\theta(t)+3y'\theta(t)+2y\theta(t)=\theta(t)-\theta(t-1)\]
	Laplacetransformera den:
	\begin{align*}
	&s^2Y(s)+3sY(s)+2Y(s)=\frac{1}{s}-e^{-s}\frac{1}{s} \lra
	(s^2+3s+2)Y(s)=\frac{1-e^{-s}}{s} \lra \\ \lra
	&Y(s)=
	\frac{1-e^{-s}}{s(s^2+3s+2)}=
	\frac{k_1}{s}+\frac{k_2}{s+1}+\frac{k_3}{s+2}-e^{-s}\left(\frac{k_1}{s}+\frac{k_2}{s+1}+\frac{k_3}{s+2}\right)
	\end{align*}
	Identifiera variablerna:
	\begin{align*}
	&k_1(s+1)(s+2)+k_2s(s+2)+k_3s(s+1)=1 \lra \\ \lra
	&k_1(s^2+3s+2)+k_2(s^2+2s)+k_3(s^2+s)=1 \lra \\ \lra
	&\begin{cases}
	k_1+k_2+k_3=0 \\
	3k_1+2k_2+k_3=0 \\
	2k_1=1
	\end{cases} \lra
	\begin{cases}
	k_1=\frac{1}{2} \\
	k_2=-1 \\
	k_3=\frac{1}{2}
	\end{cases}
	\end{align*}
	\begin{align*}
	Y(s)=
	&\frac{1}{2}\*\frac{1}{s}-\frac{1}{s+1}+\frac{1}{2}\*\frac{1}{s+2}-e^{-s}\left(\frac{1}{2}\*\frac{1}{s}-\frac{1}{s+1}+\frac{1}{2}\*\frac{1}{s+2}\right)\llra \\ \llra
	&(\frac{1}{2}-e^{-t}+\frac{1}{2}e^{-2t})\theta(t)-(\frac{1}{2}-e^{-(t-1)}+\frac{1}{2}e^{-2(t-1)})\theta(t-1)= \\ =
	&\frac{1}{2}(1-2e^{-t}+e^{-2t})\theta(t)-\frac{1}{2}(1-2e^{-(t-1)}+e^{-2(t-1)})\theta(t-1)
	\end{align*}
	Notera att $\frac{\theta(t-\alpha)}{\theta(t)}=\theta(t-\alpha)\cond{t>0}$:
	\[y(t)\theta(t)=\frac{1}{2}(1-2e^{-t}+e^{-2t})\theta(t)-\frac{1}{2}(1-2e^{-(t-1)}+e^{-2(t-1)})\theta(t-1)\]
	\[y(t)=\frac{1}{2}(1-2e^{-t}+e^{-2t})\theta(t)-\frac{1}{2}(1-2e^{-(t-1)}+e^{-2(t-1)})\theta(t-1)\cond{t>0}\]
	$y$ deriveras och då visar det sig att $y(0)=y'(0)=0$ vilket innebär att begynnelsevillkoren är uppfyllda, funktionen är alltså definierad även för $t=0$.
	
	\ans $y(t)=\frac{1}{2}(1-2e^{-t}+e^{-2t})\theta(t)-\frac{1}{2}(1-2e^{-(t-1)}+e^{-2(t-1)})\theta(t-1)\cond{t\ge0}$
\end{task}

\pagebreak
\begin{task}{5.9}
	Låt $Y(s)=\laplace(\theta y)(s)$.
	\[\laplace(\theta y')(s)=s\laplace(\theta y)(s)-y(0)=sY(s)\]
	\[\laplace(\theta y'')(s)=s\laplace(\theta y')(s)-y'(0)=s^2Y(s)\]
	\[\laplace(\theta y''')(s)=s\laplace(\theta y'')(s)-y''(0)=s^3Y(s)-1\]
	Multiplicera ekvationen med $\theta(t)$ och Laplacetransformera den:
	\begin{align*}
	&s^3Y(s)-1-s^2Y(s)+sY(s)-Y(s)=\frac{1}{s^2} \lra
	(s^3-s^2+s-1)Y(s)=\frac{1}{s^2}+1 \lra \\ \lra
	&Y(s)=\frac{s^2+1}{s^2(s^3-s^2+s-1)}=
	\frac{s^2+1}{s^2(s-1)(s^2+1)}=
	\frac{1}{s^2(s-1)}=
	\frac{k_1}{s}+\frac{k_2}{s^2}+\frac{k_3}{s-1}
	\end{align*}
	Identifiera variablerna:
	\begin{align*}
	&k_1s(s-1)+k_2(s-1)+k_3s^2=1 \lra \\ \lra
	&k_1(s^2-s)+k_2(s-1)+k_3s^2=1 \lra \\ \lra
	&\begin{cases}
	k_1+k_3=0 \\
	-k_1+k_2=0 \\
	-k_2=1
	\end{cases} \lra
	\begin{cases}
	k_1=-1 \\
	k_2=-1 \\
	k_3=1
	\end{cases}
	\end{align*}
	\begin{align*}
	Y(s)=-\frac{1}{s}-\frac{1}{s^2}+\frac{1}{s-1} \llra
	-\theta(t)-t\theta(t)+e^{t}\theta(t)=
	(e^{t}-t-1)\theta(t)
	\end{align*}
	\[y(t)\theta(t)=(e^{t}-t-1)\theta(t)\]
	\[y(t)=e^{t}-t-1\cond{t>0}\]
	$y$ deriveras 2 gånger och då visar det sig att $y(0)=y'(0)=0$, $y''(0)=1$ vilket innebär att begynnelsevillkoren är uppfyllda, funktionen är alltså definierad även för $t=0$.
	
	\ans $y(t)=e^{t}-t-1\cond{t\ge0}$
\end{task}

\begin{task}{5.10}
	Eftersom differentialekvationens högerled är $\delta(t)$ låt $Y(s)=\laplace y(s)$ ($\delta(t)\theta(t)$ saknar betydelse)
	\[sY(s)=\laplace y'(s),\qquad s^2Y(s)=\laplace y''(s),\qquad s^3Y(s)=\laplace y'''(s)\]
	Laplacetransformera ekvationen:
	\begin{align*}
	&s^3Y(s)+3s^2Y(s)+3sY(s)+Y(s)=1 \lra
	(s^3+3s^2+3s+1)Y(s)=1\lra \\ \lra
	&Y(s)=\frac{1}{s^3+3s^2+3s+1}=
	\frac{1}{2}\frac{2}{(s+1)^3}\llra
	\frac{1}{2}e^{-t}t^2\theta(t)
	\end{align*}
	Eftersom $\theta(t)$ är en faktor är $y$ kausal.
	
	\ans $y(t)=\frac{1}{2}e^{-t}t^2\theta(t)\cond{-\infty<t<+\infty}$
\end{task}

\pagebreak
\begin{task}{5.11}
	Eftersom differentialekvationens högerled innehåller $\delta(t)$ låt $Y(s)=\laplace y(s)$ ($\delta(t)\theta(t)$ saknar betydelse)
	\[sY(s)=\laplace y'(s),\qquad s^2Y(s)=\laplace y''(s)\]
	Laplacetransformera ekvationen:
	\begin{align*}
	&s^2Y(s)+sY(s)=1-2e^{-s}+\frac{1}{s}-e^{-s}\frac{1}{s} \lra
	(s^2+s)Y(s)=\frac{s-2se^{-s}+1-e^{-s}}{s}\lra \\ \lra
	&Y(s)=\frac{s-2se^{-s}+1-e^{-s}}{s(s^2+s)}=
	\frac{1}{s(s+1)}-2\frac{e^{-s}}{s(s+1)}+\frac{1}{s^2(s+1)}-\frac{e^{-s}}{s^2(s+1)}= \\ =
	&\frac{1}{s}-\frac{1}{s+1}-2e^{-s}\left(\frac{1}{s}-\frac{1}{s+1}\right)-\frac{1}{s}+\frac{1}{s^2}+\frac{1}{s+1}-e^{-s}\left(-\frac{1}{s}+\frac{1}{s^2}+\frac{1}{s+1}\right)= \\ =
	&\frac{1}{s^2}-e^{-s}\left(\frac{1}{s}+\frac{1}{s^2}-\frac{1}{s+1}\right) \llra
	t\theta(t)-(\theta(t-1)+(t-1)\theta(t-1)-e^{-(t-1)}\theta(t-1))= \\ =
	&t\theta(t)-(t-e^{-(t-1)})\theta(t-1)
	\end{align*}
	Eftersom $\theta(t-\alpha)\cond{\alpha \ge 0}$ är en faktor är $y$ kausal.
	
	För att enklare kunna rita den:
	\[y(t)=t\theta(t)-(t-e^{-(t-1)})\theta(t-1)=t(\theta(t)-\theta(t-1))+e^{-(t-1)}\theta(t-1)\]
	Funktionen $t$ i intervallet $0<t<1$, funktionen $e^{-(t-1)}$ i intervallet $1<t<\infty$
	
	\ans $y(t)=t(\theta(t)-\theta(t-1))+e^{-(t-1)}\theta(t-1)$
\end{task}

\begin{task}{5.12}
	Låt $Y(s)=\laplace(\theta y)(s)$.
	\[\laplace(\theta y')(s)=s\laplace(\theta y)(s)-y(0)=sY(s)\]
	\[\laplace(\theta y'')(s)=s\laplace(\theta y')(s)-y'(0)=s^2Y(s)\]
	Multiplicera ekvationen med $\theta(t)$ och Laplacetransformera den:
	\begin{align*}
	&s^2Y(s)+3sY(s)+2Y(s)=\frac{s}{s^2+1} \lra
	(s^2+3s+2)Y(s)=\frac{s}{s^2+1} \lra \\ \lra
	&Y(s)=\frac{s}{(s^2+1)(s^2+3s+2)}=
	\frac{s}{(s^2+1)(s+1)(s+2)}=
	\frac{k_1s+k_2}{s^2+1}+\frac{k_3}{s+1}+\frac{k_4}{s+2}
	\end{align*}
	Identifiera variablerna:
	\begin{align*}
	&(k_1s+k_2)(s^2+3s+2)+k_3(s^2+1)(s+2)+k_4(s^2+1)(s+1)=s \lra \\ \lra
	&k_1(s^3+3s^2+2s)+k_2(s^2+3s+2)+k_3(s^3+2s^2+s+2)+k_4(s^3+s^2+s+1)=s \lra \\ \lra
	&\begin{cases}
	k_1+k_3+k_4=0 \\
	3k_1+k_2+2k_3+k_4=0 \\
	2k_1+3k_2+k_3+k_4=1 \\
	2k_2+2k_3+k_4=0
	\end{cases} \lra
	\begin{cases}
	k_1=\frac{1}{10} \\
	k_2=\frac{3}{10} \\
	k_3=-\frac{1}{2} \\
	k_4=\frac{2}{5}
	\end{cases}
	\end{align*}
	\begin{align*}
	Y(s)=
	&\frac{1}{10}\*\frac{s+3}{s^2+1}-\frac{1}{2}\*\frac{1}{s+1}+\frac{2}{5}\*\frac{1}{s+2}=
	\frac{1}{10}\left(\frac{s}{s^2+1}+3\frac{1}{s^2+1}-5\frac{1}{s+1}+4\frac{1}{s+2}\right) \llra \\ \llra
	&\frac{1}{10}\left(\cos(t)+3\sin(t)-5e^{-t}+4e^{-2t}\right)\theta(t)
	\end{align*}
	\[y(t)\theta(t)=\frac{1}{10}\left(\cos(t)+3\sin(t)-5e^{-t}+4e^{-2t}\right)\theta(t)\]
	\[y(t)=\frac{1}{10}\left(\cos(t)+3\sin(t)-5e^{-t}+4e^{-2t}\right)\cond{t>0}\]
	$y$ deriveras och då visar det sig att $y(0)=y'(0)=0$ vilket innebär att begynnelsevillkoren är uppfyllda, funktionen är alltså definierad även för $t=0$.
	
	\ans $y(t)=\frac{1}{10}\left(\cos(t)+3\sin(t)-5e^{-t}+4e^{-2t}\right)\cond{t\ge0}$
\end{task}

\begin{task}{5.13 a)}
	\[f(t)=t(\theta(t)-\theta(t-1))-t(\theta(t-1)-\theta(t-2))=t\theta(t)-2t\theta(t-1)+t\theta(t-2)\]
	Hitta $f$s Laplacetransformpar:
	\[\laplace f(s)=\frac{1}{s^2}-2e^{-s}\frac{1}{s^2}+e^{-2s}\frac{1}{s^2}=\frac{1-2e^{-s}+e^{-2s}}{s^2}\]
	Utnyttja regeln $f'(t)\llra sF(s)$:
	\[\laplace f'(s)=s\frac{1-2e^{-s}+e^{-2s}}{s^2}=\frac{1-2e^{-s}+e^{-2s}}{s}\]
	\[\laplace f''(s)=s\frac{1-2e^{-s}+e^{-2s}}{s}=1-2e^{-s}+e^{-2s}\]
	Hitta inversen:
	\[f''(t)=\delta(t)-2\delta(t-1)+\delta(t-2)\]
	\ans $f''(t)=\delta(t)-2\delta(t-1)+\delta(t-2)$
\end{task}

\begin{task}{b)}
	Eftersom differentialekvationens högerled innehåller $\delta(t)$ låt $Y(s)=\laplace y(s)$ ($\delta(t)\theta(t)$ saknar betydelse)
	\[sY(s)=\laplace y'(s),\qquad s^2Y(s)=\laplace y''(s)\]
	Laplacetransformera ekvationen:
	\begin{align*}
	&s^2Y(s)+2sY(s)+2Y(s)=1-2e^{-s}+e^{-2s} \lra
	(s^2+2s+2)Y(s)=1-2e^{-s}+e^{-2s}\lra \\ \lra
	&Y(s)=\frac{1-2e^{-s}+e^{-2s}}{s^2+2s+2}=
	\frac{1}{(s+1)^2+1}-2\frac{e^{-s}}{(s+1)^2+1}+\frac{e^{-2s}}{(s+1)^2+1} \llra \\ \llra
	&e^{-t}\sin(t)\theta(t)-2e^{-(t-1)}\sin(t-1)\theta(t-1)+e^{-(t-2)}\sin(t-2)\theta(t-2)
	\end{align*}
	Eftersom $\theta(t-\alpha)\cond{\alpha \ge 0}$ är en faktor är $y$ kausal.
	
	\ans $y(t)=e^{-t}\sin(t)\theta(t)-2e^{-(t-1)}\sin(t-1)\theta(t-1)+e^{-(t-2)}\sin(t-2)\theta(t-2)$
\end{task}

\pagebreak
\begin{task}{5.14}
	Låt $X(s)=\laplace(\theta x)(s)$ och $Y(s)=\laplace(\theta y)(s)$ vilket medför att:
	\[\laplace(\theta x')(s)=s\laplace(\theta x)(s)-x(0)=sX(s)\]
	\[\laplace(\theta x'')(s)=s\laplace(\theta x')(s)-x'(0)=s^2X(s)\]
	\[\laplace(\theta y')(s)=s\laplace(\theta y)(s)-y(0)=sY(s)-1\]
	\[\laplace(\theta y'')(s)=s\laplace(\theta y')(s)-y'(0)=s^2Y(s)-s\]
	Multiplicera ekvationssystemet med $\theta(t)$:
	\[
	\begin{cases}
	x''(t)\theta(t)+2x(t)\theta(t)+y'(t)\theta(t)=0 \\
	y''(t)\theta(t)+2y(t)\theta(t)-x'(t)\theta(t)=0
	\end{cases}\]
	Laplacetransformera båda ekvationerna:
	\[
	\begin{cases}
	s^2X+2X+sY-1=0\\
	s^2Y-s+2Y-sX=0
	\end{cases} \lra
	\begin{linsys}{rrr}
	(s^2+2)X+&sY=&1 \\
	-sX+&(s^2+2)Y=&s
	\end{linsys}\]
	Använd Cramers regel:
	\begin{align*}
	\Delta(s)=\begin{detmat}
	s^2+2 & s \\
	-s & s^2+2
	\end{detmat}=
	(s^2+2)^2+s^2=
	(s^2+4)(s^2+1)
	\end{align*}
	\begin{align*}
	X(s)=
	\frac{1}{\Delta(s)}\begin{detmat}
	1 & s \\
	s & s^2+2
	\end{detmat}=
	\frac{1}{(s^2+4)(s^2+1)}\left(s^2+2-s^2\right)=
	\frac{2}{(s^2+4)(s^2+1)}
	\end{align*}
	\begin{align*}
	Y(s)=
	\frac{1}{\Delta(s)}\begin{detmat}
	s^2+2 & 1 \\
	-s & s
	\end{detmat}=
	\frac{1}{(s^2+4)(s^2+1)}\left(s^3+2s+s\right)=
	\frac{s^3+3s}{(s^2+4)(s^2+1)}
	\end{align*}
	Hitta lösningar till $X$ och $Y$ var för sig: 
	\begin{align*}
	X(s)=
	\frac{2}{(s^2+4)(s^2+1)}=
	\frac{2}{3}\left(\frac{1}{s^2+1}-\frac{1}{2}\*\frac{2}{s^2+4}\right)\llra
	&\frac{1}{3}(2\sin(t)\theta(t)-\sin(2t)\theta(t))= \\ =
	&\frac{1}{3}(2\sin(t)-\sin(2t))\theta(t)
	\end{align*}
	\begin{align*}
	Y(s)=
	\frac{s^3+3s}{(s^2+4)(s^2+1)}=
	\frac{1}{3}\left(2\frac{s}{s^2+1}+\frac{s}{s^2+4}\right)\llra
	&\frac{1}{3}\left(2\cos(t)\theta(t)+\cos(2t)\theta(t)\right)= \\ =
	&\frac{1}{3}\left(2\cos(t)+\cos(2t)\right)\theta(t)
	\end{align*}
	\[\begin{cases}
	x\theta=\frac{1}{3}(2\sin(t)-\sin(2t))\theta(t) \\
	y\theta=\frac{1}{3}(2\cos(t)+\cos(2t))\theta(t)
	\end{cases}\]
	Eftersom $\theta(t)=1$ endast när $t>0$ måste det läggas till som villkor:
	\[\begin{cases}
	x=\frac{1}{3}(2\sin(t)-\sin(2t)) \\
	y=\frac{1}{3}(2\cos(t)+\cos(2t))
	\end{cases}\cond{t > 0}\]
	$y$ och $x$ deriveras och då visar det sig att $x(0)=x'(0)=y'(0)=0$, $y(0)=1$ vilket innebär att begynnelsevillkoren är uppfyllda, funktionen är alltså definierad även för $t=0$.
	
	\ans 
	$\begin{cases}
	x=\frac{1}{3}(2\sin(t)-\sin(2t)) \\
	y=\frac{1}{3}(2\cos(t)+\cos(2t))
	\end{cases}\cond{t \ge 0}$
\end{task}

\begin{task}{5.15}
	Låt $X(s)=\laplace(\theta x)(s)$ och $Y(s)=\laplace(\theta y)(s)$ vilket medför att:
	\[\laplace(\theta x')(s)=s\laplace(\theta x)(s)-x(0)=sX(s)\]
	\[\laplace(\theta y')(s)=s\laplace(\theta y)(s)-y(0)=sY(s)-1\]
	Multiplicera ekvationssystemet med $\theta(t)$:
	\[
	\begin{cases}
	x'(t)\theta(t)+2x(t)\theta(t)+y'(t)\theta(t)=0 \\
	2x'(t)\theta(t)+3x(t)\theta(t)+2y'(t)\theta(t)+y(t)\theta(t)=0
	\end{cases}\]
	Laplacetransformera båda ekvationerna:
	\[
	\begin{cases}
	sX(s)+2X(s)+sY(s)-1=0 \\
	2sX(s)+3X(s)+2(sY(s)-1)+Y(s)=0
	\end{cases} \lra
	\begin{linsys}{rrr}
	(s+2)X+&sY=&1 \\
	(2s+3)X+&(2s+1)Y=&2
	\end{linsys}\]
	Använd Cramers regel:
	\begin{align*}
		\Delta(s)=\begin{detmat}
			s+2 & s \\
			2s+3 & 2s+1
		\end{detmat}=
		(s+2)(2s+1)-s(2s+3)=
		2(s+1)
	\end{align*}
	\begin{align*}
		X(s)=
		\frac{1}{\Delta(s)}\begin{detmat}
			1 & s \\
			2 & 2s+1
		\end{detmat}=
		\frac{1}{2(s+1)}\left(2s+1-2s\right)=
		\frac{1}{2}\*\frac{1}{s+1}
	\end{align*}
	\begin{align*}
		Y(s)=
		\frac{1}{\Delta(s)}\begin{detmat}
			s+2 & 1 \\
			2s+3 & 2
		\end{detmat}=
		\frac{1}{2(s+1)}\left(2(s+2)-(2s+3)\right)=
		\frac{1}{2}\*\frac{1}{s+1}
	\end{align*}
	Hitta lösningen till $X$ och $Y$: 
	\begin{align*}
		X(s)=Y(s)=
		\frac{1}{2}\*\frac{1}{s+1}\llra
		\frac{1}{2}e^{-t}\theta(t)
	\end{align*}
	Eftersom $\theta(t)=1$ endast när $t>0$ måste det läggas till som villkor:
	\[x\theta=y\theta=\frac{1}{2}e^{-t}\theta(t)\lra x=y=\frac{1}{2}e^{-t}\cond{t>0}\]
	$x(0)=y(0)=\frac{1}{2}e^{0}=\frac{1}{2}$ vilket innebär att begynnelsevillkoren inte är uppfyllda, det saknas alltså en lösning till ekvationssystemet.
	
	\ans lösning saknas
\end{task}

\pagebreak
\begin{task}{5.16}
	Eftersom en av differentialekvationernas högerled är $\delta(t-1)$ låt $X(s)=\laplace x(s)$ och $Y(s)=\laplace y(s)$ ($\delta(t)\theta(t)$ saknar betydelse)
	\[\laplace x'(s)=sX(s)\]
	\[\laplace y'(s)=sY(s)\]
	Laplacetransformera båda ekvationerna:
	\[
	\begin{cases}
	sX(s)+Y(s)=e^{-s} \\
	sY(s)-X(s)=\frac{1}{s}
	\end{cases} \lra
	\begin{linsys}{rrr}
	sX+&Y=&e^{-s} \\
	-X+&sY=&\frac{1}{s}
	\end{linsys}\]
	Använd Cramers regel:
	\begin{align*}
	\Delta(s)=\begin{detmat}
	s & 1 \\
	-1 & s
	\end{detmat}=
	s^2-(-1)=
	s^2+1
	\end{align*}
	\begin{align*}
	X(s)=
	\frac{1}{\Delta(s)}\begin{detmat}
	e^{-s} & 1 \\
	\frac{1}{s} & s
	\end{detmat}=
	\frac{1}{s^2+1}\left(se^{-s}-\frac{1}{s}\right)=
	e^{-s}\frac{s}{s^2+1}-\left(\frac{1}{s}-\frac{s}{s^2+1}\right)
	\end{align*}
	\begin{align*}
	Y(s)=
	\frac{1}{\Delta(s)}\begin{detmat}
	s & e^{-s} \\
	-1 & \frac{1}{s}
	\end{detmat}=
	\frac{1}{s^2+1}\left(1-(-e^{-s})\right)=
	e^{-s}\frac{1}{s^2+1}+\frac{1}{s^2+1}
	\end{align*}
	Hitta lösningar till $x$ och $y$ var för sig: 
	\begin{align*}
	X(s)=
	e^{-s}\frac{s}{s^2+1}-\left(\frac{1}{s}-\frac{s}{s^2+1}\right)\llra
	\cos(t-1)\theta(t-1)-\theta(t)+\cos(t)\theta(t)
	\end{align*}
	\begin{align*}
	Y(s)=
	e^{-s}\frac{1}{s^2+1}+\frac{1}{s^2+1} \llra
	\sin(t-1)\theta(t-1)+\sin(t)\theta(t)
	\end{align*}
	\[\begin{cases}
	x=\cos(t-1)\theta(t-1)-\theta(t)+\cos(t)\theta(t) \\
	y=\sin(t-1)\theta(t-1)+\sin(t)\theta(t)
	\end{cases}\cond{t > 0}\]
	Eftersom $\theta(t-\alpha)\cond{\alpha\ge 0}$ är en faktor i alla termerna är $y$ och $x$ kausala.
	
	\ans
	$\begin{cases}
	x=\cos(t-1)\theta(t-1)-\theta(t)+\cos(t)\theta(t) \\
	y=\sin(t-1)\theta(t-1)+\sin(t)\theta(t)
	\end{cases}\cond{-\infty < t < +\infty}$
\end{task}

\pagebreak
\begin{task}{5.17}
	Låt $X_1(s)=\laplace(\theta x_1)(s)$, $X_2(s)=\laplace(\theta x_2)(s)$ och $X_3(s)=\laplace(\theta x_3)(s)$ vilket medför att:
	\[\laplace(\theta x_1')(s)=s\laplace(\theta x_1)(s)-x_1(0)=sX_1(s)\]
	\[\laplace(\theta x_2')(s)=s\laplace(\theta x_2)(s)-x_2(0)=sX_2(s)-1\]
	\[\laplace(\theta x_3')(s)=s\laplace(\theta x_3)(s)-x_3(0)=sX_3(s)-2\]
	Multiplicera ekvationssystemet med $\theta(t)$:
	\[
	\begin{cases}
	x_1'(t)\theta(t)=x_2(t)\theta(t) \\
	x_2'(t)\theta(t)=x_3(t)\theta(t) \\
	x_3'(t)\theta(t)=-2x_1(t)\theta(t)+x_2(t)\theta(t)+2x_3(t)\theta(t)
	\end{cases}\]
	Laplacetransformera båda ekvationerna:
	\[
	\begin{cases}
	sX_1(s)=X_2(s) \\
	sX_2(s)-1=X_3(s) \\
	sX_3(s)-2=-2X_1(s)+X_2(s)+2X_3(s)
	\end{cases} \lra
	\begin{linsys}{rrrrr}
	sX_1-& X_2& &        =&0 \\
	     &sX_2&-&     X_3=&1 \\
	2X_1-& X_2&+&(s-2)X_3=&2
	\end{linsys}\]
	Använd Cramers regel:
	\begin{align*}
	\Delta(s)=\begin{detmat}
	s & -1 & 0 \\
	0 & s & -1 \\
	2 & -1 & s-2
	\end{detmat}=
	s^2(s-2)+2+0-0-s-0=
	(s-2)(s-1)(s+1)
	\end{align*}
	\begin{align*}
	X_1(s)=
	\frac{1}{\Delta(s)}\begin{detmat}
	0 & -1 & 0 \\
	1 & s & -1 \\
	2 & -1 & s-2
	\end{detmat}=
	\frac{1}{(s-2)(s-1)(s+1)}\left(2-(-(s-2))\right)=
	\frac{s}{(s-2)(s-1)(s+1)}
	\end{align*}
	\begin{align*}
	X_2(s)=
	\frac{1}{\Delta(s)}\begin{detmat}
	s & 0 & 0 \\
	0 & 1 & -1 \\
	2 & 2 & s-2
	\end{detmat}=
	\frac{1}{(s-2)(s-1)(s+1)}\left(s(s-2)-(-2s)\right)=
	\frac{s^2}{(s-2)(s-1)(s+1)}
	\end{align*}
	\begin{align*}
	X_3(s)=
	\frac{1}{\Delta(s)}\begin{detmat}
	s & -1 & 0 \\
	0 & s & 1 \\
	2 & -1 & 2
	\end{detmat}=
	\frac{1}{(s-2)(s-1)(s+1)}\left(2s^2-2-(-s)\right)=
	\frac{2s^2+s-2}{(s-2)(s-1)(s+1)}
	\end{align*}
	Hitta lösningar till $X_1$, $X_2$ och $X_3$ var för sig: 
	\begin{align*}
	X_1(s)=
	\frac{s}{(s-2)(s-1)(s+1)}=
	\frac{2}{3}\*\frac{1}{s-2}-\frac{1}{2}\*\frac{1}{s-1}-\frac{1}{6}\*\frac{1}{s+1} \llra
	\frac{2}{3}e^{2t}\theta(t)-\frac{1}{2}e^{t}\theta(t)-\frac{1}{6}e^{-t}\theta(t)
	\end{align*}
	\begin{align*}
	X_2(s)=
	\frac{s^2}{(s-2)(s-1)(s+1)}=
	\frac{4}{3}\*\frac{1}{s-2}-\frac{1}{2}\*\frac{1}{s-1}+\frac{1}{6}\*\frac{1}{s+1} \llra
	\frac{4}{3}e^{2t}\theta(t)-\frac{1}{2}e^{t}\theta(t)+\frac{1}{6}e^{-t}\theta(t)
	\end{align*}
	\begin{align*}
	X_3(s)=
	\frac{2s^2+s-2}{(s-2)(s-1)(s+1)}=
	\frac{8}{3}\*\frac{1}{s-2}-\frac{1}{2}\*\frac{1}{s-1}-\frac{1}{6}\*\frac{1}{s+1} \llra
	\frac{8}{3}e^{2t}\theta(t)-\frac{1}{2}e^{t}\theta(t)-\frac{1}{6}e^{-t}\theta(t)
	\end{align*}
	Eftersom $\theta(t)=1$ endast när $t>0$ måste det läggas till som villkor:
	\[\begin{cases}
	x_1\theta=\frac{2}{3}e^{2t}\theta(t)-\frac{1}{2}e^{t}\theta(t)-\frac{1}{6}e^{-t}\theta(t) \\
	x_2\theta=\frac{4}{3}e^{2t}\theta(t)-\frac{1}{2}e^{t}\theta(t)+\frac{1}{6}e^{-t}\theta(t) \\
	x_3\theta=\frac{8}{3}e^{2t}\theta(t)-\frac{1}{2}e^{t}\theta(t)-\frac{1}{6}e^{-t}\theta(t)
	\end{cases} \lra
	\begin{cases}
	x_1=\frac{2}{3}e^{2t}-\frac{1}{2}e^{t}-\frac{1}{6}e^{-t} \\
	x_2=\frac{4}{3}e^{2t}-\frac{1}{2}e^{t}+\frac{1}{6}e^{-t} \\
	x_3=\frac{8}{3}e^{2t}-\frac{1}{2}e^{t}-\frac{1}{6}e^{-t}
	\end{cases}\cond{t>0}\]
	Om 0 sätts in i alla funktionerna finnes att $x_1(0)=0$, $x_2(0)=1$ och $x_3(0)=2$ vilket innebär att begynnelsevillkoren är uppfyllda, funktionen är alltså definierad även för $t=0$.
	
	\ans 
	$\begin{cases}
	x_1=\frac{2}{3}e^{2t}-\frac{1}{2}e^{t}-\frac{1}{6}e^{-t} \\
	x_2=\frac{4}{3}e^{2t}-\frac{1}{2}e^{t}+\frac{1}{6}e^{-t} \\
	x_3=\frac{8}{3}e^{2t}-\frac{1}{2}e^{t}-\frac{1}{6}e^{-t}
	\end{cases}\cond{t\ge0}$
\end{task}

\begin{task}{5.18}
\end{task}
\pagebreak
\chapter{6}{Faltning}

\begin{task}{6.1}
	Använd definitionen av faltning:
	\begin{align*}
	f*g(t)=
	&\int_{-\infty}^{+\infty}\! f(t-\tau)g(\tau)\, d\tau=
	\int_{-\infty}^{+\infty}\! e^{-(t-\tau)}\theta(t-\tau)e^{-3\tau}\theta(\tau)\, d\tau=
	\left(\int_{0}^{t}\! e^{-(t-\tau)-3\tau}\, d\tau\right)\theta(t)= \\ =
	&\left(\int_{0}^{t}\! e^{-t-2\tau}\, d\tau\right)\theta(t)=
	\left[-\frac{e^{-t-2\tau}}{2}\right]_0^t\theta(t)=
	\left(-\frac{e^{-3t}}{2}-\left(-\frac{e^{-t}}{2}\right)\right)\theta(t)=
	\frac{1}{2}(e^{-t}-e^{-3t})\theta(t)
	\end{align*}
	\ans $f*g(t)=\frac{1}{2}(e^{-t}-e^{-3t})\theta(t)$
\end{task}

\begin{task}{6.2}
	\begin{align*}
	\laplace f(s)=
	\laplace (e^{-t}\theta)(s)=
	\frac{1}{s+1}
	\end{align*}
	\begin{align*}
	\laplace g(s)=
	\laplace (e^{-3t}\theta)(s)=
	\frac{1}{s+3}
	\end{align*}
	\begin{align*}
	\laplace (f*g)(s)=
	\laplace (\frac{1}{2}(e^{-t}-e^{-3t})\theta)(s)=
	\frac{1}{2}\*\frac{1}{s+1}-\frac{1}{2}\*\frac{1}{s+3}=
	\frac{1}{(s+1)(s+3)}
	\end{align*}
\end{task}

\begin{task}{6.3}
	Gör ett variabelbyte i integralen för att bevisa lagen.
	\begin{align*}
	f*g(t)=
	&\int_{-\infty}^{+\infty}\!f(t-\tau)g(\tau)\, d\tau=
	\begin{bmatrix}
	x=t-\tau \\
	\tau=t-x \\
	d\tau=-dx
	\end{bmatrix}=
	-\int_{+\infty}^{-\infty}\!f(x)g(t-x)\, dx= \\ =
	&\int_{-\infty}^{+\infty}\!g(t-x)f(x)\, dx=
	g*f(t)\mproof
	\end{align*}
\end{task}

\begin{task}{6.4 a)}
	\[f*g(t)=\int_{-\infty}^{+\infty}\!f(t-\tau)g(\tau)\, d\tau\]
\end{task}

\begin{task}{6.4 b)}
	Låt $f\theta$ och $g\theta$ vara två kausala funktioner, definitionen av faltning ger då:
	\[(f\theta)*(g\theta)(t)=
	\int_{-\infty}^{+\infty}\!f(t-\tau)\theta(t-\tau)g(\tau)\theta(\tau)\, d\tau\]
	Eftersom $\theta(t-\tau)\theta(\tau)\equiv0\cond{t\le 0}$ och $\theta(t-\tau)\theta(\tau)=0\cond{0\le\tau\le t}\cond{t > 0}$ är:
	\[(f\theta)*(g\theta)(t)=\begin{linsys}{ll}
	\int_{0}^{t}\!f(t-\tau)g(\tau)\, d\tau&\cond{t\le0} \\
	0                                     &\cond{t>0}
	\end{linsys}\]
	Vilket också kan beskrivas som:
	\[(f\theta)*(g\theta)(t)=
	\left(\int_{0}^{t}\!f(t-\tau)g(\tau)\, d\tau\right)\theta(t)\]
\end{task}

\begin{task}{6.4 c)}
	Utnyttja definitionen av Laplacetransformen och faltning:
	\begin{align*}
	\laplace(f*g)(s)=
	&\int_{-\infty}^{+\infty}\!e^{-st}\left(\int_{-\infty}^{+\infty}\!f(t-\tau)g(\tau)\,d\tau\right)dt=
	\int_{-\infty}^{+\infty}\!\left(\int_{-\infty}^{+\infty}\!e^{-st}f(t-\tau)g(\tau)\,dt\right)d\tau= \\ =
	&\int_{-\infty}^{+\infty}\!e^{-s\tau}g(\tau)\left(\int_{-\infty}^{+\infty}\!e^{-s(t-\tau)}f(t-\tau)\,dt\right)d\tau=
	\begin{bmatrix}
	u=t-\tau \\
	du=dt
	\end{bmatrix}= \\ =
	&\int_{-\infty}^{+\infty}\!e^{-s\tau}g(\tau)\left(\int_{-\infty}^{+\infty}\!e^{-su}f(u)\,du\right)d\tau=
	\int_{-\infty}^{+\infty}\!e^{-s\tau}g(\tau)\laplace f(s)\, d\tau= \\ =
	&\laplace f(s)\int_{-\infty}^{+\infty}\!e^{-s\tau}g(\tau)\, d\tau=
	\laplace f(s)\*\laplace g(s)\mproof
	\end{align*}
\end{task}

\begin{task}{6.5}
	\[\laplace f(s)=\frac{2}{s^3},\qquad
	\laplace g(s)=\frac{24}{s^5}\]
	\begin{align*}
	\laplace v(s)=
	\laplace(f*g)(s)=
	\laplace f(s)\*\laplace g(s)=
	\frac{2}{s^3}\*\frac{24}{s^5}=
	\frac{48}{s^8}=\frac{1}{105}\*\frac{7!}{s^8}\llra
	\frac{t^{7}}{105}\theta(t)=
	v(t)
	\end{align*}
	\ans $v(t)=\dfrac{t^{7}}{105}\theta(t)$
\end{task}

\begin{task}{6.6}
	Använd regeln $\laplace (f*g)(s)=\laplace f(s)\*\laplace g(s)$:
	\[\laplace f(s)=\frac{1}{(s+3)^2},\qquad
	\laplace g(s)=\frac{1}{s+1}\]
	\begin{align*}
	\laplace(f*g)(s)=
	\laplace f(s)\*\laplace g(s)=
	\frac{1}{(s+3)^2}\*\frac{1}{s+1}=
	\frac{k_1}{(s+3)^2}+\frac{k_2}{s+3}+\frac{k_3}{s+1}
	\end{align*}
	Identifiera variablerna:
	\begin{align*}
	&k_1(s+1)+k_2(s+3)(s+1)+k_3(s+3)^2=1 \lra \\ \lra
	&k_1(s+1)+k_2(s^2+4s+3)+k_3(s^2+6s+9)=1 \lra \\ \lra
	&\begin{cases}
	k_2+k_3=0 \\
	k_1+4k_2+6k_3=0 \\
	k_1+3k_2+9k_3=1
	\end{cases} \lra
	\begin{cases}
	k_1=-\frac{1}{2} \\
	k_2=-\frac{1}{4} \\
	k_3=\frac{1}{4}
	\end{cases}
	\end{align*}
	\begin{align*}
	\laplace (f*g)(s)=
	-\frac{1}{2}\*\frac{1}{(s+3)^2}-\frac{1}{4}\*\frac{1}{s+3}+\frac{1}{4}\*\frac{1}{s+1} \llra
	&-\frac{1}{2}te^{-3t}\theta(t)-\frac{1}{4}e^{-3t}\theta(t)+\frac{1}{4}e^{-t}\theta(t)= \\ =
	&\frac{1}{4}(e^{-t}-(2t+1)e^{-3t})\theta(t)
	\end{align*}
	\ans $f*g(t)=\frac{1}{4}(e^{-t}-(2t+1)e^{-3t})\theta(t)$
\end{task}

\begin{task}{6.7}
	\begin{align*}
	\laplace (u*u)(s)=
	&\laplace u(s)\*\laplace u(s)=
	(\laplace u(s))^2=
	\frac{6}{(s+1)^4} \lra \\ \lra
	\laplace u(s)=
	&\pm\frac{\sqrt{6}}{(s+1)^2}=
	\pm\sqrt{6}\frac{1}{(s+1)^2}\llra
	\pm\sqrt{6}te^{-t}\theta(t)
	\end{align*}
	\ans $u(t)=\pm\sqrt{6}te^{-t}\theta(t)$
\end{task}

\begin{task}{6.8}
	Låt $G(s)=\laplace g(s)$ och $Y(s)=\laplace y(s) \lra \laplace y'(s)=sY(s)$.
	\[2(y*g)=y-y'+g\]
	Laplacetransformera ekvationen:
	\begin{align*}
	&2Y(s)G(s)=Y(s)-sY(s)+G(s) \lra
	2\frac{1}{s+1}Y(s)=(1-s)Y(s)+\frac{1}{s+1} \lra \\ \lra
	&\left(\frac{2}{s+1}-1+s\right)Y(s)=\frac{1}{s+1} \lra
	\frac{2-(s+1)+s(s+1)}{s+1}Y(s)=\frac{1}{s+1} \lra \\ \lra
	&\frac{s^2+1}{s+1}Y(s)=\frac{1}{s+1} \lra
	Y(s)=\frac{1}{s^2+1}
	\end{align*}
	Hitta inversen:
	\begin{align*}
	\laplace^{-1}Y(t)=y(t)=\sin(t)\theta(t)
	\end{align*}
	\ans $y(t)=\sin(t)\theta(t)$
\end{task}

\begin{task}{6.9 a)}
	Låt $Y(s)=\laplace(\theta y)(s)$.
	\[\laplace(\theta y'')(s)=s\laplace(\theta y')(s)-y'(0)=s(s\laplace(\theta y)(s)-y(0))-y'(0)=s^2Y(s)\]
	Multiplicera ekvationen med $\theta(t)$ och Laplacetransformera den:
	\begin{align*}
	&s^2Y(s)+Y(s)=F(s) \lra
	(s^2+1)Y(s)=F(s) \lra
	Y(s)=\frac{1}{s^2+1}F(s)
	\end{align*}
	\ans $Y(s)=\dfrac{1}{s^2+1}F(s)$
\end{task}


\begin{task}{b)}
	\[Y(s)=\frac{1}{s^2+1}F(s)\]
	Låt $H(s)=\laplace (h\theta)(s)=\dfrac{1}{s^2+1}$.
	\[h(t)\theta(t)=\sin(t)\theta(t) \lra h(t)=\sin(t)\cond{t > 0}\]
	Hitta inversen till ekvationen:
	\begin{align*}
	y(t)\theta(t)=(h*f(t))\theta(t)
	\end{align*}
	Använd definitionen av kausal faltning:
	\begin{align*}
	y(t)=
	\int_{0}^{t}\!h(t-\tau)f(t)\, d\tau=
	\int_{0}^{t}\!\sin(t-\tau)f(t)\, d\tau\cond{t>0}
	\end{align*}
	\ans $y(t)=\int_{0}^{t}\!\sin(t-\tau)f(t)\, d\tau\cond{t>0}$
\end{task}

\pagebreak
\begin{task}{6.10}
	Låt $G(s)=\laplace(\cos\theta)(s)$ och $F(s)=\laplace(f\theta)(s)$.
	\[G(s)=\frac{s}{s^2+1}\]
	Använd definitionen av faltning baklänges:
	\begin{align*}
	\int_{0}^{t}\!\cos(t-\tau)f(\tau)\, d\tau=
	\int_{-\infty}^{+\infty}\!\cos(t-\tau)\theta(t-\tau)f(\tau)\theta(\tau)\, d\tau=
	\cos(t)\theta(t)*f(t)\theta(t)
	\end{align*}
	Sätt in i ekvationen:
	\[\cos(t)\theta(t)*f(t)\theta(t)=\sin(2t)\cond{t>0}\]
	Multiplicera ekvationen med $\theta(t)$ (notera att $\theta(t)\*\theta(t)=\theta(t)$):
	\[\cos(t)\theta(t)*f(t)\theta(t)=\sin(2t)\theta(t)\]
	Hitta Laplacetransformen:
	\begin{align*}
	&G(s)F(s)=\frac{2}{s^2+4} \lra
	\frac{s}{s^2+1}F(s)=\frac{2}{s^2+4} \lra \\ \lra
	&F(s)=
	\frac{2s^2+2}{s(s^2+4)}=
	\frac{k_1}{s}+\frac{k_2s+k_3}{s^2+4}
	\end{align*}
	Identifiera variablerna:
	\begin{align*}
	&k_1(s^2+4)+(k_2s+k_3)s=1 \lra
	\begin{cases}
	k_1+k_2=2 \\
	k_3=0 \\
	4k_1=2
	\end{cases} \lra
	\begin{cases}
	k_1=\frac{1}{2} \\
	k_2=\frac{3}{2} \\
	k_3=0
	\end{cases}
	\end{align*}
	\begin{align*}
	F(s)=
	\frac{1}{2}\left(\frac{1}{s}+3\frac{s}{s^2+4}\right) \llra
	\frac{1}{2}\left(\theta(t)+3\cos(2t)\theta(t)\right)=
	\frac{1}{2}\left(1+3\cos(2t)\right)\theta(t)
	\end{align*}
	\[f(t)\theta(t)=\frac{1}{2}\left(1+3\cos(2t)\right)\theta(t)\lra
	f(t)=\frac{1}{2}\left(1+3\cos(2t)\right)\cond{t>0}\]
	\ans $f(t)=\frac{1}{2}\left(1+3\cos(2t)\right)\cond{t>0}$
\end{task}

\pagebreak
\begin{task}{6.11}
	Genom att sätta $t=0$ fås begynnelsevärdet:
	\[y(0)+\int_{0}^{0}\!e^{\tau}y(\tau)\,d\tau=1 \lra
	y(0)=1\]
	Låt $f(t)=e^{-t}$ vilket medför:
	\[\int_{0}^{t}\!e^{-(t-\tau)}y(\tau)\,d\tau=
	\int_{0}^{t}\!f(t-\tau)y(\tau)\,d\tau=
	f(t)\theta(t)*y(t)\theta(t)\]
	Multiplicera ekvationen med $\theta(t)$ (notera att $\theta(t)\*\theta(t)=\theta(t)$):
	\[y(t)\theta(t)+f(t)\theta(t)*y(t)\theta(t)=\theta(t)\]
	Låt $Y(s)=\laplace(y\theta)(s)$ och $F(s)=\laplace(f\theta)(s)=\frac{1}{s+1}$ och Laplacetransformera ekvationen:
	\begin{align*}
	&Y(s)+F(s)\*Y(s)=\frac{1}{s} \lra
	(1+F(s))Y(s)=\frac{1}{s} \lra \\ \lra
	&Y(s)=
	\frac{1}{(1+F(s))s}=
	\frac{1}{(1+\frac{1}{s+1})s}=
	\frac{1}{\frac{(s+2)s}{s+1}}=
	\frac{s+1}{(s+2)s}=
	\frac{1}{2}\*\frac{1}{s}+\frac{1}{2}\*\frac{1}{s+2}
	\end{align*}
	Hitta den inversa Laplacetransformen:
	\begin{align*}
	y(t)\theta(t)=\frac{1}{2}(\theta(t)+e^{-2t}\theta(t))=\frac{1}{2}(1+e^{-2t})\theta(t)
	\end{align*}
	Vilket ger:
	\begin{align*}
	y(t)=\frac{1}{2}(1+e^{-2t})\cond{t>0}
	\end{align*}
	Se om begynnelsevärdet matchar det som togs fram i början:
	\begin{align*}
	y(0)=\frac{1}{2}(1+e^{0})=\frac{2}{2}=1
	\end{align*}
	Det matchar och funktionen är alltså även definierad för $t=0$.
	
	\ans $y(t)=\frac{1}{2}(1+e^{-2t})\cond{t\ge0}$
\end{task}

\begin{task}{6.12}
	Låt $Y(s)=\laplace y(s)$ och $F(s)=\laplace f(s)$ vilket medför att $\laplace y^{(4)}(s)=s^4Y(s)$ och $\laplace f''(s)=s^2F(s)$.
	
	Laplacetransformera ekvationen:
	\[s^4Y(s)+4Y(s)=s^2F(s) \lra
	(s^4+4)Y(s)=s^2F(s) \lra
	Y(s)=\frac{s^2}{s^4+4}F(s)\]
\end{task}
\pagebreak
\input{chapters/7.tex}
\pagebreak
\input{chapters/8.tex}
\pagebreak
\input{chapters/9.tex}
\pagebreak
\input{chapters/10.tex}
\pagebreak
\input{chapters/11.tex}

\end{document}