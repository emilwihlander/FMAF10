% !TeX spellcheck = sv_SE
%http://www.cs.put.poznan.pl/ksiek/latexmath.html
%https://en.wikibooks.org/wiki/LaTeX/Advanced_Mathematics
%http://www.maths.lth.se/matematiklth/personal/magnusa/kurser/endim-ht2015/B1/kurspmB1ht15.pdf

\documentclass[a4paper]{article} 
\usepackage[T1]{fontenc} 
\usepackage[utf8]{inputenc} 
\usepackage[swedish]{babel} 
\usepackage[fleqn]{amsmath}
\usepackage{amssymb}
\usepackage{cancel}
\usepackage{graphicx}
\usepackage{enumitem}
\usepackage{systeme}
\usepackage{environ}
\usepackage[top=1in, bottom=1.25in, left=1.0in, right=1.25in]{geometry}

\setlength{\parindent}{0em}
\setlength{\parskip}{1em}

\newenvironment{task}[1]
{
	\begin{description}[align=right]
		\item [#1]~~~~
}{		%input
	\end{description}
}

\newenvironment{linsys}[1]
{
	\let\oldarraycolsep\arraycolsep
	\setlength\arraycolsep{0pt}
	\left\{\begin{array}{#1}
}{
	\end{array} \right.
	\setlength\arraycolsep{\oldarraycolsep}
}

\newenvironment{detmat}
{
	\left|\begin{matrix}
}{
	\end{matrix}\right|
}

\newenvironment{mat}
{
	\left(\begin{matrix}
	}{
	\end{matrix}\right)
}

\newcommand{\adj}{\text{adj}}
\newcommand{\abs}[1]{\left|#1\right|}

\newcommand{\vek}[1]{\overrightarrow{#1}}

\newcommand\varpm{\mathbin{\vcenter{\hbox{%
				\oalign{\hfil$\scriptstyle+$\hfil\cr
					\noalign{\kern-.5ex}					
					$\scriptscriptstyle({-})$\cr}%
			}}}}

\newcommand{\taskref}[1]{\textbf{#1}}

\newcommand{\chapter}[2]{\section*{Kapitel #1: #2}}

\newcommand{\ans}{\textbf{Svar: }}
\newcommand{\mproof}{\text{~~~~V.S.V}}
\newcommand{\proof}{~~~~V.S.V}

\DeclareMathOperator{\lra}{\ \Leftrightarrow\ }
\DeclareMathOperator{\ra}{\ \Rightarrow\ }
\DeclareMathOperator{\sign}{sign}


\let\*\relax
\DeclareMathOperator{\*}{\cdot}

\let\Re\relax
\let\Im\relax
\DeclareMathOperator{\Re}{Re}
\DeclareMathOperator{\Im}{Im}

\title{Tillämpad matematik - Linjära system\\ FMAF10} 
\author{Emil Wihlander\\ dat15ewi@student.lu.se} 

\begin{document} 
\maketitle
\pagebreak
\chapter{1}{Svängningar och komplexa tal}
\begin{task}{1.1 a)}
Allmänna funktionen för odämpad harmonisk svängning är $u(t)=A\sin(\omega t + \alpha)$ där $\omega$ är vinkelfrekvensen.
\[u(t)=3\sin(2t-5) \ra \omega=2\]
\[T=\frac{2\pi}{\omega} \ra T = \frac{2\pi}{2} = \pi\]
\[f=\frac{1}{T} \ra f = \frac{1}{\pi}\]
\ans vinkelfrekvens: 2, period: $\pi$, frekvens: $\frac{1}{\pi}$
\end{task}

\begin{task}{b)}
	Allmänna funktionen för odämpad harmonisk svängning är $u(t)=A\sin(\omega t + \alpha)$ där $\omega$ är vinkelfrekvensen.
	\[u(t)=50\sin(100\pi t+1) \ra \omega=100\pi\]
	\[T=\frac{2\pi}{\omega} \ra T = \frac{2\pi}{100\pi} = \frac{1}{50}\]
	\[f=\frac{1}{T} \ra f = 50\]
	\ans vinkelfrekvens: $100\pi$, period: $\frac{1}{50}$, frekvens: 50
\end{task}

\begin{task}{1.2 a)}
\end{task}

\begin{task}{b)}
\end{task}

\begin{task}{c)}
\end{task}

\begin{task}{d)}
\end{task}

\begin{task}{e)}
\end{task}

\begin{task}{f)}
\end{task}

\begin{task}{1.3}
	Använd regeln $\sin(\alpha+\beta)=\sin\alpha\cos\beta+\cos\alpha\sin\beta$ från formelbladet.
	\begin{align*}
	u(t)= &
	6\sin(3t+\frac{\pi}{4})=
	6(\sin(3t)\cos(\frac{\pi}{4})+\cos(3t)\sin(\frac{\pi}{4}))= \\ =
	& 6\frac{1}{\sqrt{2}}\sin(3t)+6\frac{1}{\sqrt{2}}\cos(3t)=
	3\sqrt{2}\cos(3t)+3\sqrt{2}\sin(3t)
	\end{align*}
	\ans $a=b=3\sqrt{2}, \omega=3 \ra 3\sqrt{2}\cos(3t)+3\sqrt{2}\sin(3t)$
\end{task}

\pagebreak
\begin{task}{1.4 a)}
	låt $u(t)=A\sin(\omega t + \alpha)=A\sin\alpha\cos(\omega t)+A\cos\alpha\sin(\omega t)=\sqrt{3}\cos(\omega t)-\sin(\omega t)$ där $A$ är amplituden och $\alpha$ är fasförskjutningen.
	\begin{align*}
		\begin{linsys}{rr}
		A\sin\alpha=&\sqrt{3} \\
		A\cos\alpha=&-1
		\end{linsys} \lra &
		\sqrt{(A\sin\alpha)^2+(A\cos\alpha)^2}=\sqrt{(\sqrt{3})^2+(-1)^2} \lra \\ \lra
		& \sqrt{A^2}\sqrt{\sin\alpha^2+\cos\alpha^2}=\sqrt{4} \ra
		A\sqrt{1}=2 \lra
		A=2
	\end{align*}
	\begin{align*}
		\tan\alpha= &
		\frac{\sin\alpha}{\cos\alpha}=
		\frac{A\sin\alpha}{A\cos\alpha}=
		\frac{\sqrt{3}}{-1} \ra \\ \ra
		\alpha= &
		\arctan(-\frac{\sqrt{3}}{1})+\pi=
		-\frac{\pi}{6}+\pi=
		\frac{2\pi}{3}~~(+\pi \text{ ty } -4<0)
	\end{align*}
	eller:
	\begin{align*}
		u(t)= &
		\sqrt{3}\cos(\omega t)-\sin(\omega t)=
		2(\frac{\sqrt{3}}{2}\cos(\omega t)-\frac{1}{2}\sin(\omega t))= \\ =
		& 2(\sin\frac{2\pi}{3}\cos(\omega t)+\cos\frac{2\pi}{3}\sin(\omega t))=
		\sin(\omega t + \frac{2\pi}{3})
	\end{align*}
	\ans Amplitud: $2$ och fasförskjutning: $\frac{2\pi}{3}$
\end{task}

\begin{task}{1.4 b)}
	låt $u(t)=A\sin(\omega t + \alpha)=A\sin\alpha\cos(\omega t)+A\cos\alpha\sin(\omega t)=-2\cos(\omega t)-4\sin(\omega t)$ där $A$ är amplituden och $\alpha$ är fasförskjutningen.
	\begin{align*}
	\begin{linsys}{rr}
	A\sin\alpha=&-2 \\
	A\cos\alpha=&-4
	\end{linsys} \lra &
	\sqrt{(A\sin\alpha)^2+(A\cos\alpha)^2}=\sqrt{(-2)^2+(-4)^2} \lra \\ \lra
	& \sqrt{A^2}\sqrt{\sin\alpha^2+\cos\alpha^2}=\sqrt{4+16} \ra
	A\sqrt{1}=\sqrt{20} \lra
	A=2\sqrt{5}
	\end{align*}
	\begin{align*}
	\tan\alpha= &
	\frac{\sin\alpha}{\cos\alpha}=
	\frac{A\sin\alpha}{A\cos\alpha}=
	\frac{-2}{-4} \ra \\ \ra
	\alpha= &
	\arctan\frac{1}{2}+\pi~~(+\pi \text{ ty } -4<0)
	\end{align*}
	\ans Amplitud: $2\sqrt{5}$ och fasförskjutning: $\arctan\frac{1}{2}+\pi$
\end{task}

\begin{task}{1.5 a)}
	Eftersom $\abs{a+bi}=\sqrt{a^2+b^2}$.
	\[\abs{i}=\sqrt{0^2+1^2}=1\]
	\ans $\abs{i}=1$
\end{task}

\begin{task}{b)}
	Eftersom $\abs{a+bi}=\sqrt{a^2+b^2}$.
	\[\abs{-i}=\sqrt{0^2+(-1)^2}=1\]
	\ans $\abs{-i}=1$
\end{task}

\begin{task}{c)}
	Eftersom $\abs{e^{i\phi}}=1$ oberoende av vad vinkeln $\phi$ är.
	
	\ans $\abs{e^{5\pi i/7}}=1$
\end{task}

\begin{task}{1.6 a)}
	låt $e^{i\phi}=e^{5\pi i/7} \lra \phi=\frac{5\pi}{7}$. Eftersom $\frac{\pi}{2}<\phi<\pi \ra e^{5\pi i/7}$ ligger i andra kvadranten.
	
	\ans andra kvadranten
\end{task}

\begin{task}{b)}
	Låt 
	$e^{i\phi}=e^{-34\pi i/7} \lra 
	\phi=
	-\frac{34}{7}\pi=
	-\frac{35}{7}\pi+\frac{1}{7}\pi=
	-6\pi+\pi+\frac{1}{7}\pi \ra
	\phi=\pi+\frac{1}{7}\pi$.
	Eftersom perioden är $2\pi \ra e^{i\phi} = e^{i\phi}$ vilket innebär $\pi<\phi<\frac{3}{2}\pi \ra e^{-34\pi i/7}$ ligger i tredje kvadranten.
	
	\ans tredje kvadranten
\end{task}

\begin{task}{c)}
	Låt 
	$e^{i\phi}=e^{2000\pi i/13} \lra 
	\phi=
	\frac{2000}{13}\pi=
	\frac{1989}{13}\pi+\frac{11}{13}\pi=
	152\pi+\pi+\frac{11}{13}\pi \ra
	\phi=\pi+\frac{11}{13}\pi$.
	Eftersom perioden är $2\pi \ra e^{i\phi} = e^{i\phi}$ vilket innebär $\frac{3}{2}\pi<\phi<2\pi \ra e^{2000\pi i/13}$ ligger i fjärde kvadranten.
	
	\ans fjärde kvadranten
\end{task}

\begin{task}{1.7 a)}
	Absolutbelopp:
	\[\abs{2-2i}=\sqrt{2^2+(-2)^2}=\sqrt{8}\]
	Argument:
	\[\arctan\left(\frac{-2}{2}\right)+2k\pi=-\frac{\pi}{4}+2k\pi, \qquad k\in\mathbb{Z}\]
\end{task}

\begin{task}{b)}
	Absolutbelopp:
	\[\abs{\sqrt{3}-i}=\sqrt{\sqrt{3}^2+(-1)^2}=\sqrt{4}=2\]
	Argument:
	\[\arctan\left(\frac{-1}{\sqrt{3}}\right)+2k\pi=-\frac{\pi}{6}+2k\pi, \qquad k\in\mathbb{Z}\]
\end{task}

\begin{task}{c)}
	Absolutbelopp:
	\[\abs{1}=1\]
	Argument:
	\[\arctan\left(\frac{0}{1}\right)+2k\pi=2k\pi, \qquad k\in\mathbb{Z}\]
\end{task}

\begin{task}{d)}
	Absolutbelopp:
	\[\abs{-1}=1\]
	Argument:
	\[\arctan\left(\frac{0}{1}\right)+2k\pi=\pi+2k\pi, \qquad k\in\mathbb{Z}\]
\end{task}

\begin{task}{1.8 a)}
	Låt $z = -1-i = re^{i\phi}$ där $r$ är absolutbeloppet och $\phi$ är argumentet.
	\[r=\sqrt{(\Re z)^2+(\Im z)^2}=
	\sqrt{(-1)^2+(-1)^2}=
	\sqrt{2}\]
	\begin{align*}
		\phi=&\arctan\left(\frac{\Im z}{\Re z}\right)+2k\pi, \qquad k \in \mathbb{Z} \qquad \ra \\
		\phi=&\arctan\left(\frac{-1}{-1}\right)+2k\pi=
		\frac{\pi}{4}+\pi+2k\pi=
		\frac{5}{4}\pi+2k\pi, \qquad k \in \mathbb{Z}
	\end{align*}
	\[z=\sqrt{2}e^{i(3\pi/4+2k\pi)}, \qquad k \in \mathbb{Z}\]
	Partikulärlösning:
	\[z=\sqrt{2}e^{i3\pi/4}\]
	\ans $z=\sqrt{2}e^{i3/4\pi}$
\end{task}

\begin{task}{b)}
	Låt $z = i = re^{i\phi}$ där $r$ är absolutbeloppet och $\phi$ är argumentet.
	\[r=\sqrt{(\Re z)^2+(\Im z)^2}=
	\sqrt{0^2+1^2}=
	1\]
	Eftersom $\Re z = 0$ och $\Im z > 0$ är $\phi=\frac{\pi}{2} + 2k\pi \qquad k \in \mathbb{Z}$
	\[z=e^{i(\pi/2+2k\pi)}, \qquad k \in \mathbb{Z}\]
	Partikulärlösning:
	\[z=e^{i\pi/2}\]
	\ans $z=e^{i\pi/2}$
\end{task}

\begin{task}{1.9}
	Utnyttja sambandet $e^{i\theta}=\cos\theta+i\sin\theta$.
	\begin{align*}
		5e^{2\pi i/3}=
		5\left(\cos\left(\frac{2\pi}{3}\right)+i\sin\left(\frac{2\pi}{3}\right)\right)=
		5\left(-\frac{1}{2}+i\frac{\sqrt{3}}{2}\right)=
		-\frac{5}{2}+i\frac{5\sqrt{3}}{2}
	\end{align*}
	\ans $-\frac{5}{2}+i\frac{5\sqrt{3}}{2}$
\end{task}

\begin{task}{1.10 a)}
	Låt $z=re^{i\phi}, \quad r \ge 0$
	\begin{align*}
		z^4+1=0 \lra 
		&(re^{i\phi})^4=-1 \lra
		r^4e^{i4\phi}=e^{\pi+2k\pi}, \quad k\in\mathbb{Z} \lra
		\begin{cases}
		4\phi=\pi+2k\pi, \quad k\in\mathbb{Z} \\
		r^4=1
		\end{cases} \lra \\ \lra
		&\begin{cases}
		\phi=\frac{\pi}{4}+\frac{k\pi}{2}, \quad k\in\mathbb{Z} \\
		r=1
		\end{cases}
	\end{align*}
	$k=\{0,1,2,3\}$ ger alla unika lösningar.
	
	\ans $e^{\pi i/4+k\pi i/2} \qquad k=\{0,1,2,3\}$
	\\ \\
	Eller:
	\\ \\
	Använd $\sqrt{i}=(e^{\pi i/2})^{1/2}=e^{\pi i/4}=\frac{1}{\sqrt{2}}(1+i)$ och $\sqrt{-i}=(e^{-\pi i/2})^{1/2}=e^{-\pi i/4}=\frac{1}{\sqrt{2}}(1-i)$
	\begin{align*}
	z^4+1=0 \lra 
	&z^4=-1 \lra
	\sqrt{z^4}=\pm\sqrt{-1} \lra
	z^2=\pm i \lra
	\sqrt{z^2}=\pm\sqrt{\pm i} \lra \\ \lra
	&z=\pm\frac{1}{\sqrt{2}}(1\pm i)=\frac{1}{\sqrt{2}}(\pm 1\pm i)
	\end{align*}
\end{task}

\begin{task}{b)}
	Låt $z=re^{\phi i}, \quad r \ge 0$
	\begin{align*}
		z^5=32 \lra
		&(re^{\phi i})^5=32 \lra
		r^5e^{5\phi i}=32e^{2k\pi i}, \quad k\in\mathbb{Z} \lra
		\begin{cases}
		5\phi=2k\pi, \quad k\in\mathbb{Z} \\
		r^5=32
		\end{cases} \lra \\ \lra
		&\begin{cases}
		\phi=\frac{2k\pi}{5}, \quad k\in\mathbb{Z} \\
		r=2
		\end{cases}
	\end{align*}
	$k=\{0,1,2,3,4\}$ ger alla unika lösningar.
	
	\ans $e^{2k\pi i/5} \qquad k=\{0,1,2,3,4\}$
\end{task}
\pagebreak
\chapter{2}{Steg och impulsfunktioner}
\begin{task}{2.1 a)}
\end{task}

\begin{task}{b)}
\end{task}

\begin{task}{c)}
\end{task}

\begin{task}{d)}
\end{task}

\begin{task}{e)}
\end{task}

\begin{task}{2.2}
\end{task}

\begin{task}{2.3 a)}
	\[\theta(t-1)\theta(3-t)\]
	eller
	\[\theta(t-1)-\theta(t-3)\]
\end{task}

\begin{task}{b)}
	Funktionen som syns är $-0.5t+1.5$, stegfunktioner som skärmar in $]1,3[$ är (se \taskref{a)}) $\theta(t-1)-\theta(t-3)$ vilket medför:
	
	\ans $(-0.5t+1.5)(\theta(t-1)-\theta(t-3))$
\end{task}

\begin{task}{2.4 a)}
	Funktionen i intervallet $]0,1[$ är $t$. Stegfunktion: $\theta(t)-\theta(t-1)$.
	
	Funktionen i intervallet $]1,2[$ är $1$. Stegfunktion: $\theta(t-1)-\theta(t-2)$.
	
	Funktionen i intervallet $]2,3[$ är $3-t$. Stegfunktion: $\theta(t-2)-\theta(t-3)$ vilket ger:
	
	\ans $t(\theta(t)-\theta(t-1))+\theta(t-1)-\theta(t-2)+(3-t)(\theta(t-2)-\theta(t-3))$
\end{task}

\begin{task}{b)}
	Funktionen i intervallet $]0,1[$ är $t$. Stegfunktion: $\theta(t)-\theta(t-1)$.
	
	Funktionen i intervallet $]1,2[$ är $t-1$. Stegfunktion: $\theta(t-1)-\theta(t-2)$ vilket ger:
	
	\ans $t(\theta(t)-\theta(t-1))+(t-1)(\theta(t-1)-\theta(t-2))$
\end{task}

\begin{task}{2.5}
	\[p_b(t)=\frac{1}{b}(\theta(t)-\theta(t-b))\]
	Om stegfunktioner finns som en faktor i en integral kan dessa ersätta integrationsgränserna eftersom de evaluerar till noll utanför intervallet.
	\[\int_{-\infty}^{+\infty}\! (\theta(t-a)-\theta(t-b))t \, \mathrm{d}t =
	\int_{a}^{b}\! t \, \mathrm{d}t\]
	Lös med hjälp av ovanstående samband:
	\begin{align*}
	\int_{-\infty}^{+\infty}\! p_b(t)e^{-st} \, \mathrm{d}t =
	&\int_{-\infty}^{+\infty}\!\frac{1}{b}(\theta(t)-\theta(t-b))e^{-st} \, \mathrm{d}t =
	\int_{0}^{b}\!\frac{1}{b}e^{-st} \, \mathrm{d}t = \\ =
	&\left[-\frac{1}{sb}e^{-st}\right]_{0}^{b}=
	-\frac{e^{-sb}}{sb}-\left(-\frac{1}{sb}\right)=
	\frac{1-e^{-sb}}{sb}\cond{s\neq0}
	\end{align*}
	Om $s=0$:
	\[\int_{-\infty}^{+\infty}\! p_b(t)\*1 \, \mathrm{d}t =1 \qquad\text{enligt def., se boken}\]'
	\ans $\int_{-\infty}^{+\infty}\! p_b(t)e^{-st} \, \mathrm{d}t =\frac{1}{sb}(1-e^{-sb})\cond{s\neq0}$ och $1\cond{s=0}$
\end{task}
\pagebreak
\chapter{3}{Laplacetransformationer}

\begin{task}{3.1 a)}
	$f(t)=e^{-2t}\theta(t)$
	
	Se definitionen av Lapacetransformen i boken.
	\begin{align*}
	\laplace f(s)=
	&\int_{-\infty}^{+\infty}\! e^{-st}f(t) \, dt=
	\int_{-\infty}^{+\infty}\! e^{-st}e^{-2t}\theta(t) \, dt=
	\int_{0}^{+\infty}\! e^{-(s+2)t} \, dt=
	\left[-\frac{e^{-(s+2)t}}{s+2}\right]_0^{+\infty}= \\ =
	&\lim\limits_{T\rightarrow\infty}\frac{1}{s+2}(1-e^{-(s+2)T})
	\end{align*}
	Om $s>-2$ gäller att $e^{-(s+2)T} \rightarrow 0$ när $T \rightarrow \infty$ vilket medför:
	\[\laplace f(s)=\frac{1}{s+2}\cond{s>-2}\]
	Om $s=-2$:
	\[\laplace f(s)=
	\int_{0}^{+\infty}\! 1 \, dt=
	\left[t\right]_0^{+\infty}\rightarrow\infty\]
	Om $s<-2$ gäller att $e^{-(s+2)T} \rightarrow \infty$ när $T \rightarrow \infty$ vilket medför:
	\[\laplace f(s)\rightarrow-\infty\]
	Detta medför att $\laplace f(s)$ endast är konvergent när $s>-2$ och därmed är Laplacetransformen för $e^{-2t}\theta(t)$ endast definierad i det intervallet.
	
	Låt nu $s$ vara ett komplext tal, $s=a+bi$:
	\[\abs{e^{-(a+bi+2)t}}=
	\abs{e^{-(a+2)t}}\underbrace{\abs{e^{-ibt}}}_{=1}=
	e^{-(a+2)t}\]
	Här ser vi att $e^{-(s+2)t}\rightarrow 0$ då $t\rightarrow \infty$ om $a = \Re s > -2$ vilket utvidgar Laplacetransformen att inkludera hela planet $\Re s > -2$.
	
	\ans $\laplace f(s)= \dfrac{1}{s+2}\cond{\Re s > -2}$
\end{task}

\begin{task}{3.1 b)}
	$f(t)=\theta(t)-\theta(t-1)$
	
	Se definitionen av Lapacetransformen i boken.
	\begin{align*}
	\laplace f(s)=
	&\int_{-\infty}^{+\infty}\! e^{-st}f(t) \, dt=
	\int_{-\infty}^{+\infty}\! e^{-st}(\theta(t)-\theta(t-1)) \, dt=
	\int_{0}^{1}\! e^{-st} \, dt=
	\left[-\frac{e^{-st}}{s}\right]_0^{1}= \\ =
	&-\frac{e^{-s}}{s}+\frac{1}{s}=
	\frac{1-e^{-s}}{s}\cond{s\neq0}
	\end{align*}
	Om $s=0$ gäller att $e^{-st} = 1$ vilket medför:
	\[\laplace f(0)=
	\int_{0}^{1}\! 1 \, dt=
	\left[t\right]_0^1=
	1-0=1\]
	
	\ans $\laplace f(s)= \dfrac{1}{s}(1-e^{-s})\cond{s\neq0}$ och $\laplace f(0)= 1$
\end{task}

\begin{task}{3.2}
	\begin{align*}
	f(at)\longleftrightarrow
	&\int_{-\infty}^{+\infty}\! e^{-st}f(at) \, dt=
	\left[\begin{array}{l}
	x=at \\
	dt=dx/a \\
	t=-\infty\leftrightarrow x=-\infty \\
	t=+\infty\leftrightarrow x=+\infty
	\end{array}\right] =
	\int_{-\infty}^{+\infty}\! e^{-sx/a}f(x) \, \frac{dx}{a}= \\ =
	&\frac{1}{a}\int_{-\infty}^{+\infty}\! e^{-(s/a)x}f(x) \, dx=
	\frac{1}{a}\laplace f\left(\frac{s}{a}\right)\mproof
	\end{align*}
\end{task}

\begin{task}{3.3 a)}
	Låt $s=\sigma+i\omega$.
	
	Använd $\theta(t)\longleftrightarrow \dfrac{1}{s}\cond{\sigma > 0}$ och $t^n\theta(t)\longleftrightarrow\dfrac{n!}{s^{n+1}}\cond{\sigma > 0}$.
	\begin{align*}
	f(t)=
	(2+3t^2)\theta(t)=
	2\theta(t)+3t^2\theta(t)\longleftrightarrow
	2\frac{1}{s}+3\frac{2}{s^3}=
	\frac{2s^2+6}{s^3}\cond{\sigma > 0}
	\end{align*}
	\ans $(2+3t^2)\theta(t)\longleftrightarrow\dfrac{2s^2+6}{s^3}\cond{\sigma > 0}$
\end{task}

\begin{task}{3.3 b)}
	Låt $s=\sigma+i\omega$.
	
	Använd $t^ne^{at}\theta(t)\longleftrightarrow \dfrac{n!}{(s-a)^{n+1}}\cond{\sigma > \Re a}$.
	\begin{align*}
	f(t)=
	e^{3t}\theta(t)=
	t^0e^{3t}\theta(t)\longleftrightarrow
	\frac{0!}{(s-3)^{1}}=
	\frac{1}{s-3}\cond{\sigma > 3}
	\end{align*}
	\ans $e^{3t}\theta(t)\longleftrightarrow\dfrac{1}{s-3}\cond{\sigma > 3}$
\end{task}

\begin{task}{3.3 c)}
	Låt $s=\sigma+i\omega$.
	
	Använd $t^ne^{at}\theta(t)\longleftrightarrow \dfrac{n!}{(s-a)^{n+1}}\cond{\sigma > \Re a}$.
	\begin{align*}
	f(t)=
	te^{3t}\theta(t)=
	t^1e^{3t}\theta(t)\longleftrightarrow
	\frac{1!}{(s-3)^{1+1}}=
	\frac{1}{(s-3)^2}\cond{\sigma > 3}
	\end{align*}
	\ans $te^{3t}\theta(t)\longleftrightarrow\dfrac{1}{(s-3)^2}\cond{\sigma > 3}$
\end{task}

\begin{task}{3.3 d)}
	Låt $s=\sigma+i\omega$.
	
	Använd $t^ne^{at}\theta(t)\longleftrightarrow \dfrac{n!}{(s-a)^{n+1}}\cond{\sigma > \Re a}$.
	\begin{align*}
	f(t)=
	t^2e^{3t}\theta(t)\longleftrightarrow
	\frac{2!}{(s-3)^{2+1}}=
	\frac{2}{(s-3)^3}\cond{\sigma > 3}
	\end{align*}
	\ans $t^2e^{3t}\theta(t)\longleftrightarrow\dfrac{2}{(s-3)^3}\cond{\sigma > 3}$
\end{task}
\pagebreak
\chapter{4}{Den inversa Laplacetransformen}

\begin{task}{4.1}
	multiplicera faktorerna.
	\begin{align*}
	F(s)=
	&\frac{(s-1)(s+1)}{(s+2)^2(s+1+2i)(s+1-2i)}=
	\frac{s^2-1}{(s^2+2s+4)((s+1)^2+4)}= \\ =
	&\frac{s^2-1}{(s^2+4s+4)(s^2+2s+5)}=
	\frac{s^2-1}{s^4+6s^3+17s^2+28s+20}
	\end{align*}
\end{task}

\begin{task}{4.2 a)}
	Använd tabellen:
	\[\laplace f(s)=\frac{1}{s}\llra \theta(t)=f(t)\]
	\ans $f(t)=\theta(t)$
\end{task}

\begin{task}{b)}
	Faktorisera och använd tabellen:
	\[\laplace f(s)=\frac{1}{s^3}=\frac{1}{2}\*\frac{2}{s^{2+1}}\llra \frac{1}{2}t^2\theta(t)=f(t)\]
	\ans $f(t)=\dfrac{1}{2}t^2\theta(t)$
\end{task}

\begin{task}{c)}
	Faktorisera och använd tabellen:
	\begin{align*}
	\laplace f(s)=
	&\frac{s^4+6s^3-10s^2+1}{s^5}=
	\frac{1}{s}+6\frac{1}{s^2}-10\frac{1}{s^3}+\frac{1}{s^5}=
	\frac{1}{s}+6\frac{1}{s^2}-5\frac{2!}{s^3}+\frac{1}{4!}\*\frac{4!}{s^5}\llra \\ \llra
	&\theta(t)+6t\theta(t)-5t^2\theta(t)+\frac{1}{24}t^4\theta(t)=
	(1+6t-5t^2+\frac{1}{24}t^4)\theta(t)=
	f(t)
	\end{align*}
	\ans $f(t)=(1+6t-5t^2+\frac{1}{24}t^4)\theta(t)$
\end{task}

\begin{task}{4.2 a)}
	Faktorisera och använd tabellen:
	\[\laplace f(s)=\frac{2}{s+3}=2\frac{0!}{(s-(-3))^{0+1}}\llra 2e^{-3t}\theta(t)=f(t)\]
	\ans $f(t)=2e^{-3t}\theta(t)$
\end{task}

\begin{task}{b)}
	Faktorisera och använd tabellen:
	\begin{align*}
	\laplace f(s)=
	&\frac{1}{s+3}-\frac{2}{(s+3)^2}+\frac{1}{(s+3)^3}=
	\frac{1}{s+3}-2\frac{1}{(s+3)^2}+\frac{1}{2}\*\frac{2}{(s+3)^3}\llra \\ \llra
	&e^{-3t}\theta(t)-2te^{-3t}\theta(t)+\frac{1}{2}t^2e^{-3t}\theta(t)=
	(1-2t+\frac{1}{2}t^2)e^{-3t}\theta(t)=
	f(t)
	\end{align*}
	\ans $f(t)=(1-2t+\frac{1}{2}t^2)e^{-3t}\theta(t)$
\end{task}

\begin{task}{c)}
	Faktorisera och använd tabellen:
	\begin{align*}
	\laplace f(s)=
	&\frac{s+5}{(s+3)^2}=
	\frac{s+3+2}{(s+3)^2}=
	\frac{s+3}{(s+3)^2}+\frac{2}{(s+3)^2}=
	\frac{1}{s+3}+2\frac{1}{(s+3)^2}\llra \\ \llra
	&e^{-3t}\theta(t)+2te^{-3t}\theta(t)=
	(1+2t)e^{-3t}\theta(t)=
	f(t)
	\end{align*}
	\ans $f(t)=(1+2t)e^{-3t}\theta(t)$
\end{task}

\begin{task}{4.4 a)}
	Faktorisera och använd tabellen:
	\begin{align*}
	\laplace f(s)=
	&\frac{s}{s^2+6s+8}=
	\frac{s}{(s+4)(s+2)}=
	\frac{2}{s+4}-\frac{1}{s+2}=
	2\frac{1}{s+4}-\frac{1}{s+2}\llra  \\ \llra
	&2e^{-4t}\theta(t)-e^{-2t}\theta(t)=
	(2e^{-4t}-e^{-2t})\theta(t)=
	f(t)
	\end{align*}
	\ans $f(t)=(2e^{-4t}-e^{-2t})\theta(t)$
\end{task}

\begin{task}{b)}
	Kvadratkomplettera, faktorisera och använd tabellen:
	\begin{align*}
	\laplace f(s)=
	&\frac{s}{s^2+6s+10}=
	\frac{s}{(s+3)^2+1}=
	\frac{s+3}{(s+3)^2+1}-3\frac{1}{(s+3)^2+1}
	\end{align*}
	Använd dämpningsregeln:
	\[\frac{s+3}{(s+3)^2+1}\llra e^{-3t}\cos(t)\theta(t)\]
	\[\frac{1}{(s+3)^2+1}\llra e^{-3t}\sin(t)\theta(t)\]
	\[f(t)=e^{-3t}\cos(t)\theta(t)-3e^{-3t}\sin(t)\theta(t)=(\cos t-3\sin t)e^{-3t}\theta(t)\]
	\ans $f(t)=(\cos t-3\sin t)e^{-3t}\theta(t)$
\end{task}

\begin{task}{4.5 a)}
	Använd tabellen:
	\begin{align*}
	\laplace f(s)=
	&\frac{1}{s^2+16}=
	\frac{1}{4}\*\frac{4}{s^2+4^4} \llra 
	\frac{1}{4}\sin(4t)\theta(t)=f(t)
	\end{align*}
	\ans $f(t)=\frac{1}{4}\sin(4t)\theta(t)$
\end{task}

\begin{task}{b)}
	Använd tabellen:
	\begin{align*}
	\laplace f(s)=
	&\frac{s}{s^2+16}=
	\frac{s}{s^2+4^4} \llra 
	\cos(4t)\theta(t)=f(t)
	\end{align*}
	\ans $f(t)=\cos(4t)\theta(t)$
\end{task}

\begin{task}{c)}
	Kvadratkomplettera och använd dämpningsregeln och tabellen:
	\begin{align*}
	\laplace f(s)=
	&\frac{1}{s^2+4s+8}=
	\frac{1}{(s+2)^2+2^2}=
	\frac{1}{2}\*\frac{2}{(s+2)^2+2^2} \llra \\ \llra
	&\frac{1}{2}e^{-2t}\sin(2t)\theta(t)=
	f(t)
	\end{align*}
	\ans $f(t)=\frac{1}{2}e^{-2t}\sin(2t)\theta(t)$
\end{task}

\begin{task}{d)}
	Kvadratkomplettera, faktorisera och använd dämpningsregeln och tabellen:
	\begin{align*}
	\laplace f(s)=
	&\frac{s}{s^2+4s+8}=
	\frac{s}{(s+2)^2+2^2}=
	\frac{s+2}{(s+2)^2+2^2}-\frac{2}{(s+2)^2+2^2} \llra \\ \llra
	&e^{-2t}\cos(2t)\theta(t)-e^{-2t}\sin(2t)\theta(t)=
	(\cos(2t)-\sin(2t))e^{-2t}\theta(t)=
	f(t)
	\end{align*}
	\ans $f(t)=(\cos(2t)-\sin(2t))e^{-2t}\theta(t)$
\end{task}

\begin{task}{4.6 a)}
	Hitta nollställena:
	\[s^2-s-2=0 \lra
	s=\frac{1}{2}\pm\sqrt{\frac{1}{4}+2}=
	\frac{1}{2}\pm\frac{3}{2}=\{2,-1\}\]
	Faktorisera och använd dämpningsregeln och tabellen:
	\begin{align*}
	\laplace f(s)=
	&\frac{s+3}{s^2-s-2}=
	\frac{s+3}{(s-2)(s+1)}=
	\frac{k_1}{s-2}+\frac{k_2}{s+1}
	\end{align*}
	\[k_1(s+1)+k_2(s-2)=s+3 \lra
	\begin{cases}
	k_1+k_2=1 \\
	k_2-2k_1=3
	\end{cases} \lra
	\begin{cases}
	k_1=-\frac{2}{3} \\
	k_2=\frac{5}{3}
	\end{cases} \]
	\begin{align*}
	\laplace f(s)=
	-\frac{2}{3}\*\frac{1}{s-2}+\frac{5}{3}\*\frac{1}{s+1} \llra
	-\frac{2}{3}e^{-2t}\theta(t)+\frac{5}{3}e^{t}\theta(t)=
	(5e^{t}-2e^{-2t})\frac{1}{3}\theta(t)
	\end{align*}
	\ans $f(t)=(5e^{t}-2e^{-2t})\frac{1}{3}\theta(t)$
\end{task}

\begin{task}{b)}
	Hitta nollställena:
	\[s^2+3s+2=0 \lra
	s=-\frac{3}{2}\pm\sqrt{\frac{9}{4}-2}=
	-\frac{3}{2}\pm\frac{1}{2}=\{-1,-2\}\]
	Faktorisera och använd dämpningsregeln och tabellen:
	\begin{align*}
	\laplace f(s)=
	&\frac{3s+5}{s^3+3s^2+2s}=
	\frac{3s+5}{s(s+1)(s+2)}=
	\frac{k_1}{s}+\frac{k_2}{s+1}+\frac{k_3}{s+2}
	\end{align*}
	\[k_1(s+1)(s+2)+k_2s(s+2)+k_3s(s+1)=3s+5 \lra
	\begin{cases}
	k_1+k_2+k_3=0 \\
	3k_1+2k_2+k_3=3 \\
	2k_1=5
	\end{cases} \lra
	\begin{cases}
	k_1=\frac{5}{2} \\
	k_2=-2 \\
	k_3=-\frac{1}{2}
	\end{cases} \]
	\begin{align*}
	\laplace f(s)=
	&\frac{5}{2}\*\frac{1}{s}-2\*\frac{1}{s+1}-\frac{1}{2}\*\frac{1}{s+2} \llra
	\frac{5}{2}\theta(t)-2e^{-t}\theta(t)-\frac{1}{2}e^{-2t}\theta(t)= \\ =
	&\frac{1}{2}(5-4e^{-t}-e^{-2t})\theta(t)
	\end{align*}
	\ans $f(t)=\frac{1}{2}(5-4e^{-t}-e^{-2t})\theta(t)$
\end{task}

\begin{task}{c)}
	Hitta nollställena:
	\[s^2+4s+3=0 \lra
	s=-2\pm\sqrt{4-3}=
	-2\pm1=\{-1,-3\}\]
	Faktorisera och använd dämpningsregeln och tabellen:
	\begin{align*}
	\laplace f(s)=
	&\frac{s^3-5s}{s^2+4s+3}=
	s\frac{s^2-5}{(s+1)(s+3)}=
	s\left(\frac{k_1}{s+1}+\frac{k_2}{s+3}+k_3\right)
	\end{align*}
	\[k_1(s+3)+k_2(s+1)+k_3(s+3)(s+1)=s^2-5 \lra
	\begin{cases}
	k_3=1 \\
	k_1+k_2+4k_3=0 \\
	3k_1+k_2+3k_3=-5
	\end{cases} \lra
	\begin{cases}
	k_1=-2 \\
	k_2=-2 \\
	k_3=1
	\end{cases} \]
	Använd regeln $sF(s)\llra f'(t)$ och $\delta(t)=1$ för att bestämma inversen till $s$.
	\[s=s\*1\llra\delta'(t)\]
	\begin{align*}
	\laplace f(s)=
	&-2\frac{s}{s+1}-2\frac{s}{s+3}+s =
	-2\frac{s+1}{s+1}+2\frac{1}{s+1}-2\frac{s+3}{s+3}+6\frac{1}{s+3}+s= \\ =
	&-4+2\frac{1}{s+1}+6\frac{1}{s+3}+s\llra
	-4\delta(t)+2e^{-t}\theta(t)+6e^{-3t}\theta(t)+\delta'(t)= \\ =
	&\delta'(t)-4\delta(t)+(2e^{-t}+6e^{-3t})\theta(t)
	\end{align*}
	\ans $f(t)=\delta'(t)-4\delta(t)+(2e^{-t}+6e^{-3t})\theta(t)$
\end{task}

\begin{task}{4.7}
	En av polerna är $s=-1$ (se anvisningen).
	\begin{align*}
	&s^3+5s^2+9s+5=(s+1)(s^2+As+B)=s^3+(A+1)s^2+(B+A)s+B \lra \\ \lra
	&\begin{cases}
	A+1=5 \\
	B+A=9 \\
	B=5
	\end{cases} \lra
	\begin{cases}
	A=4 \\
	B=5
	\end{cases} \lra
	s^3+5s^2+9s+5=(s+1)(s^2+4s+5)
	\end{align*}
	Faktorisera och använd dämpningsregeln och tabellen:
	\[\frac{s}{s^3+5s^2+9s+5}=\frac{k_1}{s+1}+\frac{k_2s+k_3}{s^2+4s+5}\]
	\[k_1(s^2+4s+5)+(k_2s+k_3)(s+1)=s \lra
	\begin{cases}
	k_1+k_2=0 \\
	4k_1+k_2+k_3=1 \\
	5k_1+k_3=0
	\end{cases} \lra
	\begin{cases}
	k_1=-\frac{1}{2} \\
	k_2=\frac{1}{2} \\
	k_3=\frac{5}{2}
	\end{cases} \]
	\begin{align*}
	\frac{s}{s^3+5s^2+9s+5}=
	&-\frac{1}{2}\*\frac{1}{s+1}+\frac{1}{2}\*\frac{s+5}{s^2+4s+5}=
	-\frac{1}{2}\*\frac{1}{s+1}+\frac{1}{2}\*\frac{s+2+3}{(s+2)^2+1}= \\ =
	&-\frac{1}{2}\*\frac{1}{s+1}+\frac{1}{2}\*\frac{s+2}{(s+2)^2+1}+\frac{3}{2}\*\frac{1}{(s+2)^2+1} \llra \\ \llra
	&-\frac{1}{2}e^{-t}\theta(t)+\frac{1}{2}e^{-2t}\cos(t)\theta(t)+\frac{3}{2}e^{-2t}\sin(t)\theta(t)= \\ =
	&\frac{1}{2}(-e^{-t}+(\cos t+3\sin t)e^{-2t})\theta(t)
	\end{align*}
	\ans $\laplace^{-1}\left(\dfrac{s}{s^3+5s^2+9s+5}\right)=\dfrac{1}{2}(-e^{-t}+(\cos t+3\sin t)e^{-2t})\theta(t)$
\end{task}

\begin{task}{4.8}
	Faktorisera och använd tabellen:
	\[\frac{1}{s^3(s^2+1)}=\frac{k_1s+k_2}{s^2+1}+\frac{k_3}{s}+\frac{k_4}{s^2}+\frac{k_5}{s^3}\]
	\[s^3(k_1s+k_2)+k_3s^2(s^2+1)+k_4s(s^2+1)+k_5(s^2+1)=1 \lra
	\begin{cases}
	k_1+k_3=0 \\
	k_2+k_4=0 \\
	k_3+k_5=0 \\
	k_4=0 \\
	k_5=1
	\end{cases} \lra
	\begin{cases}
	k_1=1 \\
	k_2=0 \\
	k_3=-1 \\
	k_4=0 \\
	k_5=1
	\end{cases} \]
	\begin{align*}
	\frac{1}{s^3(s^2+1)}=
	\frac{s}{s^2+1}-\frac{1}{s}+\frac{1}{s^3}=
	\frac{s}{s^2+1}-\frac{1}{s}+\frac{1}{2}\*\frac{2}{s^3} \llra
	&\cos(t)\theta(t)-\theta(t)+\frac{1}{2}t^2\theta(t) = \\ =
	&\left(\cos(t)-1+\frac{1}{2}t^2\right)\theta(t)
	\end{align*}
	\ans $\laplace^{-1}\left(\dfrac{1}{s^3(s^2+1)}\right)=\left(\cos(t)-1+\frac{1}{2}t^2\right)\theta(t)$
\end{task}

\begin{task}{4.9 a)}
	Låt $\laplace g(s)=\dfrac{1}{s}\llra\theta(t)$ och använd reglerna för dämpning och förskjutning.
	\[\laplace f(s)=\frac{e^{-s}}{s+1}=e^{-s}\laplace g(s+1) \llra e^{-(t-1)}g(t-1)=e^{1-t}\theta(t-1)\]
	\ans $f(t)=e^{1-t}\theta(t-1)$
\end{task}

\begin{task}{b)}
	Se \taskref{4.6 b)} för uträkning av nollställena.
	\[\laplace f(s)=\frac{se^{-s}}{s^2+3s+2}=e^{-s}\left(\frac{k_1}{s+1}+\frac{k_2}{s+2}\right)\]
	\[k_1(s+2)+k_2(s+1)=s\lra
	\begin{cases}
	k_1+k_2=1 \\
	2k_1+k_2=0
	\end{cases}\lra
	\begin{cases}
	k_1=-1 \\
	k_2=2
	\end{cases}\]
	Använd reglerna för dämpning och förskjutning:
	\begin{align*}
	\laplace f(s)=
	e^{-s}\left(-\frac{1}{s+1}+2\frac{1}{s+2}\right)\llra
	&e^{-(t-1)}\theta(t-1)+2e^{-2(t-1)}\theta(t-1)= \\ =
	&(-e^{1-t}+2e^{2-2t})\theta(t-1)
	\end{align*}
	\ans $f(t)=(-e^{1-t}+2e^{2-2t})\theta(t-1)$
\end{task}

\begin{task}{c)}
	Se \taskref{b)} för beräkning av den andra termen.
	\[\laplace f(s)=\frac{1+se^{-s}}{s^2+3s+2}=\frac{1}{(s+1)(s+2)}+\frac{se^{-s}}{(s+1)(s+2)}=\frac{k_1}{s+1}+\frac{k_2}{s+2}+e^{-s}\left(-\frac{1}{s+1}+2\frac{1}{s+2}\right)\]
	\[k_1(s+2)+k_2(s+1)=1\lra
	\begin{cases}
	k_1+k_2=0 \\
	2k_1+k_2=1
	\end{cases}\lra
	\begin{cases}
	k_1=1 \\
	k_2=-1
	\end{cases}\]
	Använd regeln för dämpning (samt \taskref{b)}):
	\begin{align*}
	\laplace f(s)=
	&\frac{1}{s+1}-\frac{1}{s+2}+e^{-s}\left(-\frac{1}{s+1}+2\frac{1}{s+2}\right)\llra \\ \llra
	&e^{-t}\theta(t)-e^{-2}\theta(t)+(-e^{1-t}+2e^{2-2t})\theta(t-1)= \\ =
	&(e^{-t}-e^{-2t})\theta(t)+(-e^{1-t}+2e^{2-2t})\theta(t-1)
	\end{align*}
	\ans $f(t)=(e^{-t}-e^{-2t})\theta(t)+(-e^{1-t}+2e^{2-2t})\theta(t-1)$
\end{task}

\begin{task}{4.10 a)}
	Låt $\laplace g(s)=\dfrac{1}{s}\llra\theta(t)$ och använd reglerna för dämpning och förskjutning.
	\[\laplace f(s)=e^{-5s}\frac{1}{s+2}=e^{-5s}\laplace g(s+2) \llra e^{-2(t-5)}g(t-5)=e^{2(5-t)}\theta(t-5)\]
	\ans $f(t)=e^{2(5-t)}\theta(t-5)$
\end{task}

\begin{task}{b)}
	Faktorisera och använd regeln för förskjutning.
	\begin{align*}
	\laplace f(s)=
	&(e^{-\pi s}+e^{-2\pi s})\frac{1}{s^2+1}=
	e^{-\pi s}\frac{1}{s^2+1}+e^{-2\pi s}\frac{1}{s^2+1} \llra \\ \llra
	&\sin(t-\pi)\theta(t-\pi)+\sin(t-2\pi)\theta(t-2\pi)
	\end{align*}
	\ans $f(t)=\sin(t-\pi)\theta(t-\pi)+\sin(t-2\pi)\theta(t-2\pi)$
\end{task}

\begin{task}{c)}
	Faktorisera och använd reglerna för förskjutning och dämpning.
	\begin{align*}
	\laplace f(s)=
	&\frac{e^{2s}}{s^2+s}=
	e^{2s}\left(\frac{k_1}{s}+\frac{k_2}{s+1}\right)
	\end{align*}
	\[k_1(s+1)+k_2s=1\lra
	\begin{cases}
	k_1+k_2=0 \\
	k_1=1
	\end{cases}\lra
	\begin{cases}
	k_1=1 \\
	k_2=-1
	\end{cases}\]
	\begin{align*}
	\laplace f(s)=
	e^{2s}\left(\frac{1}{s}-\frac{1}{s+1}\right)=
	e^{2s}\frac{1}{s}-e^{2s}\frac{1}{s+1} \llra
	&\theta(t+2)-e^{-(t+2)}\theta(t+2)= \\ =
	&(1-e^{-t-2})\theta(t+2)
	\end{align*}
	\ans $f(t)=(1-e^{-t-2})\theta(t+2)$
\end{task}

\begin{task}{d)}
	Faktorisera och använd reglerna för förskjutning och dämpning.
	\begin{align*}
	\laplace f(s)=
	&\frac{2-2e^{-s}-se^{-s}}{s^2-1}= \\ =
	&\left(\frac{1}{s-1}-\frac{1}{s+1}\right)-e^{-s}\left(\frac{1}{s-1}-\frac{1}{s+1}\right)-\frac{1}{2}e^{-s}\left(\frac{1}{s-1}+\frac{1}{s+1}\right)= \\ =
	&\frac{1}{s-1}-\frac{1}{s+1}-\frac{3}{2}e^{-s}\frac{1}{s-1}+\frac{1}{2}e^{-s}\frac{1}{s+1} \llra \\ \llra
	&e^{t}\theta(t)-e^{-t}\theta(t)-\frac{3}{2}e^{t-1}\theta(t-1)+\frac{1}{2}e^{-(t-1)}\theta(t-1)= \\ =
	&(e^{t}-e^{-t})\theta(t)+\frac{1}{2}(e^{1-t}-3e^{t-1})\theta(t-1)
	\end{align*}
	\ans $f(t)=(e^{t}-e^{-t})\theta(t)+\frac{1}{2}(e^{1-t}-3e^{t-1})\theta(t-1)$
\end{task}

\begin{task}{4.11 a)}
	Bygynnelsevärdessatsen kan inte användas eftersom $F(s)$ inte uppfyller villkoret om att vara ett äkta bråk (både nämnaren och täljaren har graden 2).
\end{task}

\begin{task}{b)}
	\begin{align*}
	\lim\limits_{s\rightarrow +\infty}sF(s)=
	&\lim\limits_{s\rightarrow +\infty}\frac{s^2}{(s+1)(s-2)}=
	\lim\limits_{s\rightarrow +\infty}\frac{s^2}{s^2-s-2}=
	\lim\limits_{s\rightarrow +\infty}\frac{\frac{s^2}{s^2}}{\frac{s^2}{s^2}-\frac{s}{s^2}-\frac{2}{s^2}}= \\ =
	&\lim\limits_{s\rightarrow +\infty}\frac{1}{1-\frac{1}{s}-\frac{2}{s^2}}\rightarrow
	\frac{1}{1-0-0}=
	1=\lim\limits_{t\rightarrow +0}f(t)
	\end{align*}
	\ans $\lim\limits_{t\rightarrow +0}f(t)=1$
\end{task}

\begin{task}{c)}
	\begin{align*}
	\lim\limits_{s\rightarrow +\infty}sF(s)=
	&\lim\limits_{s\rightarrow +\infty}\frac{s}{s(s+1)(s+2)}=
	\lim\limits_{s\rightarrow +\infty}\frac{1}{s^2+3s+2}=
	\lim\limits_{s\rightarrow +\infty}\frac{\frac{1}{s^2}}{\frac{s^2}{s^2}+\frac{3s}{s^2}+\frac{2}{s^2}}= \\ =
	&\lim\limits_{s\rightarrow +\infty}\frac{\frac{1}{s^2}}{1+\frac{3}{s}+\frac{2}{s^2}}\rightarrow
	\frac{0}{1+0+0}=
	0=\lim\limits_{t\rightarrow +0}f(t)
	\end{align*}
	\ans $\lim\limits_{t\rightarrow +0}f(t)=0$
\end{task}

\begin{task}{d)}
	\begin{align*}
	\lim\limits_{s\rightarrow +\infty}sF(s)=
	&\lim\limits_{s\rightarrow +\infty}\frac{s(s^2+3s+2)}{(s+1)^3}=
	\lim\limits_{s\rightarrow +\infty}\frac{s^3+3s^2+2s}{s^3+3s^2+3s+1}=
	\lim\limits_{s\rightarrow +\infty}\frac{\frac{s^3}{s^3}+\frac{3s^2}{s^3}+\frac{2s}{s^3}}{\frac{s^3}{s^3}+\frac{3s^2}{s^3}+\frac{3s}{s^3}+\frac{1}{s^3}}= \\ =
	&\lim\limits_{s\rightarrow +\infty}\frac{1+\frac{3}{s}+\frac{2}{s^2}}{1+\frac{3}{s}+\frac{3}{s^2}+\frac{1}{s^3}}\rightarrow
	\frac{1+0+0}{1+0+0+0}=
	1=\lim\limits_{t\rightarrow +0}f(t)
	\end{align*}
	\ans $\lim\limits_{t\rightarrow +0}f(t)=1$
\end{task}

\begin{task}{e)}
	\begin{align*}
	\lim\limits_{s\rightarrow +\infty}sF(s)=
	&\lim\limits_{s\rightarrow +\infty}\frac{s}{s(s+1)(s^2+1)}=
	\lim\limits_{s\rightarrow +\infty}\frac{1}{s^3+s^2+s+1}=
	\lim\limits_{s\rightarrow +\infty}\frac{\frac{1}{s^3}}{\frac{s^3}{s^3}+\frac{s^2}{s^3}+\frac{s}{s^3}+\frac{1}{s^3}}= \\ =
	&\lim\limits_{s\rightarrow +\infty}\frac{\frac{1}{s^3}}{1+\frac{1}{s}+\frac{1}{s^2}+\frac{1}{s^3}}\rightarrow
	\frac{0}{1+0+0+0}=
	0=\lim\limits_{t\rightarrow +0}f(t)
	\end{align*}
	\ans $\lim\limits_{t\rightarrow +0}f(t)=0$
\end{task}

\begin{task}{4.12 a)}
	Faktorera och förenkla:
	\[sF(s)=
	s\frac{s^2+3s+2}{(s+1)(s+3)}=
	s\frac{(s+1)(s+2)}{(s+1)(s+3)}=
	s\frac{s+2}{s+3}\]
	Polen till $sF(s)$ är negativa.
	\begin{align*}
	\lim\limits_{s\rightarrow 0}sF(s)=
	\lim\limits_{s\rightarrow 0}s\frac{s^2+3s+2}{(s+1)(s+3)}\rightarrow
	0\*\frac{0+0+2}{(0+1)(0+3)}=
	0=\lim\limits_{t\rightarrow +\infty}f(t)
	\end{align*}
	\ans $\lim\limits_{t\rightarrow +\infty}f(t)=0$
\end{task}

\begin{task}{b)}
	Faktorera och förenkla:
	\[sF(s)=
	s\frac{s}{(s+1)(s-2)}\]
	Båda polerna till $sF(s)$ är inte negativa och därmed gäller inte slutvärdessatsen.
\end{task}
\pagebreak
\chapter{5}{Lösning av differentialekvationer genom Laplacetransformation}

\begin{task}{5.1}
	Låt $Y(s)=\laplace(\theta y)(s)$ vilket medför att:
	\[\laplace(\theta y')(s)=s\laplace(\theta y)(s)-y(0)=sY(s)-1\]
	\[\laplace(\theta y'')(s)=s\laplace(\theta y')(s)-y'(0)=s(sY(s)-1)-2=s^2Y(s)-s-2\]
	Multiplicera ekvationen med $\theta(t)$ och hitta Laplacetransformparet:
	\[y''(t)\theta(t)+2y'(t)\theta(t)+5y(t)\theta(t)=e^{-t}\theta(t) \llra
	s^2Y(s)-s-2+2(sY(s)-1)+5Y(s)=\frac{1}{s+1}\]
	\begin{align*}
	&s^2Y(s)-s-2+2(sY(s)-1)+5Y(s)=\frac{1}{s+1} \lra
	(s^2+2s+5)Y(s)=\frac{1}{s+1}+s+4 \lra \\ \lra
	&Y(s)=
	\frac{1+s(s+1)+4(s+1)}{(s+1)(s^2+2s+5)}=
	\frac{s^2+5s+5}{(s+1)((s+1)^2+4)}=
	\frac{k_1}{s+1}+\frac{k_2s+k_3}{(s+1)^2+4}
	\end{align*}
	Identifiera variablerna:
	\begin{align*}
	&k_1((s+1)^2+4)+(k_2s+k_3)(s+1)=s^2+5s+5 \lra \\ \lra
	&k_1(s^2+2s+5)+k_2(s^2+s)+k_3(s+1)=s^2+5s+5 \lra \\ \lra
	&\begin{cases}
	k_1+k_2=1 \\
	2k_1+k_2+k_3=5 \\
	5k_1+k_3=5
	\end{cases} \lra
	\begin{cases}
	k_1=\frac{1}{4} \\
	k_2=\frac{3}{4} \\
	k_3=\frac{15}{4}
	\end{cases}
	\end{align*}
	\begin{align*}
	Y(s)=
	&\frac{1}{4}\left(\frac{1}{s+1}+\frac{3s+15}{(s+1)^2+4}\right)=
	\frac{1}{4}\left(\frac{1}{s+1}+3\frac{s+1}{(s+1)^2+2^2}+6\frac{2}{(s+1)^2+2^2}\right) \llra \\ \llra
	&\frac{1}{4}\left(e^{-t}\theta(t)+3e^{-t}\cos(2t)\theta(t)+6e^{-t}\sin(2t)\theta(t)\right)=
	\frac{e^{-t}}{4}\left(1+3\cos(2t)+6\sin(2t)\right)\theta(t)
	\end{align*}
	\[y(t)\theta(t)=\frac{e^{-t}}{4}\left(1+3\cos(2t)+6\sin(2t)\right)\theta(t)\]
	\[y(t)=\frac{e^{-t}}{4}\left(1+3\cos(2t)+6\sin(2t)\right)\]
	\ans $y(t)=\frac{e^{-t}}{4}\left(1+3\cos(2t)+6\sin(2t)\right)\cond{t \ge 0}$
\end{task}
\pagebreak
\chapter{6}{Faltning}

\begin{task}{6.1}
	Använd definitionen av faltning:
	\begin{align*}
	f*g(t)=
	&\int_{-\infty}^{+\infty}\! f(t-\tau)g(\tau)\, d\tau=
	\int_{-\infty}^{+\infty}\! e^{-(t-\tau)}\theta(t-\tau)e^{-3\tau}\theta(\tau)\, d\tau=
	\left(\int_{0}^{t}\! e^{-(t-\tau)-3\tau}\, d\tau\right)\theta(t)= \\ =
	&\left(\int_{0}^{t}\! e^{-t-2\tau}\, d\tau\right)\theta(t)=
	\left[-\frac{e^{-t-2\tau}}{2}\right]_0^t\theta(t)=
	\left(-\frac{e^{-3t}}{2}-\left(-\frac{e^{-t}}{2}\right)\right)\theta(t)=
	\frac{1}{2}(e^{-t}-e^{-3t})\theta(t)
	\end{align*}
	\ans $f*g(t)=\frac{1}{2}(e^{-t}-e^{-3t})\theta(t)$
\end{task}
\pagebreak
\input{chapters/7.tex}
\pagebreak
\input{chapters/8.tex}
\pagebreak
\input{chapters/9.tex}
\pagebreak
\input{chapters/10.tex}
\pagebreak
\input{chapters/11.tex}

\end{document}