\chapter{2}{Steg och impulsfunktioner}
\begin{task}{2.1 a)}
\end{task}

\begin{task}{b)}
\end{task}

\begin{task}{c)}
\end{task}

\begin{task}{d)}
\end{task}

\begin{task}{e)}
\end{task}

\begin{task}{2.2}
\end{task}

\begin{task}{2.3 a)}
	\[\theta(t-1)\theta(3-t)\]
	eller
	\[\theta(t-1)-\theta(t-3)\]
\end{task}

\begin{task}{b)}
	Funktionen som syns är $-0.5t+1.5$, stegfunktioner som skärmar in $]1,3[$ är (se \taskref{a)}) $\theta(t-1)-\theta(t-3)$ vilket medför:
	
	\ans $(-0.5t+1.5)(\theta(t-1)-\theta(t-3))$
\end{task}

\begin{task}{2.4 a)}
	Funktionen i intervallet $]0,1[$ är $t$. Stegfunktion: $\theta(t)-\theta(t-1)$.
	
	Funktionen i intervallet $]1,2[$ är $1$. Stegfunktion: $\theta(t-1)-\theta(t-2)$.
	
	Funktionen i intervallet $]2,3[$ är $3-t$. Stegfunktion: $\theta(t-2)-\theta(t-3)$ vilket ger:
	
	\ans $t(\theta(t)-\theta(t-1))+\theta(t-1)-\theta(t-2)+(3-t)(\theta(t-2)-\theta(t-3))$
\end{task}

\begin{task}{b)}
	Funktionen i intervallet $]0,1[$ är $t$. Stegfunktion: $\theta(t)-\theta(t-1)$.
	
	Funktionen i intervallet $]1,2[$ är $t-1$. Stegfunktion: $\theta(t-1)-\theta(t-2)$ vilket ger:
	
	\ans $t(\theta(t)-\theta(t-1))+(t-1)(\theta(t-1)-\theta(t-2))$
\end{task}

\pagebreak
\begin{task}{2.5}
	\[p_b(t)=\frac{1}{b}(\theta(t)-\theta(t-b))\]
	Om stegfunktioner finns som en faktor i en integral kan dessa ersätta integrationsgränserna eftersom de evaluerar till noll utanför intervallet.
	\[\int_{-\infty}^{+\infty}\! (\theta(t-a)-\theta(t-b))t \, dt =
	\int_{a}^{b}\! t \, dt\]
	Lös med hjälp av ovanstående samband:
	\begin{align*}
	\int_{-\infty}^{+\infty}\! p_b(t)e^{-st} \, dt =
	&\int_{-\infty}^{+\infty}\!\frac{1}{b}(\theta(t)-\theta(t-b))e^{-st} \, dt =
	\int_{0}^{b}\!\frac{1}{b}e^{-st} \, dt = \\ =
	&\left[-\frac{1}{sb}e^{-st}\right]_{0}^{b}=
	-\frac{e^{-sb}}{sb}-\left(-\frac{1}{sb}\right)=
	\frac{1-e^{-sb}}{sb}\cond{s\neq0}
	\end{align*}
	Om $s=0$:
	\[\int_{-\infty}^{+\infty}\! p_b(t)\*1 \, dt =1 \qquad\text{enligt def., se boken}\]'
	\ans $\int_{-\infty}^{+\infty}\! p_b(t)e^{-st} \, dt =\frac{1}{sb}(1-e^{-sb})\cond{s\neq0}$ och $1\cond{s=0}$
\end{task}

\begin{task}{2.6}
	Räknelag (se boken s. 21):
	\[\int_{-\infty}^{+\infty}\! \delta(t-a)f(t) \, dt =f(a)\cond{\text{om } f \text{ är kontinuerlig i }t=a}\]
	Eftersom $e^{-st}$ är kontinuerlig för alla $t$ använd räknelagen:
	\[\int_{-\infty}^{+\infty}\! \delta(t-a)e^{-st} \, dt =
	e^{-sa}\]
	\ans $e^{-sa}$
\end{task}

\begin{task}{2.7}
	Räknelag (se boken s. 21):
	\[\frac{d}{dt}(\theta(t-a))=\delta(t-a)\]
	Använd räknelagen:
	\[\frac{d}{dt}p_b=
	\frac{1}{b}\frac{d}{dt}(\theta(t)-\theta(t-b))=
	\frac{1}{b}\left(\frac{d}{dt}\theta(t)-\frac{d}{dt}\theta(t-b)\right)=
	\frac{1}{b}\left(\delta(t)-\delta(t-b)\right)\]
	\ans $\frac{1}{b}\left(\delta(t)-\delta(t-b)\right)$
\end{task}

\begin{task}{2.8 a)}
	Räknelag (se boken s. 21):
	\[f(t)\delta(t)=f(0)\delta(t)\cond{\text{om } f \text{ är kontinuerlig i }t=0}\]
	Låt $f(t)=t$, eftersom $t$ är kontinuerlig använd räknelagen:
	\[t\delta(t)=
	f(t)\delta(t)=
	f(0)\delta(t)=
	0\*\delta(t)=
	0\]
	\ans $0$
\end{task}

\begin{task}{b)}
	Räknelag (se boken s. 21):
	\[f(t)\delta(t-a)=f(a)\delta(t-a)\cond{\text{om } f \text{ är kontinuerlig i }t=a}\]
	Låt $f(t)=t$, eftersom $t$ är kontinuerlig använd räknelagen:
	\[t\delta(t-1)=
	f(t)\delta(t-1)=
	f(1)\delta(t-1)=
	1\*\delta(t-1)=
	\delta(t-1)\]
	\ans $\delta(t-1)$
\end{task}

\begin{task}{c)}
	Räknelag (se boken s. 21):
	\[f(t)\delta(t-a)=f(a)\delta(t-a)\cond{\text{om } f \text{ är kontinuerlig i }t=a}\]
	Låt $f(t)=e^{-t}$, eftersom $e^{-t}$ är kontinuerlig använd räknelagen:
	\[e^{-t}\delta(t-2)=
	f(t)\delta(t-2)=
	f(2)\delta(t-2)=
	e^{-2}\delta(t-1)\]
	\ans $e^{-2}\delta(t-1)$
\end{task}

\begin{task}{d)}
	Räknelag (se boken s. 21):
	\[f(t)\delta(t-a)=f(a)\delta(t-a)\cond{\text{om } f \text{ är kontinuerlig i }t=a}\]
	Låt $f(t)=\sin t$, eftersom $\sin t$ är kontinuerlig använd räknelagen:
	\[\sin t\delta(t-\pi)=
	f(t)\delta(t-\pi)=
	f(\pi)\delta(t-\pi)=
	0\*\delta(t-\pi)=
	0\]
	\ans $0$
\end{task}

\begin{task}{2.9}
	Använd sats 2.1 (s. 22):
	\[f(t)= t^2(\theta(t)-\theta(t-1))+(2-t)(\theta(t-1)-\theta(t-2))\]
	Eftersom funktionen saknar språng är ($\frac{d}{dt}t^2=2t$ och $\frac{d}{dt}(2-t)=-1$):
	\[f'(t)=f_p'(t)=2t(\theta(t)-\theta(t-1))-(\theta(t-1)-\theta(t-2))\]
	Eftersom $f'(t)$ har språng i $t=1$ och $t=2$ måste storleken på dessa beräknas (högra funktionen minus den vänstra):
	\[t=1 \ra (-1)-2\*1=-3\]
	\[t=2 \ra 0-(-1)=1\]
	$f'(t)$ är deriverbar i alla punkter utom $t=\{0,1,2\}$, $t=0$ saknar dock språng.
	\[f''(t)=f_p''(t)+b_1\delta(t-a_1)+b_2\delta(t-a_2) \qquad\text{där}\qquad a_1=1,\;b_1=-3,\;a_2=2,\;b_2=1\]
	$\frac{d}{dt}2t=2$ och $\frac{d}{dt}(-1)=0$:
	\begin{align*}
	f''(t)=
	&2(\theta(t)-\theta(t-1))+0\*(\theta(t-1)-\theta(t-2))-3\delta(t-1)+1\*\delta(t-2)= \\ =
	&2(\theta(t)-\theta(t-1))-3\delta(t-1)+\delta(t-2)
	\end{align*}
	\ans 
	\[f'(t)=2t(\theta(t)-\theta(t-1))-(\theta(t-1)-\theta(t-2))\]
	\[f''(t)=2(\theta(t)-\theta(t-1))-3\delta(t-1)+\delta(t-2)\]
\end{task}

\begin{task}{2.10 a)}
	Sinus med amplitud 2 och vinkelfrekvensen 2, samt från 0 till $\pi/2$:
	\[f(t)=2\sin 2t(\theta(t)-\theta(t-\pi/2))\]
\end{task}

\begin{task}{b)}
	Använd sats 2.1 (s. 22):
	\[f(t)= 2\sin 2t(\theta(t)-\theta(t-\pi/2))\]
	Eftersom funktionen saknar språng är ($\frac{d}{dt}2\sin 2t=4\cos2t$):
	\[f'(t)=f_p'(t)=4\cos2t(\theta(t)-\theta(t-\pi/2))\]
	Eftersom $f'(t)$ har språng i $t=0$ och $t=\pi/2$ måste storleken på dessa beräknas (högra funktionen minus den vänstra):
	\[t=0 \ra 4\cos(2\*0)-0=4\]
	\[t=\pi/2 \ra 0-4\cos(2\*\pi/2)=4\]
	$f'(t)$ är deriverbar i alla punkter utom $t=\{0,\pi/2\}$.
	\[f''(t)=f_p''(t)+b_1\delta(t-a_1)+b_2\delta(t-a_2) \qquad\text{där}\qquad a_1=0,\;b_1=4,\;a_2=\pi/2,\;b_2=4\]
	$\frac{d}{dt}4\cos2t=-8\sin2t$:
	\begin{align*}
	f''(t)=
	&-8\sin2t(\theta(t)-\theta(t-\pi/2))+4\delta(t)+4\delta(t-\pi/2)
	\end{align*}
	\ans 
	\[f'(t)=4\cos2t(\theta(t)-\theta(t-\pi/2))\]
	\[f''(t)=-8\sin2t(\theta(t)-\theta(t-\pi/2))+4\delta(t)+4\delta(t-\pi/2)\]
\end{task}

\begin{task}{2.11}
	Beskriv $\abs{x}$ med hjälp av stegfunktioner:
	\[f(x) = \abs{x} = -x(1-\theta(x))+x\theta(x)\]
	Eftersom funktionen saknar språng är ($\frac{d}{dx}x=1$):
	\[f'(x)=f_p'(x) = -1\*(1-\theta(x))+1\*\theta(x) = -1+\theta(x)+\theta(x) = 2\theta(x)-1\]
	Eftersom $f'(t)$ har språng i $t=0$ måste storleken på denna beräknas (högra funktionen minus den vänstra):
	\[x=0 \ra (2-1)-(-1)=2\]
	$f'(t)$ är deriverbar i alla punkter utom $t=0$.
	\[f''(x)=f_p''(x)+b\delta(x-a)\qquad\text{där}\qquad a=0,\;b=2\]
	$\frac{d}{dt}1=0$:
	\[f''(x) = -0\*(1-\theta(x))+0\*\theta(x)+2\delta(x-0)=2\delta(x)\]
	\ans 
	\[f'(t)=2\theta(x)-1\]
	\[f''(t)=2\delta(x)\]
\end{task}

\begin{task}{2.12}
	Använd sambandet på s. 17:
	\begin{align*}
	v(t)=
	&\int_{-\infty}^{t} \! \tau^a\theta(\tau) \, d\tau =
	\begin{linsys}{ll}
		0,& t\le0 \\
		\displaystyle\int_0^t\! \tau^a \, d\tau,\quad & t>0
	\end{linsys}=
	\left(\int_0^t\! \tau^a \, d\tau\right)\theta(t)= \\ =
	&\left(\left[\frac{\tau^{a+1}}{a+1}\right]_0^t\right)\theta(t)=
	\left(\frac{t^{a+1}}{a+1}-\frac{0^{a+1}}{a+1}\right)\theta(t)=
	\frac{t^{a+1}}{a+1}\theta(t)\cond{a>-1}
	\end{align*}
	\ans $v(t)=\frac{t^{a+1}}{a+1}\theta(t)\cond{a>-1}$
\end{task}

\begin{task}{2.13 a)}
	Använd sambandet på s. 17:
	\begin{align*}
	&\int_{-\infty}^{t} \! e^{-\tau}\theta(\tau) \, d\tau =
	\left(\int_0^t\! e^{-\tau} \, d\tau\right)\theta(t)= \\ =
	&\left(\left[-e^{-\tau}\right]_0^t\right)\theta(t)=
	\left(-e^{-t}-\left(-e^{-0}\right)\right)\theta(t)=
	\left(1-e^{-t}\right)\theta(t)
	\end{align*}
	\ans $\left(1-e^{-t}\right)\theta(t)$
\end{task}

\begin{task}{b)}
	Använd sambandet på s. 17:
	\begin{align*}
	&\int_{-\infty}^{t} \! e^{-\tau}\theta(\tau-1) \, d\tau =
	\left(\int_{-1}^t\! e^{-\tau} \, d\tau\right)\theta(t-1)= \\ =
	&\left(\left[-e^{-\tau}\right]_{-1}^t\right)\theta(t-1)=
	\left(-e^{-t}-\left(-e^{-1}\right)\right)\theta(t-1)=
	\left(e^{-1}-e^{-t}\right)\theta(t-1)
	\end{align*}
	\ans $\left(e^{-1}-e^{-t}\right)\theta(t-1)$
\end{task}

\begin{task}{c)}
	Använd sambandet på s. 17:
	\begin{align*}
	&\int_{-\infty}^{t} \! e^{\tau}(1-\theta(\tau)) \, d\tau =
	\int_{-\infty}^{t} \! e^{\tau} \, d\tau - \int_{-\infty}^{t} \! e^{\tau}\theta(\tau) \, d\tau=
	\left[e^{\tau}\right]_{-\infty}^t - \left(\int_{0}^t\! e^{\tau} \, d\tau\right)\theta(t)= \\ =
	&e^t-\overbrace{e^{-\infty}}^0-\left(\left[e^{\tau}\right]_{0}^t\right)\theta(t)=
	e^t-\left(e^{t}-e^{0}\right)\theta(t)=
	e^t-\left(e^{t}-1\right)\theta(t)= \\ =
	&e^t-e^{t}\theta(t)+\theta(t)=
	e^t(1-\theta(t))+\theta(t)
	\end{align*}
	\ans $e^t(1-\theta(t))+\theta(t)$
\end{task}

\begin{task}{d)}
	Använd sambandet på s. 17:
	\begin{align*}
	&\int_{-\infty}^{t} \! e^{\tau}\theta(1-\tau) \, d\tau =
	\int_{-\infty}^{t} \! e^{\tau}(1-\theta(\tau-1)) \, d\tau =
	\int_{-\infty}^{t} \! e^{\tau} \, d\tau - \int_{-\infty}^{t} \! e^{\tau}\theta(\tau-1) \, d\tau= \\ =
	&\left[e^{\tau}\right]_{-\infty}^t - \left(\int_{1}^t\! e^{\tau} \, d\tau\right)\theta(t-1)=
	e^t-\overbrace{e^{-\infty}}^0-\left(\left[e^{\tau}\right]_{1}^t\right)\theta(t-1)= \\ =
	&e^t-\left(e^{t}-e^{1}\right)\theta(t-1)=
	e^t-e^{t}\theta(t-1)+e\theta(t-1)= \\ =
	&e^t(1-\theta(t-1))+e\theta(t-1)=
	e^t\theta(1-t)+e\theta(t-1)
	\end{align*}
	\ans $e^t\theta(1-t)+e\theta(t-1)$
\end{task}

\begin{task}{2.14 a)}
	Låt $f(t)=e^{2t}\theta(t)$:
	\begin{align*}
	F(t)=
	&\int \! e^{2t}\theta(t) \, dt=
	\theta(t)\int_{0}^{t} \! e^{2t} \, dt+C=
	\theta(t)\left[\frac{1}{2}e^{2t}\right]_{0}^{t}+C= \\ =
	&\frac{1}{2}(e^{2t}-e^{2\*0})\theta(t)+C=
	\frac{1}{2}(e^{2t}-1)\theta(t)+C
	\end{align*}
	\ans $F(t)=\frac{1}{2}(e^{2t}-1)\theta(t)+C$
\end{task}

\begin{task}{b)}
	Låt $f(t)=(t-1)\theta(t)$:
	\begin{align*}
	F(t)=
	&\int \! (t-1)\theta(t) \, dt=
	\theta(t)\int_{0}^{t} \! t-1 \, dt+C=
	\theta(t)\left[\frac{t^2}{2}-t\right]_{0}^{t}+C= \\ =
	&\left(\frac{t^2}{2}-t-\left(\frac{0^2}{2}-0\right)\right)\theta(t)+C=
	\left(\frac{t^2}{2}-t\right)\theta(t)+C
	\end{align*}
	\ans $F(t)=(\frac{t^2}{2}-t)\theta(t)+C$
\end{task}

\begin{task}{c)}
	Låt $f(t)=(t-1)\theta(t-1)$:
	\begin{align*}
	F(t)=
	&\int \! (t-1)\theta(t-1) \, dt=
	\theta(t-1)\int_{1}^{t} \! t-1 \, dt+C=
	\theta(t-1)\left[\frac{t^2}{2}-t\right]_{1}^{t}+C= \\ =
	&\left(\frac{t^2}{2}-t-\left(\frac{1^2}{2}-1\right)\right)\theta(t-1)+C=
	\left(\frac{t^2}{2}-t+\frac{1}{2}\right)\theta(t-1)+C= \\ =
	&\frac{t^2-2t+1}{2}\theta(t-1)+C=
	\frac{(t-1)^2}{2}\theta(t-1)+C
	\end{align*}
	\ans $F(t)=\frac{(t-1)^2}{2}\theta(t-1)+C$
\end{task}

\begin{task}{d)}
	Låt $f(t)=t\theta(t-3)$:
	\begin{align*}
	F(t)=
	&\int \! t\theta(t-3) \, dt=
	\theta(t-3)\int_{3}^{t} \! t \, dt+C=
	\theta(t-3)\left[\frac{t^2}{2}\right]_{3}^{t}+C= \\ =
	&\left(\frac{t^2}{2}-\frac{3^2}{2}\right)\theta(t-3)+C=
	\frac{1}{2}\left(t^2-9\right)\theta(t-3)+C
	\end{align*}
	\ans $F(t)=\frac{1}{2}\left(t^2-9\right)\theta(t-3)+C$
\end{task}