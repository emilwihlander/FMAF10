\chapter{2}{Steg och impulsfunktioner}
\begin{task}{2.1 a)}
\end{task}

\begin{task}{b)}
\end{task}

\begin{task}{c)}
\end{task}

\begin{task}{d)}
\end{task}

\begin{task}{e)}
\end{task}

\begin{task}{2.2}
\end{task}

\begin{task}{2.3 a)}
	\[\theta(t-1)\theta(3-t)\]
	eller
	\[\theta(t-1)-\theta(t-3)\]
\end{task}

\begin{task}{b)}
	Funktionen som syns är $-0.5t+1.5$, stegfunktioner som skärmar in $]1,3[$ är (se \taskref{a)}) $\theta(t-1)-\theta(t-3)$ vilket medför:
	
	\ans $(-0.5t+1.5)(\theta(t-1)-\theta(t-3))$
\end{task}

\begin{task}{2.4 a)}
	Funktionen i intervallet $]0,1[$ är $t$. Stegfunktion: $\theta(t)-\theta(t-1)$.
	
	Funktionen i intervallet $]1,2[$ är $1$. Stegfunktion: $\theta(t-1)-\theta(t-2)$.
	
	Funktionen i intervallet $]2,3[$ är $3-t$. Stegfunktion: $\theta(t-2)-\theta(t-3)$ vilket ger:
	
	\ans $t(\theta(t)-\theta(t-1))+\theta(t-1)-\theta(t-2)+(3-t)(\theta(t-2)-\theta(t-3))$
\end{task}

\begin{task}{b)}
	Funktionen i intervallet $]0,1[$ är $t$. Stegfunktion: $\theta(t)-\theta(t-1)$.
	
	Funktionen i intervallet $]1,2[$ är $t-1$. Stegfunktion: $\theta(t-1)-\theta(t-2)$ vilket ger:
	
	\ans $t(\theta(t)-\theta(t-1))+(t-1)(\theta(t-1)-\theta(t-2))$
\end{task}

\pagebreak
\begin{task}{2.5}
	\[p_b(t)=\frac{1}{b}(\theta(t)-\theta(t-b))\]
	Om stegfunktioner finns som en faktor i en integral kan dessa ersätta integrationsgränserna eftersom de evaluerar till noll utanför intervallet.
	\[\int_{-\infty}^{+\infty}\! (\theta(t-a)-\theta(t-b))t \, dt =
	\int_{a}^{b}\! t \, dt\]
	Lös med hjälp av ovanstående samband:
	\begin{align*}
	\int_{-\infty}^{+\infty}\! p_b(t)e^{-st} \, dt =
	&\int_{-\infty}^{+\infty}\!\frac{1}{b}(\theta(t)-\theta(t-b))e^{-st} \, dt =
	\int_{0}^{b}\!\frac{1}{b}e^{-st} \, dt = \\ =
	&\left[-\frac{1}{sb}e^{-st}\right]_{0}^{b}=
	-\frac{e^{-sb}}{sb}-\left(-\frac{1}{sb}\right)=
	\frac{1-e^{-sb}}{sb}\cond{s\neq0}
	\end{align*}
	Om $s=0$:
	\[\int_{-\infty}^{+\infty}\! p_b(t)\*1 \, dt =1 \qquad\text{enligt def., se boken}\]'
	\ans $\int_{-\infty}^{+\infty}\! p_b(t)e^{-st} \, dt =\frac{1}{sb}(1-e^{-sb})\cond{s\neq0}$ och $1\cond{s=0}$
\end{task}

\begin{task}{2.6}
	Räknelag (se boken s. 22):
	\[\int_{-\infty}^{+\infty}\! \delta(t-a)f(t) \, dt =f(a)\cond{\text{om } f \text{ är kontinuerlig i }t=a}\]
	Eftersom $e^{-st}$ är kontinuerlig för alla $t$ använd räknelagen:
	\[\int_{-\infty}^{+\infty}\! \delta(t-a)e^{-st} \, dt =
	e^{-sa}\]
	\ans $e^{-sa}$
\end{task}

\begin{task}{2.7}
	Räknelag (se boken s. 22):
	\[\frac{d}{dt}(\theta(t-a))=\delta(t-a)\]
	Använd räknelagen:
	\[\frac{d}{dt}p_b=
	\frac{1}{b}\frac{d}{dt}(\theta(t)-\theta(t-b))=
	\frac{1}{b}\left(\frac{d}{dt}\theta(t)-\frac{d}{dt}\theta(t-b)\right)=
	\frac{1}{b}\left(\delta(t)-\delta(t-b)\right)\]
	\ans $\frac{1}{b}\left(\delta(t)-\delta(t-b)\right)$
\end{task}