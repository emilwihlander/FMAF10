\chapter{3}{Laplacetransformationer}

\begin{task}{3.1 a)}
	$f(t)=e^{-2t}\theta(t)$
	
	Se definitionen av Lapacetransformen i boken.
	\begin{align*}
	\laplace f(s)=
	&\int_{-\infty}^{+\infty}\! e^{-st}f(t) \, dt=
	\int_{-\infty}^{+\infty}\! e^{-st}e^{-2t}\theta(t) \, dt=
	\int_{0}^{+\infty}\! e^{-(s+2)t} \, dt=
	\left[-\frac{e^{-(s+2)t}}{s+2}\right]_0^{+\infty}= \\ =
	&\lim\limits_{T\rightarrow\infty}\frac{1}{s+2}(1-e^{-(s+2)T})
	\end{align*}
	Om $s>-2$ gäller att $e^{-(s+2)T} \rightarrow 0$ när $T \rightarrow \infty$ vilket medför:
	\[\laplace f(s)=\frac{1}{s+2}\cond{s>-2}\]
	Om $s=-2$:
	\[\laplace f(s)=
	\int_{0}^{+\infty}\! 1 \, dt=
	\left[t\right]_0^{+\infty}\rightarrow\infty\]
	Om $s<-2$ gäller att $e^{-(s+2)T} \rightarrow \infty$ när $T \rightarrow \infty$ vilket medför:
	\[\laplace f(s)\rightarrow-\infty\]
	Detta medför att $\laplace f(s)$ endast är konvergent när $s>-2$ och därmed är Laplacetransformen för $e^{-2t}\theta(t)$ endast definierad i det intervallet.
	
	Låt nu $s$ vara ett komplext tal, $s=a+bi$:
	\[\abs{e^{-(a+bi+2)t}}=
	\abs{e^{-(a+2)t}}\underbrace{\abs{e^{-ibt}}}_{=1}=
	e^{-(a+2)t}\]
	Här ser vi att $e^{-(s+2)t}\rightarrow 0$ då $t\rightarrow \infty$ om $a = \Re s > -2$ vilket utvidgar Laplacetransformen att inkludera hela planet $\Re s > -2$.
	
	\ans $\laplace f(s)= \dfrac{1}{s+2}\cond{\Re s > -2}$
\end{task}

\begin{task}{3.1 b)}
	$f(t)=\theta(t)-\theta(t-1)$
	
	Se definitionen av Lapacetransformen i boken.
	\begin{align*}
	\laplace f(s)=
	&\int_{-\infty}^{+\infty}\! e^{-st}f(t) \, dt=
	\int_{-\infty}^{+\infty}\! e^{-st}(\theta(t)-\theta(t-1)) \, dt=
	\int_{0}^{1}\! e^{-st} \, dt=
	\left[-\frac{e^{-st}}{s}\right]_0^{1}= \\ =
	&-\frac{e^{-s}}{s}+\frac{1}{s}=
	\frac{1-e^{-s}}{s}\cond{s\neq0}
	\end{align*}
	Om $s=0$ gäller att $e^{-st} = 1$ vilket medför:
	\[\laplace f(0)=
	\int_{0}^{1}\! 1 \, dt=
	\left[t\right]_0^1=
	1-0=1\]
	
	\ans $\laplace f(s)= \dfrac{1}{s}(1-e^{-s})\cond{s\neq0}$ och $\laplace f(0)= 1$
\end{task}