\chapter{3}{Laplacetransformationer}

\begin{task}{3.1 a)}
	$f(t)=e^{-2t}\theta(t)$
	
	Se definitionen av Lapacetransformen i boken.
	\begin{align*}
	\laplace f(s)=
	&\int_{-\infty}^{+\infty}\! e^{-st}f(t) \, dt=
	\int_{-\infty}^{+\infty}\! e^{-st}e^{-2t}\theta(t) \, dt=
	\int_{0}^{+\infty}\! e^{-(s+2)t} \, dt=
	\left[-\frac{e^{-(s+2)t}}{s+2}\right]_0^{+\infty}= \\ =
	&\lim\limits_{T\rightarrow\infty}\frac{1}{s+2}(1-e^{-(s+2)T})
	\end{align*}
	Om $s>-2$ gäller att $e^{-(s+2)T} \rightarrow 0$ när $T \rightarrow \infty$ vilket medför:
	\[\laplace f(s)=\frac{1}{s+2}\cond{s>-2}\]
	Om $s=-2$:
	\[\laplace f(s)=
	\int_{0}^{+\infty}\! 1 \, dt=
	\left[t\right]_0^{+\infty}\rightarrow\infty\]
	Om $s<-2$ gäller att $e^{-(s+2)T} \rightarrow \infty$ när $T \rightarrow \infty$ vilket medför:
	\[\laplace f(s)\rightarrow-\infty\]
	Detta medför att $\laplace f(s)$ endast är konvergent när $s>-2$ och därmed är Laplacetransformen för $e^{-2t}\theta(t)$ endast definierad i det intervallet.
	
	Låt nu $s$ vara ett komplext tal, $s=a+bi$:
	\[\abs{e^{-(a+bi+2)t}}=
	\abs{e^{-(a+2)t}}\underbrace{\abs{e^{-ibt}}}_{=1}=
	e^{-(a+2)t}\]
	Här ser vi att $e^{-(s+2)t}\rightarrow 0$ då $t\rightarrow \infty$ om $a = \Re s > -2$ vilket utvidgar Laplacetransformen att inkludera hela planet $\Re s > -2$.
	
	\ans $\laplace f(s)= \dfrac{1}{s+2}\cond{\Re s > -2}$
\end{task}

\begin{task}{3.1 b)}
	$f(t)=\theta(t)-\theta(t-1)$
	
	Se definitionen av Lapacetransformen i boken.
	\begin{align*}
	\laplace f(s)=
	&\int_{-\infty}^{+\infty}\! e^{-st}f(t) \, dt=
	\int_{-\infty}^{+\infty}\! e^{-st}(\theta(t)-\theta(t-1)) \, dt=
	\int_{0}^{1}\! e^{-st} \, dt=
	\left[-\frac{e^{-st}}{s}\right]_0^{1}= \\ =
	&-\frac{e^{-s}}{s}+\frac{1}{s}=
	\frac{1-e^{-s}}{s}\cond{s\neq0}
	\end{align*}
	Om $s=0$ gäller att $e^{-st} = 1$ vilket medför:
	\[\laplace f(0)=
	\int_{0}^{1}\! 1 \, dt=
	\left[t\right]_0^1=
	1-0=1\]
	
	\ans $\laplace f(s)= \dfrac{1}{s}(1-e^{-s})\cond{s\neq0}$ och $\laplace f(0)= 1$
\end{task}

\begin{task}{3.2}
	\begin{align*}
	f(at)\longleftrightarrow
	&\int_{-\infty}^{+\infty}\! e^{-st}f(at) \, dt=
	\left[\begin{array}{l}
	x=at \\
	dt=dx/a \\
	t=-\infty\leftrightarrow x=-\infty \\
	t=+\infty\leftrightarrow x=+\infty
	\end{array}\right] =
	\int_{-\infty}^{+\infty}\! e^{-sx/a}f(x) \, \frac{dx}{a}= \\ =
	&\frac{1}{a}\int_{-\infty}^{+\infty}\! e^{-(s/a)x}f(x) \, dx=
	\frac{1}{a}\laplace f\left(\frac{s}{a}\right)\mproof
	\end{align*}
\end{task}

\begin{task}{3.3 a)}
	Låt $s=\sigma+i\omega$.
	
	Använd $\theta(t)\longleftrightarrow \dfrac{1}{s}\cond{\sigma > 0}$ och $t^n\theta(t)\longleftrightarrow\dfrac{n!}{s^{n+1}}\cond{\sigma > 0}$.
	\begin{align*}
	f(t)=
	(2+3t^2)\theta(t)=
	2\theta(t)+3t^2\theta(t)\longleftrightarrow
	2\frac{1}{s}+3\frac{2}{s^3}=
	\frac{2s^2+6}{s^3}\cond{\sigma > 0}
	\end{align*}
	\ans $(2+3t^2)\theta(t)\longleftrightarrow\dfrac{2s^2+6}{s^3}\cond{\sigma > 0}$
\end{task}

\begin{task}{3.3 b)}
	Låt $s=\sigma+i\omega$.
	
	Använd $t^ne^{at}\theta(t)\longleftrightarrow \dfrac{n!}{(s-a)^{n+1}}\cond{\sigma > \Re a}$.
	\begin{align*}
	f(t)=
	e^{3t}\theta(t)=
	t^0e^{3t}\theta(t)\longleftrightarrow
	\frac{0!}{(s-3)^{1}}=
	\frac{1}{s-3}\cond{\sigma > 3}
	\end{align*}
	\ans $e^{3t}\theta(t)\longleftrightarrow\dfrac{1}{s-3}\cond{\sigma > 3}$
\end{task}

\begin{task}{3.3 c)}
	Låt $s=\sigma+i\omega$.
	
	Använd $t^ne^{at}\theta(t)\longleftrightarrow \dfrac{n!}{(s-a)^{n+1}}\cond{\sigma > \Re a}$.
	\begin{align*}
	f(t)=
	te^{3t}\theta(t)=
	t^1e^{3t}\theta(t)\longleftrightarrow
	\frac{1!}{(s-3)^{1+1}}=
	\frac{1}{(s-3)^2}\cond{\sigma > 3}
	\end{align*}
	\ans $te^{3t}\theta(t)\longleftrightarrow\dfrac{1}{(s-3)^2}\cond{\sigma > 3}$
\end{task}

\begin{task}{3.3 d)}
	Låt $s=\sigma+i\omega$.
	
	Använd $t^ne^{at}\theta(t)\longleftrightarrow \dfrac{n!}{(s-a)^{n+1}}\cond{\sigma > \Re a}$.
	\begin{align*}
	f(t)=
	t^2e^{3t}\theta(t)\longleftrightarrow
	\frac{2!}{(s-3)^{2+1}}=
	\frac{2}{(s-3)^3}\cond{\sigma > 3}
	\end{align*}
	\ans $t^2e^{3t}\theta(t)\longleftrightarrow\dfrac{2}{(s-3)^3}\cond{\sigma > 3}$
\end{task}