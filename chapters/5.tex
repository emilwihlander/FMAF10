\chapter{5}{Lösning av differentialekvationer genom Laplacetransformation}

\begin{task}{5.1}
	Låt $Y(s)=\laplace(\theta y)(s)$ vilket medför att:
	\[\laplace(\theta y')(s)=s\laplace(\theta y)(s)-y(0)=sY(s)-1\]
	\[\laplace(\theta y'')(s)=s\laplace(\theta y')(s)-y'(0)=s(sY(s)-1)-2=s^2Y(s)-s-2\]
	Multiplicera ekvationen med $\theta(t)$ och hitta Laplacetransformparet:
	\[y''(t)\theta(t)+2y'(t)\theta(t)+5y(t)\theta(t)=e^{-t}\theta(t) \llra
	s^2Y(s)-s-2+2(sY(s)-1)+5Y(s)=\frac{1}{s+1}\]
	\begin{align*}
	&s^2Y(s)-s-2+2(sY(s)-1)+5Y(s)=\frac{1}{s+1} \lra
	(s^2+2s+5)Y(s)=\frac{1}{s+1}+s+4 \lra \\ \lra
	&Y(s)=
	\frac{1+s(s+1)+4(s+1)}{(s+1)(s^2+2s+5)}=
	\frac{s^2+5s+5}{(s+1)((s+1)^2+4)}=
	\frac{k_1}{s+1}+\frac{k_2s+k_3}{(s+1)^2+4}
	\end{align*}
	Identifiera variablerna:
	\begin{align*}
	&k_1((s+1)^2+4)+(k_2s+k_3)(s+1)=s^2+5s+5 \lra \\ \lra
	&k_1(s^2+2s+5)+k_2(s^2+s)+k_3(s+1)=s^2+5s+5 \lra \\ \lra
	&\begin{cases}
	k_1+k_2=1 \\
	2k_1+k_2+k_3=5 \\
	5k_1+k_3=5
	\end{cases} \lra
	\begin{cases}
	k_1=\frac{1}{4} \\
	k_2=\frac{3}{4} \\
	k_3=\frac{15}{4}
	\end{cases}
	\end{align*}
	\begin{align*}
	Y(s)=
	&\frac{1}{4}\left(\frac{1}{s+1}+\frac{3s+15}{(s+1)^2+4}\right)=
	\frac{1}{4}\left(\frac{1}{s+1}+3\frac{s+1}{(s+1)^2+2^2}+6\frac{2}{(s+1)^2+2^2}\right) \llra \\ \llra
	&\frac{1}{4}\left(e^{-t}\theta(t)+3e^{-t}\cos(2t)\theta(t)+6e^{-t}\sin(2t)\theta(t)\right)=
	\frac{e^{-t}}{4}\left(1+3\cos(2t)+6\sin(2t)\right)\theta(t)
	\end{align*}
	\[y(t)\theta(t)=\frac{e^{-t}}{4}\left(1+3\cos(2t)+6\sin(2t)\right)\theta(t)\]
	\[y(t)=\frac{e^{-t}}{4}\left(1+3\cos(2t)+6\sin(2t)\right)\]
	Eftersom $y(0)=\frac{e^0}{4}(1+3\cos(0)+6\sin(0))=1$ är funktionen, utöver $t>0$, definierad för $t=0$.
	
	\ans $y(t)=\frac{e^{-t}}{4}\left(1+3\cos(2t)+6\sin(2t)\right)\cond{t \ge 0}$
\end{task}

\begin{task}{5.2}
	Låt $Y(s)=\laplace(\theta y)(s)$ vilket medför att:
	\[\laplace(\theta y')(s)=s\laplace(\theta y)(s)-y(0)=sY(s)\]
	\[\laplace(\theta y'')(s)=s\laplace(\theta y')(s)-y'(0)=s^2Y(s)-1\]
	Multiplicera vänsterledet med $\theta(t)$ och hitta Laplacetransformparet:
	\[y''(t)\theta(t)-2y'(t)\theta(t)+2y(t)\theta(t) \llra
	s^2Y(s)-s-2+2(sY(s)-1)+5Y(s)=\frac{1}{s+1}\]
	\begin{align*}
	&s^2Y(s)-1-2sY(s)+2Y(s)=0 \lra
	(s^2-2s+2)Y(s)=1 \lra \\ \lra
	&Y(s)=
	\frac{1}{s^2-2s+2}=
	\frac{1}{(s-1)^2+1} \llra
	e^{t}\sin(t)\theta(t)
	\end{align*}
	\[y(t)\theta(t)=e^{t}\sin(t)\theta(t)\]
	\[y(t)=e^{t}\sin(t)\]
	Eftersom $y(0)=e^{0}\sin(0)=0$ är funktionen, utöver $t>0$, definierad för $t=0$.
	
	\ans $y(t)=e^{t}\sin(t)\cond{t \ge 0}$
\end{task}

\begin{task}{5.3}
	Låt $Y(s)=\laplace(\theta y)(s)$ vilket medför att:
	\[\laplace(\theta y')(s)=s\laplace(\theta y)(s)-y(0)=sY(s)\]
	\[\laplace(\theta y'')(s)=s\laplace(\theta y')(s)-y'(0)=s^2Y(s)\]
	Multiplicera ekvationen med $\theta(t)$ och hitta Laplacetransformparet:
	\[y''(t)\theta(t)+2y'(t)\theta(t)+y(t)\theta(t)=e^{-2t}\theta(t) \llra
	s^2Y(s)+2sY(s)+Y(s)=\frac{1}{s+2}\]
	\begin{align*}
	&s^2Y(s)+2sY(s)+Y(s)=\frac{1}{s+2} \lra
	(s^2+2s+1)Y(s)=\frac{1}{s+2} \lra \\ \lra
	&Y(s)=
	\frac{1}{(s+2)(s^2+2s+1)}=
	\frac{1}{(s+2)(s+1)^2}=
	\frac{k_1}{s+2}+\frac{k_2}{s+1}+\frac{k_3}{(s+1)^2}
	\end{align*}
	Identifiera variablerna:
	\begin{align*}
	&k_1(s+1)^2+k_2(s+2)(s+1)+k_3(s+2)=1 \lra \\ \lra
	&k_1(s^2+2s+1)+k_2(s^2+3s+2)+k_3(s+2)=1 \lra \\ \lra
	&\begin{cases}
	k_1+k_2=0 \\
	2k_1+3k_2+k_3=0 \\
	k_1+2k_2+2k_3=1
	\end{cases} \lra
	\begin{cases}
	k_1=1 \\
	k_2=-1 \\
	k_3=1
	\end{cases}
	\end{align*}
	\begin{align*}
	Y(s)=
	\frac{1}{s+2}-\frac{1}{s+1}+\frac{1}{(s+1)^2}\llra
	&e^{-2t}\theta(t)-e^{-t}\theta(t)+e^{-t}t\theta(t)= \\ =
	&(e^{-2t}+e^{-t}(t-1))\theta(t)
	\end{align*}
	\[y(t)\theta(t)=(e^{-2t}+e^{-t}(t-1))\theta(t)\]
	\[y(t)=e^{-2t}+e^{-t}(t-1)\]
	Eftersom $y(0)=e^{0}+e^{0}(0-1)=0$ är funktionen, utöver $t>0$, definierad för $t=0$.
	
	\ans $y(t)=e^{-2t}+e^{-t}(t-1)\cond{t \ge 0}$
\end{task}

\pagebreak
\begin{task}{5.4}
	Låt $Y_1(s)=\laplace(\theta y_1)(s)$ och $Y_2(s)=\laplace(\theta y_2)(s)$ vilket medför att:
	\[\laplace(\theta y_1')(s)=s\laplace(\theta y_1)(s)-y_1(0)=sY_1(s)-1\]
	\[\laplace(\theta y_2')(s)=s\laplace(\theta y_2)(s)-y_2(0)=sY_2(s)\]
	Multiplicera ekvationssystemet med $\theta(t)$:
	\[
	\begin{cases}
	y_1\theta(t)-2y_2'\theta=2\theta \\
	y_1'\theta(t)+2y_2=-2t\theta
	\end{cases}\]
	Laplacetransformera båda ekvationerna:
	\[
	\begin{cases}
	Y_1(s)-2sY_2(s)=\frac{2}{s} \\
	sY_1(s)-1+2Y_2(s)=-\frac{2}{s^2}
	\end{cases} \lra
	\begin{linsys}{rrr}
	 Y_1(s)-&2sY_2(s)=&\frac{2}{s} \\
	sY_1(s)+& 2Y_2(s)=&\frac{s^2-2}{s^2}
	\end{linsys}\]
	Använd Cramers regel:
	\begin{align*}
	\Delta(s)=\begin{detmat}
	1 & -2s \\
	s & 2
	\end{detmat}=
	2-(-2s)s=2(s^2+1)
	\end{align*}
	\begin{align*}
	Y_1(s)=
	\frac{1}{\Delta(s)}\begin{detmat}
	\frac{2}{s} & -2s \\
	\frac{s^2-2}{s^2} & 2
	\end{detmat}=
	\frac{1}{2(s^2+1)}\left(\frac{4}{s}-\frac{(-2s)(s^2-2)}{s^2}\right)=
	\frac{s}{s^2+1}
	\end{align*}
	\begin{align*}
	Y_2(s)=
	\frac{1}{\Delta(s)}\begin{detmat}
	1 &\frac{2}{s} \\
	s &\frac{s^2-2}{s^2}
	\end{detmat}=
	\frac{1}{2(s^2+1)}\left(\frac{s^2-2}{s^2}-2\right)=
	\frac{1}{2}\*\frac{1}{s^2+1}-\frac{1}{s^2}
	\end{align*}
	\[\begin{cases}
	Y_1(s)=\frac{s}{s^2+1} \\
	Y_2(s)=\frac{1}{2}\*\frac{1}{s^2+1}-\frac{1}{s^2}
	\end{cases}\llra
	\begin{cases}
	y_1\theta=\cos(t)\theta(t) \\
	y_2\theta=\frac{1}{2}\sin(t)\theta(t)-t\theta(t)
	\end{cases}\]
	Eftersom $\theta(t)=1$ endast när $t>0$ måste det läggas till som villkor:
	\[\begin{cases}
	y_1=\cos(t) \\
	y_2=\frac{1}{2}\sin(t)-t
	\end{cases}\cond{t > 0}\]
	Eftersom $y_1(0)=\cos(0)=1$ och $y_2(0)=\frac{1}{2}\sin(0)-0=0$ är de de också definierade för $t=0$.
	
	\ans 
	$\begin{cases}
	y_1=\cos(t) \\
	y_2=\frac{1}{2}\sin(t)-t
	\end{cases}\cond{t \ge 0}$
\end{task}

\begin{task}{5.5}
	Eftersom differentialekvationen saknar begynnelsevillkor låt $Y(s)=\laplace y(s)$ 
	\[sY(s)=\laplace y'(s),\qquad s^2Y(s)=\laplace y''(s)\]
	Laplacetransformera ekvationen:
	\begin{align*}
	&s^2Y(s)+2sY(s)+2Y(s)=1 \lra
	(s^2+2s+2)Y(s)=1\lra \\ \lra
	&Y(s)=\frac{1}{s^2+2s+2}=
	\frac{1}{(s+1)^2+1}\llra
	e^{-t}\sin(t)\theta(t)
	\end{align*}
	Eftersom $\theta(t)$ är en faktor är lösningen kausal.
	
	\ans $y(t)=e^{-t}\sin(t)\theta(t)$
\end{task}
