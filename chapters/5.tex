\chapter{5}{Lösning av differentialekvationer genom Laplacetransformation}

\begin{task}{5.1}
	Låt $Y(s)=\laplace(\theta y)(s)$ vilket medför att:
	\[\laplace(\theta y')(s)=s\laplace(\theta y)(s)-y(0)=sY(s)-1\]
	\[\laplace(\theta y'')(s)=s\laplace(\theta y')(s)-y'(0)=s(sY(s)-1)-2=s^2Y(s)-s-2\]
	Multiplicera ekvationen med $\theta(t)$ och hitta Laplacetransformparet:
	\[y''(t)\theta(t)+2y'(t)\theta(t)+5y(t)\theta(t)=e^{-t}\theta(t) \llra
	s^2Y(s)-s-2+2(sY(s)-1)+5Y(s)=\frac{1}{s+1}\]
	\begin{align*}
	&s^2Y(s)-s-2+2(sY(s)-1)+5Y(s)=\frac{1}{s+1} \lra
	(s^2+2s+5)Y(s)=\frac{1}{s+1}+s+4 \lra \\ \lra
	&Y(s)=
	\frac{1+s(s+1)+4(s+1)}{(s+1)(s^2+2s+5)}=
	\frac{s^2+5s+5}{(s+1)((s+1)^2+4)}=
	\frac{k_1}{s+1}+\frac{k_2s+k_3}{(s+1)^2+4}
	\end{align*}
	Identifiera variablerna:
	\begin{align*}
	&k_1((s+1)^2+4)+(k_2s+k_3)(s+1)=s^2+5s+5 \lra \\ \lra
	&k_1(s^2+2s+5)+k_2(s^2+s)+k_3(s+1)=s^2+5s+5 \lra \\ \lra
	&\begin{cases}
	k_1+k_2=1 \\
	2k_1+k_2+k_3=5 \\
	5k_1+k_3=5
	\end{cases} \lra
	\begin{cases}
	k_1=\frac{1}{4} \\
	k_2=\frac{3}{4} \\
	k_3=\frac{15}{4}
	\end{cases}
	\end{align*}
	\begin{align*}
	Y(s)=
	&\frac{1}{4}\left(\frac{1}{s+1}+\frac{3s+15}{(s+1)^2+4}\right)=
	\frac{1}{4}\left(\frac{1}{s+1}+3\frac{s+1}{(s+1)^2+2^2}+6\frac{2}{(s+1)^2+2^2}\right) \llra \\ \llra
	&\frac{1}{4}\left(e^{-t}\theta(t)+3e^{-t}\cos(2t)\theta(t)+6e^{-t}\sin(2t)\theta(t)\right)=
	\frac{e^{-t}}{4}\left(1+3\cos(2t)+6\sin(2t)\right)\theta(t)
	\end{align*}
	\[y(t)\theta(t)=\frac{e^{-t}}{4}\left(1+3\cos(2t)+6\sin(2t)\right)\theta(t)\]
	\[y(t)=\frac{e^{-t}}{4}\left(1+3\cos(2t)+6\sin(2t)\right)\]
	Eftersom $y(0)=\frac{e^0}{4}(1+3\cos(0)+6\sin(0))=1$ är funktionen, utöver $t>0$, definierad för $t=0$.
	
	\ans $y(t)=\frac{e^{-t}}{4}\left(1+3\cos(2t)+6\sin(2t)\right)\cond{t \ge 0}$
\end{task}

\begin{task}{5.2}
	Låt $Y(s)=\laplace(\theta y)(s)$ vilket medför att:
	\[\laplace(\theta y')(s)=s\laplace(\theta y)(s)-y(0)=sY(s)\]
	\[\laplace(\theta y'')(s)=s\laplace(\theta y')(s)-y'(0)=s^2Y(s)-1\]
	Multiplicera vänsterledet med $\theta(t)$ och hitta Laplacetransformparet:
	\[y''(t)\theta(t)-2y'(t)\theta(t)+2y(t)\theta(t) \llra
	s^2Y(s)-s-2+2(sY(s)-1)+5Y(s)=\frac{1}{s+1}\]
	\begin{align*}
	&s^2Y(s)-1-2sY(s)+2Y(s)=0 \lra
	(s^2-2s+2)Y(s)=1 \lra \\ \lra
	&Y(s)=
	\frac{1}{s^2-2s+2}=
	\frac{1}{(s-1)^2+1} \llra
	e^{t}\sin(t)\theta(t)
	\end{align*}
	\[y(t)\theta(t)=e^{t}\sin(t)\theta(t)\]
	\[y(t)=e^{t}\sin(t)\]
	Eftersom $y(0)=e^{0}\sin(0)=0$ är funktionen, utöver $t>0$, definierad för $t=0$.
	
	\ans $y(t)=e^{t}\sin(t)\cond{t \ge 0}$
\end{task}

\begin{task}{5.3}
	Låt $Y(s)=\laplace(\theta y)(s)$ vilket medför att:
	\[\laplace(\theta y')(s)=s\laplace(\theta y)(s)-y(0)=sY(s)\]
	\[\laplace(\theta y'')(s)=s\laplace(\theta y')(s)-y'(0)=s^2Y(s)\]
	Multiplicera ekvationen med $\theta(t)$ och hitta Laplacetransformparet:
	\[y''(t)\theta(t)+2y'(t)\theta(t)+y(t)\theta(t)=e^{-2t}\theta(t) \llra
	s^2Y(s)+2sY(s)+Y(s)=\frac{1}{s+2}\]
	\begin{align*}
	&s^2Y(s)+2sY(s)+Y(s)=\frac{1}{s+2} \lra
	(s^2+2s+1)Y(s)=\frac{1}{s+2} \lra \\ \lra
	&Y(s)=
	\frac{1}{(s+2)(s^2+2s+1)}=
	\frac{1}{(s+2)(s+1)^2}=
	\frac{k_1}{s+2}+\frac{k_2}{s+1}+\frac{k_3}{(s+1)^2}
	\end{align*}
	Identifiera variablerna:
	\begin{align*}
	&k_1(s+1)^2+k_2(s+2)(s+1)+k_3(s+2)=1 \lra \\ \lra
	&k_1(s^2+2s+1)+k_2(s^2+3s+2)+k_3(s+2)=1 \lra \\ \lra
	&\begin{cases}
	k_1+k_2=0 \\
	2k_1+3k_2+k_3=0 \\
	k_1+2k_2+2k_3=1
	\end{cases} \lra
	\begin{cases}
	k_1=1 \\
	k_2=-1 \\
	k_3=1
	\end{cases}
	\end{align*}
	\begin{align*}
	Y(s)=
	\frac{1}{s+2}-\frac{1}{s+1}+\frac{1}{(s+1)^2}\llra
	&e^{-2t}\theta(t)-e^{-t}\theta(t)+e^{-t}t\theta(t)= \\ =
	&(e^{-2t}+e^{-t}(t-1))\theta(t)
	\end{align*}
	\[y(t)\theta(t)=(e^{-2t}+e^{-t}(t-1))\theta(t)\]
	\[y(t)=e^{-2t}+e^{-t}(t-1)\]
	Eftersom $y(0)=e^{0}+e^{0}(0-1)=0$ är funktionen, utöver $t>0$, definierad för $t=0$.
	
	\ans $y(t)=e^{-2t}+e^{-t}(t-1)\cond{t \ge 0}$
\end{task}

\pagebreak
\begin{task}{5.4}
	Låt $Y_1(s)=\laplace(\theta y_1)(s)$ och $Y_2(s)=\laplace(\theta y_2)(s)$ vilket medför att:
	\[\laplace(\theta y_1')(s)=s\laplace(\theta y_1)(s)-y_1(0)=sY_1(s)-1\]
	\[\laplace(\theta y_2')(s)=s\laplace(\theta y_2)(s)-y_2(0)=sY_2(s)\]
	Multiplicera ekvationssystemet med $\theta(t)$:
	\[
	\begin{cases}
	y_1\theta(t)-2y_2'\theta=2\theta \\
	y_1'\theta(t)+2y_2=-2t\theta
	\end{cases}\]
	Laplacetransformera båda ekvationerna:
	\[
	\begin{cases}
	Y_1(s)-2sY_2(s)=\frac{2}{s} \\
	sY_1(s)-1+2Y_2(s)=-\frac{2}{s^2}
	\end{cases} \lra
	\begin{linsys}{rrr}
	 Y_1(s)-&2sY_2(s)=&\frac{2}{s} \\
	sY_1(s)+& 2Y_2(s)=&\frac{s^2-2}{s^2}
	\end{linsys}\]
	Använd Cramers regel:
	\begin{align*}
	\Delta(s)=\begin{detmat}
	1 & -2s \\
	s & 2
	\end{detmat}=
	2-(-2s)s=2(s^2+1)
	\end{align*}
	\begin{align*}
	Y_1(s)=
	\frac{1}{\Delta(s)}\begin{detmat}
	\frac{2}{s} & -2s \\
	\frac{s^2-2}{s^2} & 2
	\end{detmat}=
	\frac{1}{2(s^2+1)}\left(\frac{4}{s}-\frac{(-2s)(s^2-2)}{s^2}\right)=
	\frac{s}{s^2+1}
	\end{align*}
	\begin{align*}
	Y_2(s)=
	\frac{1}{\Delta(s)}\begin{detmat}
	1 &\frac{2}{s} \\
	s &\frac{s^2-2}{s^2}
	\end{detmat}=
	\frac{1}{2(s^2+1)}\left(\frac{s^2-2}{s^2}-2\right)=
	\frac{1}{2}\*\frac{1}{s^2+1}-\frac{1}{s^2}
	\end{align*}
	\[\begin{cases}
	Y_1(s)=\frac{s}{s^2+1} \\
	Y_2(s)=\frac{1}{2}\*\frac{1}{s^2+1}-\frac{1}{s^2}
	\end{cases}\llra
	\begin{cases}
	y_1\theta=\cos(t)\theta(t) \\
	y_2\theta=\frac{1}{2}\sin(t)\theta(t)-t\theta(t)
	\end{cases}\]
	Eftersom $\theta(t)=1$ endast när $t>0$ måste det läggas till som villkor:
	\[\begin{cases}
	y_1=\cos(t) \\
	y_2=\frac{1}{2}\sin(t)-t
	\end{cases}\cond{t > 0}\]
	Eftersom $y_1(0)=\cos(0)=1$ och $y_2(0)=\frac{1}{2}\sin(0)-0=0$ är de de också definierade för $t=0$.
	
	\ans 
	$\begin{cases}
	y_1=\cos(t) \\
	y_2=\frac{1}{2}\sin(t)-t
	\end{cases}\cond{t \ge 0}$
\end{task}

\begin{task}{5.5}
	Eftersom differentialekvationens högerled är $\delta(t)$ låt $Y(s)=\laplace y(s)$ ($\delta(t)\theta(t)$ saknar betydelse)
	\[sY(s)=\laplace y'(s),\qquad s^2Y(s)=\laplace y''(s)\]
	Laplacetransformera ekvationen:
	\begin{align*}
	&s^2Y(s)+2sY(s)+2Y(s)=1 \lra
	(s^2+2s+2)Y(s)=1\lra \\ \lra
	&Y(s)=\frac{1}{s^2+2s+2}=
	\frac{1}{(s+1)^2+1}\llra
	e^{-t}\sin(t)\theta(t)
	\end{align*}
	Eftersom $\theta(t)$ är en faktor är $y$ kausal.
	
	\ans $y(t)=e^{-t}\sin(t)\theta(t)$
\end{task}

\begin{task}{5.6}
	För att använda Laplacetransformation med begynnelsevärden måste $t > 0$. Så börja med att hitta lösningen för det intervallet och utvidga lösningen efter.
	
	Låt $Y(s)=\laplace(\theta y)(s)$.
	\[\laplace(\theta y')(s)=s\laplace(\theta y)(s)-y(0)=sY(s)\]
	\[\laplace(\theta y'')(s)=s\laplace(\theta y')(s)-y'(0)=s^2Y(s)\]
	\[\laplace(\theta y^{(3)})(s)=s\laplace(\theta y'')(s)-y''(0)=s^3Y(s)\]
	\[\laplace(\theta y^{(4)})(s)=s\laplace(\theta y^{(3)})(s)-y^{(3)}(0)=s^4Y(s)-1\]
	Multiplicera ekvationen med $\theta(t)$ och Laplacetransformera den:
	\begin{align*}
	&s^4Y(s)-1=Y(s) \lra
	(s^4-1)Y(s)=1 \lra \\ \lra
	Y(s)=&\frac{1}{s^4-1}=
	\frac{1}{(s^2+1)(s+1)(s-1)}=
	\frac{k_1s+k_2}{s^2+1}+\frac{k_3}{s+1}+\frac{k_4}{s-1}
	\end{align*}
	Identifiera variablerna:
	\begin{align*}
	&(k_1s+k_2)(s+1)(s-1)+k_3(s^2+1)(s-1)+k_4(s^2+1)(s+1)=1 \lra \\ \lra
	&k_1(s^3-s)+k_2(s^2-1)+k_3(s^3-s^2+s-1)+k_4(s^3+s^2+s+1)=1 \lra \\ \lra
	&\begin{cases}
	k_1+k_3+k_4=0 \\
	k_2-k_3+k_4=0 \\
	-k_1+k_3+k_4=0 \\
	-k_2-k_3+k_4=1
	\end{cases} \lra
	\begin{cases}
	k_1=0 \\
	k_2=-\frac{1}{2} \\
	k_3=-\frac{1}{4} \\
	k_4=\frac{1}{4}
	\end{cases}
	\end{align*}
	\begin{align*}
	Y(s)=
	&-\frac{1}{2}\*\frac{1}{s^2+1}-\frac{1}{4}\frac{1}{s+1}+\frac{1}{4}\*\frac{1}{s-1}=
	\frac{1}{4}\left(\frac{1}{s-1}-\frac{1}{s+1}-2\frac{1}{s^2+1}\right) \llra \\ \llra
	&\frac{1}{4}(e^{t}\theta(t)-e^{-t}\theta(t)-\sin(t)\theta(t))=
	\frac{1}{4}(e^{t}-e^{-t}-2\sin(t))\theta(t)
	\end{align*}
	\[y(t)\theta(t)=\frac{1}{4}(e^{t}-e^{-t}-2\sin(t))\theta(t)\]
	\[y(t)=\frac{1}{4}(e^{t}-e^{-t}-2\sin(t))\cond{t>0}\]
	$y$ deriveras 3 gånger och då visar det sig att $y(0)=y'(0)=y''(0)=0$, $y^{(3)}(0)=1$ vilket innebär att begynnelsevillkoren är uppfyllda.
	
	Eftersom termerna är definierade för $t<0$ och deriveras på samma sätt för negativa värden kan intervallet för $t$ utvidgas till $t\in\mathbb{R}$.
	
	\ans $y(t)=\frac{1}{4}(e^{t}-e^{-t}-2\sin(t))\cond{t\in\mathbb{R}}$
\end{task}

\pagebreak
\begin{task}{5.7}
	Låt $Y(s)=\laplace(\theta y)(s)$.
	\[\laplace(\theta y')(s)=s\laplace(\theta y)(s)-y(0)=sY(s)\]
	\[\laplace(\theta y'')(s)=s\laplace(\theta y')(s)-y'(0)=s^2Y(s)\]
	\[\laplace(\theta y''')(s)=s\laplace(\theta y'')(s)-y''(0)=s^3Y(s)\]
	Multiplicera ekvationen med $\theta(t)$ och Laplacetransformera den:
	\begin{align*}
	&s^3Y(s)+3s^2Y(s)+3sY(s)+Y(s)=\frac{1}{(s+1)^2} \lra
	(s^3+3s^2+3s+1)Y(s)=\frac{1}{(s+1)^2} \lra \\ \lra
	&Y(s)=\frac{1}{(s+1)^2(s^3+3s^2+3s+1)}=
	\frac{1}{(s+1)^5}=
	\frac{1}{24}\*\frac{4!}{(s+1)^5} \llra
	\frac{1}{24}e^{-t}t^4\theta(t)
	\end{align*}
	\[y(t)\theta(t)=\frac{1}{24}e^{-t}t^4\theta(t)\]
	\[y(t)=\frac{1}{24}e^{-t}t^4\cond{t>0}\]
	$y$ deriveras 2 gånger och då visar det sig att $y(0)=y'(0)=y''(0)=0$ vilket innebär att begynnelsevillkoren är uppfyllda, funktionen är alltså definierad även för $t=0$.
	
	\ans $y(t)=\frac{1}{24}e^{-t}t^4\cond{t\ge0}$
\end{task}

\pagebreak
\begin{task}{5.8}
	Låt $Y(s)=\laplace(\theta y)(s)$.
	\[\laplace(\theta y')(s)=s\laplace(\theta y)(s)-y(0)=sY(s)\]
	\[\laplace(\theta y'')(s)=s\laplace(\theta y')(s)-y'(0)=s^2Y(s)\]
	Multiplicera ekvationen med $\theta(t)$ (notera $\theta(t-\alpha)\theta(t-\beta)=\theta(t-\alpha)\cond{\alpha \ge \beta}$):
	\[y''\theta(t)+3y'\theta(t)+2y\theta(t)=\theta(t)\theta(t)-\theta(t-1)\theta(t) \lra
	y''\theta(t)+3y'\theta(t)+2y\theta(t)=\theta(t)-\theta(t-1)\]
	Laplacetransformera den:
	\begin{align*}
	&s^2Y(s)+3sY(s)+2Y(s)=\frac{1}{s}-e^{-s}\frac{1}{s} \lra
	(s^2+3s+2)Y(s)=\frac{1-e^{-s}}{s} \lra \\ \lra
	&Y(s)=
	\frac{1-e^{-s}}{s(s^2+3s+2)}=
	\frac{k_1}{s}+\frac{k_2}{s+1}+\frac{k_3}{s+2}-e^{-s}\left(\frac{k_1}{s}+\frac{k_2}{s+1}+\frac{k_3}{s+2}\right)
	\end{align*}
	Identifiera variablerna:
	\begin{align*}
	&k_1(s+1)(s+2)+k_2s(s+2)+k_3s(s+1)=1 \lra \\ \lra
	&k_1(s^2+3s+2)+k_2(s^2+2s)+k_3(s^2+s)=1 \lra \\ \lra
	&\begin{cases}
	k_1+k_2+k_3=0 \\
	3k_1+2k_2+k_3=0 \\
	2k_1=1
	\end{cases} \lra
	\begin{cases}
	k_1=\frac{1}{2} \\
	k_2=-1 \\
	k_3=\frac{1}{2}
	\end{cases}
	\end{align*}
	\begin{align*}
	Y(s)=
	&\frac{1}{2}\*\frac{1}{s}-\frac{1}{s+1}+\frac{1}{2}\*\frac{1}{s+2}-e^{-s}\left(\frac{1}{2}\*\frac{1}{s}-\frac{1}{s+1}+\frac{1}{2}\*\frac{1}{s+2}\right)\llra \\ \llra
	&(\frac{1}{2}-e^{-t}+\frac{1}{2}e^{-2t})\theta(t)-(\frac{1}{2}-e^{-(t-1)}+\frac{1}{2}e^{-2(t-1)})\theta(t-1)= \\ =
	&\frac{1}{2}(1-2e^{-t}+e^{-2t})\theta(t)-\frac{1}{2}(1-2e^{-(t-1)}+e^{-2(t-1)})\theta(t-1)
	\end{align*}
	Notera att $\frac{\theta(t-\alpha)}{\theta(t)}=\theta(t-\alpha)\cond{t>0}$:
	\[y(t)\theta(t)=\frac{1}{2}(1-2e^{-t}+e^{-2t})\theta(t)-\frac{1}{2}(1-2e^{-(t-1)}+e^{-2(t-1)})\theta(t-1)\]
	\[y(t)=\frac{1}{2}(1-2e^{-t}+e^{-2t})\theta(t)-\frac{1}{2}(1-2e^{-(t-1)}+e^{-2(t-1)})\theta(t-1)\cond{t>0}\]
	$y$ deriveras och då visar det sig att $y(0)=y'(0)=0$ vilket innebär att begynnelsevillkoren är uppfyllda, funktionen är alltså definierad även för $t=0$.
	
	\ans $y(t)=\frac{1}{2}(1-2e^{-t}+e^{-2t})\theta(t)-\frac{1}{2}(1-2e^{-(t-1)}+e^{-2(t-1)})\theta(t-1)\cond{t\ge0}$
\end{task}

\pagebreak
\begin{task}{5.9}
	Låt $Y(s)=\laplace(\theta y)(s)$.
	\[\laplace(\theta y')(s)=s\laplace(\theta y)(s)-y(0)=sY(s)\]
	\[\laplace(\theta y'')(s)=s\laplace(\theta y')(s)-y'(0)=s^2Y(s)\]
	\[\laplace(\theta y''')(s)=s\laplace(\theta y'')(s)-y''(0)=s^3Y(s)-1\]
	Multiplicera ekvationen med $\theta(t)$ och Laplacetransformera den:
	\begin{align*}
	&s^3Y(s)-1-s^2Y(s)+sY(s)-Y(s)=\frac{1}{s^2} \lra
	(s^3-s^2+s-1)Y(s)=\frac{1}{s^2}+1 \lra \\ \lra
	&Y(s)=\frac{s^2+1}{s^2(s^3-s^2+s-1)}=
	\frac{s^2+1}{s^2(s-1)(s^2+1)}=
	\frac{1}{s^2(s-1)}=
	\frac{k_1}{s}+\frac{k_2}{s^2}+\frac{k_3}{s-1}
	\end{align*}
	Identifiera variablerna:
	\begin{align*}
	&k_1s(s-1)+k_2(s-1)+k_3s^2=1 \lra \\ \lra
	&k_1(s^2-s)+k_2(s-1)+k_3s^2=1 \lra \\ \lra
	&\begin{cases}
	k_1+k_3=0 \\
	-k_1+k_2=0 \\
	-k_2=1
	\end{cases} \lra
	\begin{cases}
	k_1=-1 \\
	k_2=-1 \\
	k_3=1
	\end{cases}
	\end{align*}
	\begin{align*}
	Y(s)=-\frac{1}{s}-\frac{1}{s^2}+\frac{1}{s-1} \llra
	-\theta(t)-t\theta(t)+e^{t}\theta(t)=
	(e^{t}-t-1)\theta(t)
	\end{align*}
	\[y(t)\theta(t)=(e^{t}-t-1)\theta(t)\]
	\[y(t)=e^{t}-t-1\cond{t>0}\]
	$y$ deriveras 2 gånger och då visar det sig att $y(0)=y'(0)=0$, $y''(0)=1$ vilket innebär att begynnelsevillkoren är uppfyllda, funktionen är alltså definierad även för $t=0$.
	
	\ans $y(t)=e^{t}-t-1\cond{t\ge0}$
\end{task}

\begin{task}{5.10}
	Eftersom differentialekvationens högerled är $\delta(t)$ låt $Y(s)=\laplace y(s)$ ($\delta(t)\theta(t)$ saknar betydelse)
	\[sY(s)=\laplace y'(s),\qquad s^2Y(s)=\laplace y''(s),\qquad s^3Y(s)=\laplace y'''(s)\]
	Laplacetransformera ekvationen:
	\begin{align*}
	&s^3Y(s)+3s^2Y(s)+3sY(s)+Y(s)=1 \lra
	(s^3+3s^2+3s+1)Y(s)=1\lra \\ \lra
	&Y(s)=\frac{1}{s^3+3s^2+3s+1}=
	\frac{1}{2}\frac{2}{(s+1)^3}\llra
	\frac{1}{2}e^{-t}t^2\theta(t)
	\end{align*}
	Eftersom $\theta(t)$ är en faktor är $y$ kausal.
	
	\ans $y(t)=\frac{1}{2}e^{-t}t^2\theta(t)\cond{-\infty<t<+\infty}$
\end{task}

\pagebreak
\begin{task}{5.11}
	Eftersom differentialekvationens högerled innehåller $\delta(t)$ låt $Y(s)=\laplace y(s)$ ($\delta(t)\theta(t)$ saknar betydelse)
	\[sY(s)=\laplace y'(s),\qquad s^2Y(s)=\laplace y''(s)\]
	Laplacetransformera ekvationen:
	\begin{align*}
	&s^2Y(s)+sY(s)=1-2e^{-s}+\frac{1}{s}-e^{-s}\frac{1}{s} \lra
	(s^2+s)Y(s)=\frac{s-2se^{-s}+1-e^{-s}}{s}\lra \\ \lra
	&Y(s)=\frac{s-2se^{-s}+1-e^{-s}}{s(s^2+s)}=
	\frac{1}{s(s+1)}-2\frac{e^{-s}}{s(s+1)}+\frac{1}{s^2(s+1)}-\frac{e^{-s}}{s^2(s+1)}= \\ =
	&\frac{1}{s}-\frac{1}{s+1}-2e^{-s}\left(\frac{1}{s}-\frac{1}{s+1}\right)-\frac{1}{s}+\frac{1}{s^2}+\frac{1}{s+1}-e^{-s}\left(-\frac{1}{s}+\frac{1}{s^2}+\frac{1}{s+1}\right)= \\ =
	&\frac{1}{s^2}-e^{-s}\left(\frac{1}{s}+\frac{1}{s^2}-\frac{1}{s+1}\right) \llra
	t\theta(t)-(\theta(t-1)+(t-1)\theta(t-1)-e^{-(t-1)}\theta(t-1))= \\ =
	&t\theta(t)-(t-e^{-(t-1)})\theta(t-1)
	\end{align*}
	Eftersom $\theta(t-\alpha)\cond{\alpha \ge 0}$ är en faktor är $y$ kausal.
	
	För att enklare kunna rita den:
	\[y(t)=t\theta(t)-(t-e^{-(t-1)})\theta(t-1)=t(\theta(t)-\theta(t-1))+e^{-(t-1)}\theta(t-1)\]
	Funktionen $t$ i intervallet $0<t<1$, funktionen $e^{-(t-1)}$ i intervallet $1<t<\infty$
	
	\ans $y(t)=t(\theta(t)-\theta(t-1))+e^{-(t-1)}\theta(t-1)$
\end{task}

\begin{task}{5.12}
	Låt $Y(s)=\laplace(\theta y)(s)$.
	\[\laplace(\theta y')(s)=s\laplace(\theta y)(s)-y(0)=sY(s)\]
	\[\laplace(\theta y'')(s)=s\laplace(\theta y')(s)-y'(0)=s^2Y(s)\]
	Multiplicera ekvationen med $\theta(t)$ och Laplacetransformera den:
	\begin{align*}
	&s^2Y(s)+3sY(s)+2Y(s)=\frac{s}{s^2+1} \lra
	(s^2+3s+2)Y(s)=\frac{s}{s^2+1} \lra \\ \lra
	&Y(s)=\frac{s}{(s^2+1)(s^2+3s+2)}=
	\frac{s}{(s^2+1)(s+1)(s+2)}=
	\frac{k_1s+k_2}{s^2+1}+\frac{k_3}{s+1}+\frac{k_4}{s+2}
	\end{align*}
	Identifiera variablerna:
	\begin{align*}
	&(k_1s+k_2)(s^2+3s+2)+k_3(s^2+1)(s+2)+k_4(s^2+1)(s+1)=s \lra \\ \lra
	&k_1(s^3+3s^2+2s)+k_2(s^2+3s+2)+k_3(s^3+2s^2+s+2)+k_4(s^3+s^2+s+1)=s \lra \\ \lra
	&\begin{cases}
	k_1+k_3+k_4=0 \\
	3k_1+k_2+2k_3+k_4=0 \\
	2k_1+3k_2+k_3+k_4=1 \\
	2k_2+2k_3+k_4=0
	\end{cases} \lra
	\begin{cases}
	k_1=\frac{1}{10} \\
	k_2=\frac{3}{10} \\
	k_3=-\frac{1}{2} \\
	k_4=\frac{2}{5}
	\end{cases}
	\end{align*}
	\begin{align*}
	Y(s)=
	&\frac{1}{10}\*\frac{s+3}{s^2+1}-\frac{1}{2}\*\frac{1}{s+1}+\frac{2}{5}\*\frac{1}{s+2}=
	\frac{1}{10}\left(\frac{s}{s^2+1}+3\frac{1}{s^2+1}-5\frac{1}{s+1}+4\frac{1}{s+2}\right) \llra \\ \llra
	&\frac{1}{10}\left(\cos(t)+3\sin(t)-5e^{-t}+4e^{-2t}\right)\theta(t)
	\end{align*}
	\[y(t)\theta(t)=\frac{1}{10}\left(\cos(t)+3\sin(t)-5e^{-t}+4e^{-2t}\right)\theta(t)\]
	\[y(t)=\frac{1}{10}\left(\cos(t)+3\sin(t)-5e^{-t}+4e^{-2t}\right)\cond{t>0}\]
	$y$ deriveras och då visar det sig att $y(0)=y'(0)=0$ vilket innebär att begynnelsevillkoren är uppfyllda, funktionen är alltså definierad även för $t=0$.
	
	\ans $y(t)=\frac{1}{10}\left(\cos(t)+3\sin(t)-5e^{-t}+4e^{-2t}\right)\cond{t\ge0}$
\end{task}

\begin{task}{5.13 a)}
	\[f(t)=t(\theta(t)-\theta(t-1))-t(\theta(t-1)-\theta(t-2))=t\theta(t)-2t\theta(t-1)+t\theta(t-2)\]
	Hitta $f$s Laplacetransformpar:
	\[\laplace f(s)=\frac{1}{s^2}-2e^{-s}\frac{1}{s^2}+e^{-2s}\frac{1}{s^2}=\frac{1-2e^{-s}+e^{-2s}}{s^2}\]
	Utnyttja regeln $f'(t)\llra sF(s)$:
	\[\laplace f'(s)=s\frac{1-2e^{-s}+e^{-2s}}{s^2}=\frac{1-2e^{-s}+e^{-2s}}{s}\]
	\[\laplace f''(s)=s\frac{1-2e^{-s}+e^{-2s}}{s}=1-2e^{-s}+e^{-2s}\]
	Hitta inversen:
	\[f''(t)=\delta(t)-2\delta(t-1)+\delta(t-2)\]
	\ans $f''(t)=\delta(t)-2\delta(t-1)+\delta(t-2)$
\end{task}

\begin{task}{b)}
	Eftersom differentialekvationens högerled innehåller $\delta(t)$ låt $Y(s)=\laplace y(s)$ ($\delta(t)\theta(t)$ saknar betydelse)
	\[sY(s)=\laplace y'(s),\qquad s^2Y(s)=\laplace y''(s)\]
	Laplacetransformera ekvationen:
	\begin{align*}
	&s^2Y(s)+2sY(s)+2Y(s)=1-2e^{-s}+e^{-2s} \lra
	(s^2+2s+2)Y(s)=1-2e^{-s}+e^{-2s}\lra \\ \lra
	&Y(s)=\frac{1-2e^{-s}+e^{-2s}}{s^2+2s+2}=
	\frac{1}{(s+1)^2+1}-2\frac{e^{-s}}{(s+1)^2+1}+\frac{e^{-2s}}{(s+1)^2+1} \llra \\ \llra
	&e^{-t}\sin(t)\theta(t)-2e^{-(t-1)}\sin(t-1)\theta(t-1)+e^{-(t-2)}\sin(t-2)\theta(t-2)
	\end{align*}
	Eftersom $\theta(t-\alpha)\cond{\alpha \ge 0}$ är en faktor är $y$ kausal.
	
	\ans $y(t)=e^{-t}\sin(t)\theta(t)-2e^{-(t-1)}\sin(t-1)\theta(t-1)+e^{-(t-2)}\sin(t-2)\theta(t-2)$
\end{task}