\chapter{4}{Den inversa Laplacetransformen}

\begin{task}{4.1}
	multiplicera faktorerna.
	\begin{align*}
	F(s)=
	&\frac{(s-1)(s+1)}{(s+2)^2(s+1+2i)(s+1-2i)}=
	\frac{s^2-1}{(s^2+2s+4)((s+1)^2+4)}= \\ =
	&\frac{s^2-1}{(s^2+4s+4)(s^2+2s+5)}=
	\frac{s^2-1}{s^4+6s^3+17s^2+28s+20}
	\end{align*}
\end{task}

\begin{task}{4.2 a)}
	Använd tabellen:
	\[\laplace f(s)=\frac{1}{s}\llra \theta(t)=f(t)\]
	\ans $f(t)=\theta(t)$
\end{task}

\begin{task}{b)}
	Faktorisera och använd tabellen:
	\[\laplace f(s)=\frac{1}{s^3}=\frac{1}{2}\*\frac{2}{s^{2+1}}\llra \frac{1}{2}t^2\theta(t)=f(t)\]
	\ans $f(t)=\dfrac{1}{2}t^2\theta(t)$
\end{task}

\begin{task}{c)}
	Faktorisera och använd tabellen:
	\begin{align*}
	\laplace f(s)=
	&\frac{s^4+6s^3-10s^2+1}{s^5}=
	\frac{1}{s}+6\frac{1}{s^2}-10\frac{1}{s^3}+\frac{1}{s^5}=
	\frac{1}{s}+6\frac{1}{s^2}-5\frac{2!}{s^3}+\frac{1}{4!}\*\frac{4!}{s^5}\llra \\ \llra
	&\theta(t)+6t\theta(t)-5t^2\theta(t)+\frac{1}{24}t^4\theta(t)=
	(1+6t-5t^2+\frac{1}{24}t^4)\theta(t)=
	f(t)
	\end{align*}
	\ans $f(t)=(1+6t-5t^2+\frac{1}{24}t^4)\theta(t)$
\end{task}

\begin{task}{4.2 a)}
	Faktorisera och använd tabellen:
	\[\laplace f(s)=\frac{2}{s+3}=2\frac{0!}{(s-(-3))^{0+1}}\llra 2e^{-3t}\theta(t)=f(t)\]
	\ans $f(t)=2e^{-3t}\theta(t)$
\end{task}

\begin{task}{b)}
	Faktorisera och använd tabellen:
	\begin{align*}
	\laplace f(s)=
	&\frac{1}{s+3}-\frac{2}{(s+3)^2}+\frac{1}{(s+3)^3}=
	\frac{1}{s+3}-2\frac{1}{(s+3)^2}+\frac{1}{2}\*\frac{2}{(s+3)^3}\llra \\ \llra
	&e^{-3t}\theta(t)-2te^{-3t}\theta(t)+\frac{1}{2}t^2e^{-3t}\theta(t)=
	(1-2t+\frac{1}{2}t^2)e^{-3t}\theta(t)=
	f(t)
	\end{align*}
	\ans $f(t)=(1-2t+\frac{1}{2}t^2)e^{-3t}\theta(t)$
\end{task}

\begin{task}{c)}
	Faktorisera och använd tabellen:
	\begin{align*}
	\laplace f(s)=
	&\frac{s+5}{(s+3)^2}=
	\frac{s+3+2}{(s+3)^2}=
	\frac{s+3}{(s+3)^2}+\frac{2}{(s+3)^2}=
	\frac{1}{s+3}+2\frac{1}{(s+3)^2}\llra \\ \llra
	&e^{-3t}\theta(t)+2te^{-3t}\theta(t)=
	(1+2t)e^{-3t}\theta(t)=
	f(t)
	\end{align*}
	\ans $f(t)=(1+2t)e^{-3t}\theta(t)$
\end{task}

\begin{task}{4.4 a)}
	Faktorisera och använd tabellen:
	\begin{align*}
	\laplace f(s)=
	&\frac{s}{s^2+6s+8}=
	\frac{s}{(s+4)(s+2)}=
	\frac{2}{s+4}-\frac{1}{s+2}=
	2\frac{1}{s+4}-\frac{1}{s+2}\llra  \\ \llra
	&2e^{-4t}\theta(t)-e^{-2t}\theta(t)=
	(2e^{-4t}-e^{-2t})\theta(t)=
	f(t)
	\end{align*}
	\ans $f(t)=(2e^{-4t}-e^{-2t})\theta(t)$
\end{task}

\begin{task}{b)}
	Kvadratkomplettera, faktorisera och använd tabellen:
	\begin{align*}
	\laplace f(s)=
	&\frac{s}{s^2+6s+10}=
	\frac{s}{(s+3)^2+1}=
	\frac{s+3}{(s+3)^2+1}-3\frac{1}{(s+3)^2+1}
	\end{align*}
	Använd dämpningsregeln:
	\[\frac{s+3}{(s+3)^2+1}\llra e^{-3t}\cos(t)\theta(t)\]
	\[\frac{1}{(s+3)^2+1}\llra e^{-3t}\sin(t)\theta(t)\]
	\[f(t)=e^{-3t}\cos(t)\theta(t)-3e^{-3t}\sin(t)\theta(t)=(\cos t-3\sin t)e^{-3t}\theta(t)\]
	\ans $f(t)=(\cos t-3\sin t)e^{-3t}\theta(t)$
\end{task}

\begin{task}{4.5 a)}
	Använd tabellen:
	\begin{align*}
	\laplace f(s)=
	&\frac{1}{s^2+16}=
	\frac{1}{4}\*\frac{4}{s^2+4^4} \llra 
	\frac{1}{4}\sin(4t)\theta(t)=f(t)
	\end{align*}
	\ans $f(t)=\frac{1}{4}\sin(4t)\theta(t)$
\end{task}

\begin{task}{b)}
	Använd tabellen:
	\begin{align*}
	\laplace f(s)=
	&\frac{s}{s^2+16}=
	\frac{s}{s^2+4^4} \llra 
	\cos(4t)\theta(t)=f(t)
	\end{align*}
	\ans $f(t)=\cos(4t)\theta(t)$
\end{task}

\begin{task}{c)}
	Kvadratkomplettera och använd dämpningsregeln och tabellen:
	\begin{align*}
	\laplace f(s)=
	&\frac{1}{s^2+4s+8}=
	\frac{1}{(s+2)^2+2^2}=
	\frac{1}{2}\*\frac{2}{(s+2)^2+2^2} \llra \\ \llra
	&\frac{1}{2}e^{-2t}\sin(2t)\theta(t)=
	f(t)
	\end{align*}
	\ans $f(t)=\frac{1}{2}e^{-2t}\sin(2t)\theta(t)$
\end{task}

\begin{task}{d)}
	Kvadratkomplettera, faktorisera och använd dämpningsregeln och tabellen:
	\begin{align*}
	\laplace f(s)=
	&\frac{s}{s^2+4s+8}=
	\frac{s}{(s+2)^2+2^2}=
	\frac{s+2}{(s+2)^2+2^2}-\frac{2}{(s+2)^2+2^2} \llra \\ \llra
	&e^{-2t}\cos(2t)\theta(t)-e^{-2t}\sin(2t)\theta(t)=
	(\cos(2t)-\sin(2t))e^{-2t}\theta(t)=
	f(t)
	\end{align*}
	\ans $f(t)=(\cos(2t)-\sin(2t))e^{-2t}\theta(t)$
\end{task}

\begin{task}{4.6 a)}
	Hitta nollställena:
	\[s^2-s-2=0 \lra
	s=\frac{1}{2}\pm\sqrt{\frac{1}{4}+2}=
	\frac{1}{2}\pm\frac{3}{2}=\{2,-1\}\]
	Faktorisera och använd dämpningsregeln och tabellen:
	\begin{align*}
	\laplace f(s)=
	&\frac{s+3}{s^2-s-2}=
	\frac{s+3}{(s-2)(s+1)}=
	\frac{k_1}{s-2}+\frac{k_2}{s+1}
	\end{align*}
	\[k_1(s+1)+k_2(s-2)=s+3 \lra
	\begin{cases}
	k_1+k_2=1 \\
	k_2-2k_1=3
	\end{cases} \lra
	\begin{cases}
	k_1=-\frac{2}{3} \\
	k_2=\frac{5}{3}
	\end{cases} \]
	\begin{align*}
	\laplace f(s)=
	-\frac{2}{3}\*\frac{1}{s-2}+\frac{5}{3}\*\frac{1}{s+1} \llra
	-\frac{2}{3}e^{-2t}\theta(t)+\frac{5}{3}e^{t}\theta(t)=
	(5e^{t}-2e^{-2t})\frac{1}{3}\theta(t)
	\end{align*}
	\ans $f(t)=(5e^{t}-2e^{-2t})\frac{1}{3}\theta(t)$
\end{task}

\begin{task}{b)}
	Hitta nollställena:
	\[s^2+3s+2=0 \lra
	s=-\frac{3}{2}\pm\sqrt{\frac{9}{4}-2}=
	-\frac{3}{2}\pm\frac{1}{2}=\{-1,-2\}\]
	Faktorisera och använd dämpningsregeln och tabellen:
	\begin{align*}
	\laplace f(s)=
	&\frac{3s+5}{s^3+3s^2+2s}=
	\frac{3s+5}{s(s+1)(s+2)}=
	\frac{k_1}{s}+\frac{k_2}{s+1}+\frac{k_3}{s+2}
	\end{align*}
	\[k_1(s+1)(s+2)+k_2s(s+2)+k_3s(s+1)=3s+5 \lra
	\begin{cases}
	k_1+k_2+k_3=0 \\
	3k_1+2k_2+k_3=3 \\
	2k_1=5
	\end{cases} \lra
	\begin{cases}
	k_1=\frac{5}{2} \\
	k_2=-2 \\
	k_3=-\frac{1}{2}
	\end{cases} \]
	\begin{align*}
	\laplace f(s)=
	&\frac{5}{2}\*\frac{1}{s}-2\*\frac{1}{s+1}-\frac{1}{2}\*\frac{1}{s+2} \llra
	\frac{5}{2}\theta(t)-2e^{-t}\theta(t)-\frac{1}{2}e^{-2t}\theta(t)= \\ =
	&\frac{1}{2}(5-4e^{-t}-e^{-2t})\theta(t)
	\end{align*}
	\ans $f(t)=\frac{1}{2}(5-4e^{-t}-e^{-2t})\theta(t)$
\end{task}

\begin{task}{c)}
	Hitta nollställena:
	\[s^2+4s+3=0 \lra
	s=-2\pm\sqrt{4-3}=
	-2\pm1=\{-1,-3\}\]
	Faktorisera och använd dämpningsregeln och tabellen:
	\begin{align*}
	\laplace f(s)=
	&\frac{s^3-5s}{s^2+4s+3}=
	s\frac{s^2-5}{(s+1)(s+3)}=
	s\left(\frac{k_1}{s+1}+\frac{k_2}{s+3}+k_3\right)
	\end{align*}
	\[k_1(s+3)+k_2(s+1)+k_3(s+3)(s+1)=s^2-5 \lra
	\begin{cases}
	k_3=1 \\
	k_1+k_2+4k_3=0 \\
	3k_1+k_2+3k_3=-5
	\end{cases} \lra
	\begin{cases}
	k_1=-2 \\
	k_2=-2 \\
	k_3=1
	\end{cases} \]
	Använd regeln $sF(s)\llra f'(t)$ och $\delta(t)=1$ för att bestämma inversen till $s$.
	\[s=s\*1\llra\delta'(t)\]
	\begin{align*}
	\laplace f(s)=
	&-2\frac{s}{s+1}-2\frac{s}{s+3}+s =
	-2\frac{s+1}{s+1}+2\frac{1}{s+1}-2\frac{s+3}{s+3}+6\frac{1}{s+3}+s= \\ =
	&-4+2\frac{1}{s+1}+6\frac{1}{s+3}+s\llra
	-4\delta(t)+2e^{-t}\theta(t)+6e^{-3t}\theta(t)+\delta'(t)= \\ =
	&\delta'(t)-4\delta(t)+(2e^{-t}+6e^{-3t})\theta(t)
	\end{align*}
	\ans $f(t)=\delta'(t)-4\delta(t)+(2e^{-t}+6e^{-3t})\theta(t)$
\end{task}

\begin{task}{4.7}
	En av polerna är $s=-1$ (se anvisningen).
	\begin{align*}
	&s^3+5s^2+9s+5=(s+1)(s^2+As+B)=s^3+(A+1)s^2+(B+A)s+B \lra \\ \lra
	&\begin{cases}
	A+1=5 \\
	B+A=9 \\
	B=5
	\end{cases} \lra
	\begin{cases}
	A=4 \\
	B=5
	\end{cases} \lra
	s^3+5s^2+9s+5=(s+1)(s^2+4s+5)
	\end{align*}
	Faktorisera och använd dämpningsregeln och tabellen:
	\[\frac{s}{s^3+5s^2+9s+5}=\frac{k_1}{s+1}+\frac{k_2s+k_3}{s^2+4s+5}\]
	\[k_1(s^2+4s+5)+(k_2s+k_3)(s+1)=s \lra
	\begin{cases}
	k_1+k_2=0 \\
	4k_1+k_2+k_3=1 \\
	5k_1+k_3=0
	\end{cases} \lra
	\begin{cases}
	k_1=-\frac{1}{2} \\
	k_2=\frac{1}{2} \\
	k_3=\frac{5}{2}
	\end{cases} \]
	\begin{align*}
	\frac{s}{s^3+5s^2+9s+5}=
	&-\frac{1}{2}\*\frac{1}{s+1}+\frac{1}{2}\*\frac{s+5}{s^2+4s+5}=
	-\frac{1}{2}\*\frac{1}{s+1}+\frac{1}{2}\*\frac{s+2+3}{(s+2)^2+1}= \\ =
	&-\frac{1}{2}\*\frac{1}{s+1}+\frac{1}{2}\*\frac{s+2}{(s+2)^2+1}+\frac{3}{2}\*\frac{1}{(s+2)^2+1} \llra \\ \llra
	&-\frac{1}{2}e^{-t}\theta(t)+\frac{1}{2}e^{-2t}\cos(t)\theta(t)+\frac{3}{2}e^{-2t}\sin(t)\theta(t)= \\ =
	&\frac{1}{2}(-e^{-t}+(\cos t+3\sin t)e^{-2t})\theta(t)
	\end{align*}
	\ans $\laplace^{-1}\left(\dfrac{s}{s^3+5s^2+9s+5}\right)=\dfrac{1}{2}(-e^{-t}+(\cos t+3\sin t)e^{-2t})\theta(t)$
\end{task}

\begin{task}{4.8}
	Faktorisera och använd tabellen:
	\[\frac{1}{s^3(s^2+1)}=\frac{k_1s+k_2}{s^2+1}+\frac{k_3}{s}+\frac{k_4}{s^2}+\frac{k_5}{s^3}\]
	\[s^3(k_1s+k_2)+k_3s^2(s^2+1)+k_4s(s^2+1)+k_5(s^2+1)=1 \lra
	\begin{cases}
	k_1+k_3=0 \\
	k_2+k_4=0 \\
	k_3+k_5=0 \\
	k_4=0 \\
	k_5=1
	\end{cases} \lra
	\begin{cases}
	k_1=1 \\
	k_2=0 \\
	k_3=-1 \\
	k_4=0 \\
	k_5=1
	\end{cases} \]
	\begin{align*}
	\frac{1}{s^3(s^2+1)}=
	\frac{s}{s^2+1}-\frac{1}{s}+\frac{1}{s^3}=
	\frac{s}{s^2+1}-\frac{1}{s}+\frac{1}{2}\*\frac{2}{s^3} \llra
	&\cos(t)\theta(t)-\theta(t)+\frac{1}{2}t^2\theta(t) = \\ =
	&\left(\cos(t)-1+\frac{1}{2}t^2\right)\theta(t)
	\end{align*}
	\ans $\laplace^{-1}\left(\dfrac{1}{s^3(s^2+1)}\right)=\left(\cos(t)-1+\frac{1}{2}t^2\right)\theta(t)$
\end{task}

\begin{task}{4.9 a)}
	Låt $\laplace g(s)=\dfrac{1}{s}\llra\theta(t)$ och använd reglerna för dämpning och förskjutning.
	\[\laplace f(s)=\frac{e^{-s}}{s+1}=e^{-s}\laplace g(s+1) \llra e^{-(t-1)}g(t-1)=e^{1-t}\theta(t-1)\]
	\ans $f(t)=e^{1-t}\theta(t-1)$
\end{task}

\begin{task}{b)}
	Se \taskref{4.6 b)} för uträkning av nollställena.
	\[\laplace f(s)=\frac{se^{-s}}{s^2+3s+2}=e^{-s}\left(\frac{k_1}{s+1}+\frac{k_2}{s+2}\right)\]
	\[k_1(s+2)+k_2(s+1)=s\lra
	\begin{cases}
	k_1+k_2=1 \\
	2k_1+k_2=0
	\end{cases}\lra
	\begin{cases}
	k_1=-1 \\
	k_2=2
	\end{cases}\]
	Använd reglerna för dämpning och förskjutning:
	\begin{align*}
	\laplace f(s)=
	e^{-s}\left(-\frac{1}{s+1}+2\frac{1}{s+2}\right)\llra
	&e^{-(t-1)}\theta(t-1)+2e^{-2(t-1)}\theta(t-1)= \\ =
	&(-e^{1-t}+2e^{2-2t})\theta(t-1)
	\end{align*}
	\ans $f(t)=(-e^{1-t}+2e^{2-2t})\theta(t-1)$
\end{task}

\begin{task}{c)}
	Se \taskref{b)} för beräkning av den andra termen.
	\[\laplace f(s)=\frac{1+se^{-s}}{s^2+3s+2}=\frac{1}{(s+1)(s+2)}+\frac{se^{-s}}{(s+1)(s+2)}=\frac{k_1}{s+1}+\frac{k_2}{s+2}+e^{-s}\left(-\frac{1}{s+1}+2\frac{1}{s+2}\right)\]
	\[k_1(s+2)+k_2(s+1)=1\lra
	\begin{cases}
	k_1+k_2=0 \\
	2k_1+k_2=1
	\end{cases}\lra
	\begin{cases}
	k_1=1 \\
	k_2=-1
	\end{cases}\]
	Använd regeln för dämpning (samt \taskref{b)}):
	\begin{align*}
	\laplace f(s)=
	&\frac{1}{s+1}-\frac{1}{s+2}+e^{-s}\left(-\frac{1}{s+1}+2\frac{1}{s+2}\right)\llra \\ \llra
	&e^{-t}\theta(t)-e^{-2}\theta(t)+(-e^{1-t}+2e^{2-2t})\theta(t-1)= \\ =
	&(e^{-t}-e^{-2t})\theta(t)+(-e^{1-t}+2e^{2-2t})\theta(t-1)
	\end{align*}
	\ans $f(t)=(e^{-t}-e^{-2t})\theta(t)+(-e^{1-t}+2e^{2-2t})\theta(t-1)$
\end{task}

\begin{task}{4.10 a)}
	Låt $\laplace g(s)=\dfrac{1}{s}\llra\theta(t)$ och använd reglerna för dämpning och förskjutning.
	\[\laplace f(s)=e^{-5s}\frac{1}{s+2}=e^{-5s}\laplace g(s+2) \llra e^{-2(t-5)}g(t-5)=e^{2(5-t)}\theta(t-5)\]
	\ans $f(t)=e^{2(5-t)}\theta(t-5)$
\end{task}

\begin{task}{b)}
	Faktorisera och använd regeln för förskjutning.
	\begin{align*}
	\laplace f(s)=
	&(e^{-\pi s}+e^{-2\pi s})\frac{1}{s^2+1}=
	e^{-\pi s}\frac{1}{s^2+1}+e^{-2\pi s}\frac{1}{s^2+1} \llra \\ \llra
	&\sin(t-\pi)\theta(t-\pi)+\sin(t-2\pi)\theta(t-2\pi)
	\end{align*}
	\ans $f(t)=\sin(t-\pi)\theta(t-\pi)+\sin(t-2\pi)\theta(t-2\pi)$
\end{task}

\begin{task}{c)}
	Faktorisera och använd reglerna för förskjutning och dämpning.
	\begin{align*}
	\laplace f(s)=
	&\frac{e^{2s}}{s^2+s}=
	e^{2s}\left(\frac{k_1}{s}+\frac{k_2}{s+1}\right)
	\end{align*}
	\[k_1(s+1)+k_2s=1\lra
	\begin{cases}
	k_1+k_2=0 \\
	k_1=1
	\end{cases}\lra
	\begin{cases}
	k_1=1 \\
	k_2=-1
	\end{cases}\]
	\begin{align*}
	\laplace f(s)=
	e^{2s}\left(\frac{1}{s}-\frac{1}{s+1}\right)=
	e^{2s}\frac{1}{s}-e^{2s}\frac{1}{s+1} \llra
	&\theta(t+2)-e^{-(t+2)}\theta(t+2)= \\ =
	&(1-e^{-t-2})\theta(t+2)
	\end{align*}
	\ans $f(t)=(1-e^{-t-2})\theta(t+2)$
\end{task}

\begin{task}{d)}
	Faktorisera och använd reglerna för förskjutning och dämpning.
	\begin{align*}
	\laplace f(s)=
	&\frac{2-2e^{-s}-se^{-s}}{s^2-1}= \\ =
	&\left(\frac{1}{s-1}-\frac{1}{s+1}\right)-e^{-s}\left(\frac{1}{s-1}-\frac{1}{s+1}\right)-\frac{1}{2}e^{-s}\left(\frac{1}{s-1}+\frac{1}{s+1}\right)= \\ =
	&\frac{1}{s-1}-\frac{1}{s+1}-\frac{3}{2}e^{-s}\frac{1}{s-1}+\frac{1}{2}e^{-s}\frac{1}{s+1} \llra \\ \llra
	&e^{t}\theta(t)-e^{-t}\theta(t)-\frac{3}{2}e^{t-1}\theta(t-1)+\frac{1}{2}e^{-(t-1)}\theta(t-1)= \\ =
	&(e^{t}-e^{-t})\theta(t)+\frac{1}{2}(e^{1-t}-3e^{t-1})\theta(t-1)
	\end{align*}
	\ans $f(t)=(e^{t}-e^{-t})\theta(t)+\frac{1}{2}(e^{1-t}-3e^{t-1})\theta(t-1)$
\end{task}

\begin{task}{4.11 a)}
	Bygynnelsevärdessatsen kan inte användas eftersom $F(s)$ inte uppfyller villkoret om att vara ett äkta bråk (både nämnaren och täljaren har graden 2).
\end{task}

\begin{task}{b)}
	\begin{align*}
	\lim\limits_{s\rightarrow +\infty}sF(s)=
	&\lim\limits_{s\rightarrow +\infty}\frac{s^2}{(s+1)(s-2)}=
	\lim\limits_{s\rightarrow +\infty}\frac{s^2}{s^2-s-2}=
	\lim\limits_{s\rightarrow +\infty}\frac{\frac{s^2}{s^2}}{\frac{s^2}{s^2}-\frac{s}{s^2}-\frac{2}{s^2}}= \\ =
	&\lim\limits_{s\rightarrow +\infty}\frac{1}{1-\frac{1}{s}-\frac{2}{s^2}}\rightarrow
	\frac{1}{1-0-0}=
	1=\lim\limits_{t\rightarrow +0}f(t)
	\end{align*}
	\ans $\lim\limits_{t\rightarrow +0}f(t)=1$
\end{task}

\begin{task}{c)}
	\begin{align*}
	\lim\limits_{s\rightarrow +\infty}sF(s)=
	&\lim\limits_{s\rightarrow +\infty}\frac{s}{s(s+1)(s+2)}=
	\lim\limits_{s\rightarrow +\infty}\frac{1}{s^2+3s+2}=
	\lim\limits_{s\rightarrow +\infty}\frac{\frac{1}{s^2}}{\frac{s^2}{s^2}+\frac{3s}{s^2}+\frac{2}{s^2}}= \\ =
	&\lim\limits_{s\rightarrow +\infty}\frac{\frac{1}{s^2}}{1+\frac{3}{s}+\frac{2}{s^2}}\rightarrow
	\frac{0}{1+0+0}=
	0=\lim\limits_{t\rightarrow +0}f(t)
	\end{align*}
	\ans $\lim\limits_{t\rightarrow +0}f(t)=0$
\end{task}

\begin{task}{d)}
	\begin{align*}
	\lim\limits_{s\rightarrow +\infty}sF(s)=
	&\lim\limits_{s\rightarrow +\infty}\frac{s(s^2+3s+2)}{(s+1)^3}=
	\lim\limits_{s\rightarrow +\infty}\frac{s^3+3s^2+2s}{s^3+3s^2+3s+1}=
	\lim\limits_{s\rightarrow +\infty}\frac{\frac{s^3}{s^3}+\frac{3s^2}{s^3}+\frac{2s}{s^3}}{\frac{s^3}{s^3}+\frac{3s^2}{s^3}+\frac{3s}{s^3}+\frac{1}{s^3}}= \\ =
	&\lim\limits_{s\rightarrow +\infty}\frac{1+\frac{3}{s}+\frac{2}{s^2}}{1+\frac{3}{s}+\frac{3}{s^2}+\frac{1}{s^3}}\rightarrow
	\frac{1+0+0}{1+0+0+0}=
	1=\lim\limits_{t\rightarrow +0}f(t)
	\end{align*}
	\ans $\lim\limits_{t\rightarrow +0}f(t)=1$
\end{task}

\begin{task}{e)}
	\begin{align*}
	\lim\limits_{s\rightarrow +\infty}sF(s)=
	&\lim\limits_{s\rightarrow +\infty}\frac{s}{s(s+1)(s^2+1)}=
	\lim\limits_{s\rightarrow +\infty}\frac{1}{s^3+s^2+s+1}=
	\lim\limits_{s\rightarrow +\infty}\frac{\frac{1}{s^3}}{\frac{s^3}{s^3}+\frac{s^2}{s^3}+\frac{s}{s^3}+\frac{1}{s^3}}= \\ =
	&\lim\limits_{s\rightarrow +\infty}\frac{\frac{1}{s^3}}{1+\frac{1}{s}+\frac{1}{s^2}+\frac{1}{s^3}}\rightarrow
	\frac{0}{1+0+0+0}=
	0=\lim\limits_{t\rightarrow +0}f(t)
	\end{align*}
	\ans $\lim\limits_{t\rightarrow +0}f(t)=0$
\end{task}