\chapter{1}{Svängningar och komplexa tal}
\begin{task}{1.1 a)}
Allmänna funktionen för odämpad harmonisk svängning är $u(t)=A\sin(\omega t + \alpha)$ där $\omega$ är vinkelfrekvensen.
\[u(t)=3\sin(2t-5) \ra \omega=2\]
\[T=\frac{2\pi}{\omega} \ra T = \frac{2\pi}{2} = \pi\]
\[f=\frac{1}{T} \ra f = \frac{1}{\pi}\]
\ans vinkelfrekvens: 2, period: $\pi$, frekvens: $\frac{1}{\pi}$
\end{task}

\begin{task}{b)}
	Allmänna funktionen för odämpad harmonisk svängning är $u(t)=A\sin(\omega t + \alpha)$ där $\omega$ är vinkelfrekvensen.
	\[u(t)=50\sin(100\pi t+1) \ra \omega=100\pi\]
	\[T=\frac{2\pi}{\omega} \ra T = \frac{2\pi}{100\pi} = \frac{1}{50}\]
	\[f=\frac{1}{T} \ra f = 50\]
	\ans vinkelfrekvens: $100\pi$, period: $\frac{1}{50}$, frekvens: 50
\end{task}

\begin{task}{1.2 a)}
\end{task}

\begin{task}{b)}
\end{task}

\begin{task}{c)}
\end{task}

\begin{task}{d)}
\end{task}

\begin{task}{e)}
\end{task}

\begin{task}{f)}
\end{task}

\begin{task}{1.3}
	Använd regeln $\sin(\alpha+\beta)=\sin\alpha\cos\beta+\cos\alpha\sin\beta$ från formelbladet.
	\begin{align*}
	u(t)= &
	6\sin(3t+\frac{\pi}{4})=
	6(\sin(3t)\cos(\frac{\pi}{4})+\cos(3t)\sin(\frac{\pi}{4}))= \\ =
	& 6\frac{1}{\sqrt{2}}\sin(3t)+6\frac{1}{\sqrt{2}}\cos(3t)=
	3\sqrt{2}\cos(3t)+3\sqrt{2}\sin(3t)
	\end{align*}
	\ans $a=b=3\sqrt{2}, \omega=3 \ra 3\sqrt{2}\cos(3t)+3\sqrt{2}\sin(3t)$
\end{task}

\pagebreak
\begin{task}{1.4 a)}
	låt $u(t)=A\sin(\omega t + \alpha)=A\sin\alpha\cos(\omega t)+A\cos\alpha\sin(\omega t)=\sqrt{3}\cos(\omega t)-\sin(\omega t)$ där $A$ är amplituden och $\alpha$ är fasförskjutningen.
	\begin{align*}
		\begin{linsys}{rr}
		A\sin\alpha=&\sqrt{3} \\
		A\cos\alpha=&-1
		\end{linsys} \lra &
		\sqrt{(A\sin\alpha)^2+(A\cos\alpha)^2}=\sqrt{(\sqrt{3})^2+(-1)^2} \lra \\ \lra
		& \sqrt{A^2}\sqrt{\sin\alpha^2+\cos\alpha^2}=\sqrt{4} \ra
		A\sqrt{1}=2 \lra
		A=2
	\end{align*}
	\begin{align*}
		\tan\alpha= &
		\frac{\sin\alpha}{\cos\alpha}=
		\frac{A\sin\alpha}{A\cos\alpha}=
		\frac{\sqrt{3}}{-1} \ra \\ \ra
		\alpha= &
		\arctan(-\frac{\sqrt{3}}{1})+\pi=
		-\frac{\pi}{6}+\pi=
		\frac{2\pi}{3}~~(+\pi \text{ ty } -4<0)
	\end{align*}
	eller:
	\begin{align*}
		u(t)= &
		\sqrt{3}\cos(\omega t)-\sin(\omega t)=
		2(\frac{\sqrt{3}}{2}\cos(\omega t)-\frac{1}{2}\sin(\omega t))= \\ =
		& 2(\sin\frac{2\pi}{3}\cos(\omega t)+\cos\frac{2\pi}{3}\sin(\omega t))=
		\sin(\omega t + \frac{2\pi}{3})
	\end{align*}
	\ans Amplitud: $2$ och fasförskjutning: $\frac{2\pi}{3}$
\end{task}

\begin{task}{1.4 b)}
	låt $u(t)=A\sin(\omega t + \alpha)=A\sin\alpha\cos(\omega t)+A\cos\alpha\sin(\omega t)=-2\cos(\omega t)-4\sin(\omega t)$ där $A$ är amplituden och $\alpha$ är fasförskjutningen.
	\begin{align*}
	\begin{linsys}{rr}
	A\sin\alpha=&-2 \\
	A\cos\alpha=&-4
	\end{linsys} \lra &
	\sqrt{(A\sin\alpha)^2+(A\cos\alpha)^2}=\sqrt{(-2)^2+(-4)^2} \lra \\ \lra
	& \sqrt{A^2}\sqrt{\sin\alpha^2+\cos\alpha^2}=\sqrt{4+16} \ra
	A\sqrt{1}=\sqrt{20} \lra
	A=2\sqrt{5}
	\end{align*}
	\begin{align*}
	\tan\alpha= &
	\frac{\sin\alpha}{\cos\alpha}=
	\frac{A\sin\alpha}{A\cos\alpha}=
	\frac{-2}{-4} \ra \\ \ra
	\alpha= &
	\arctan\frac{1}{2}+\pi~~(+\pi \text{ ty } -4<0)
	\end{align*}
	\ans Amplitud: $2\sqrt{5}$ och fasförskjutning: $\arctan\frac{1}{2}+\pi$
\end{task}

\begin{task}{1.5 a)}
	Eftersom $\abs{a+bi}=\sqrt{a^2+b^2}$.
	\[\abs{i}=\sqrt{0^2+1^2}=1\]
	\ans $\abs{i}=1$
\end{task}

\begin{task}{b)}
	Eftersom $\abs{a+bi}=\sqrt{a^2+b^2}$.
	\[\abs{-i}=\sqrt{0^2+(-1)^2}=1\]
	\ans $\abs{-i}=1$
\end{task}

\begin{task}{c)}
	Eftersom $\abs{e^{i\phi}}=1$ oberoende av vad vinkeln $\phi$ är.
	
	\ans $\abs{e^{5\pi i/7}}=1$
\end{task}

\begin{task}{1.6 a)}
	låt $e^{i\phi}=e^{5\pi i/7} \lra \phi=\frac{5\pi}{7}$. Eftersom $\frac{\pi}{2}<\phi<\pi \ra e^{5\pi i/7}$ ligger i andra kvadranten.
	
	\ans andra kvadranten
\end{task}

\begin{task}{b)}
	Låt 
	$e^{i\phi}=e^{-34\pi i/7} \lra 
	\phi=
	-\frac{34}{7}\pi=
	-\frac{35}{7}\pi+\frac{1}{7}\pi=
	-6\pi+\pi+\frac{1}{7}\pi \ra
	\phi=\pi+\frac{1}{7}\pi$.
	Eftersom perioden är $2\pi \ra e^{i\phi} = e^{i\phi}$ vilket innebär $\pi<\phi<\frac{3}{2}\pi \ra e^{-34\pi i/7}$ ligger i tredje kvadranten.
	
	\ans tredje kvadranten
\end{task}

\begin{task}{c)}
	Låt 
	$e^{i\phi}=e^{2000\pi i/13} \lra 
	\phi=
	\frac{2000}{13}\pi=
	\frac{1989}{13}\pi+\frac{11}{13}\pi=
	152\pi+\pi+\frac{11}{13}\pi \ra
	\phi=\pi+\frac{11}{13}\pi$.
	Eftersom perioden är $2\pi \ra e^{i\phi} = e^{i\phi}$ vilket innebär $\frac{3}{2}\pi<\phi<2\pi \ra e^{2000\pi i/13}$ ligger i fjärde kvadranten.
	
	\ans fjärde kvadranten
\end{task}

\begin{task}{1.7 a)}
	Absolutbelopp:
	\[\abs{2-2i}=\sqrt{2^2+(-2)^2}=\sqrt{8}\]
	Argument:
	\[\arctan\left(\frac{-2}{2}\right)+2k\pi=-\frac{\pi}{4}+2k\pi, \qquad k\in\mathbb{Z}\]
\end{task}

\begin{task}{b)}
	Absolutbelopp:
	\[\abs{\sqrt{3}-i}=\sqrt{\sqrt{3}^2+(-1)^2}=\sqrt{4}=2\]
	Argument:
	\[\arctan\left(\frac{-1}{\sqrt{3}}\right)+2k\pi=-\frac{\pi}{6}+2k\pi, \qquad k\in\mathbb{Z}\]
\end{task}

\begin{task}{c)}
	Absolutbelopp:
	\[\abs{1}=1\]
	Argument:
	\[\arctan\left(\frac{0}{1}\right)+2k\pi=2k\pi, \qquad k\in\mathbb{Z}\]
\end{task}

\begin{task}{d)}
	Absolutbelopp:
	\[\abs{-1}=1\]
	Argument:
	\[\arctan\left(\frac{0}{1}\right)+2k\pi=\pi+2k\pi, \qquad k\in\mathbb{Z}\]
\end{task}

\begin{task}{1.8 a)}
	Låt $z = -1-i = re^{i\phi}$ där $r$ är absolutbeloppet och $\phi$ är argumentet.
	\[r=\sqrt{(\Re z)^2+(\Im z)^2}=
	\sqrt{(-1)^2+(-1)^2}=
	\sqrt{2}\]
	\begin{align*}
		\phi=&\arctan\left(\frac{\Im z}{\Re z}\right)+2k\pi, \qquad k \in \mathbb{Z} \qquad \ra \\
		\phi=&\arctan\left(\frac{-1}{-1}\right)+2k\pi=
		\frac{\pi}{4}+\pi+2k\pi=
		\frac{5}{4}\pi+2k\pi, \qquad k \in \mathbb{Z}
	\end{align*}
	\[z=\sqrt{2}e^{i(3\pi/4+2k\pi)}, \qquad k \in \mathbb{Z}\]
	Partikulärlösning:
	\[z=\sqrt{2}e^{i3\pi/4}\]
	\ans $z=\sqrt{2}e^{i3/4\pi}$
\end{task}

\begin{task}{b)}
	Låt $z = i = re^{i\phi}$ där $r$ är absolutbeloppet och $\phi$ är argumentet.
	\[r=\sqrt{(\Re z)^2+(\Im z)^2}=
	\sqrt{0^2+1^2}=
	1\]
	Eftersom $\Re z = 0$ och $\Im z > 0$ är $\phi=\frac{\pi}{2} + 2k\pi \qquad k \in \mathbb{Z}$
	\[z=e^{i(\pi/2+2k\pi)}, \qquad k \in \mathbb{Z}\]
	Partikulärlösning:
	\[z=e^{i\pi/2}\]
	\ans $z=e^{i\pi/2}$
\end{task}

\begin{task}{1.9}
	Utnyttja sambandet $e^{i\theta}=\cos\theta+i\sin\theta$.
	\begin{align*}
		5e^{2\pi i/3}=
		5\left(\cos\left(\frac{2\pi}{3}\right)+i\sin\left(\frac{2\pi}{3}\right)\right)=
		5\left(-\frac{1}{2}+i\frac{\sqrt{3}}{2}\right)=
		-\frac{5}{2}+i\frac{5\sqrt{3}}{2}
	\end{align*}
	\ans $-\frac{5}{2}+i\frac{5\sqrt{3}}{2}$
\end{task}

\begin{task}{1.10 a)}
	Låt $z=re^{i\phi}, \quad r \ge 0$
	\begin{align*}
		z^4+1=0 \lra 
		&(re^{i\phi})^4=-1 \lra
		r^4e^{i4\phi}=e^{\pi+2k\pi}, \quad k\in\mathbb{Z} \lra
		\begin{cases}
		4\phi=\pi+2k\pi, \quad k\in\mathbb{Z} \\
		r^4=1
		\end{cases} \lra \\ \lra
		&\begin{cases}
		\phi=\frac{\pi}{4}+\frac{k\pi}{2}, \quad k\in\mathbb{Z} \\
		r=1
		\end{cases}
	\end{align*}
	$k=\{0,1,2,3\}$ ger alla unika lösningar.
	
	\ans $e^{\pi i/4+k\pi i/2} \qquad k=\{0,1,2,3\}$
	\\ \\
	Eller:
	\\ \\
	Använd $\sqrt{i}=(e^{\pi i/2})^{1/2}=e^{\pi i/4}=\frac{1}{\sqrt{2}}(1+i)$ och $\sqrt{-i}=(e^{-\pi i/2})^{1/2}=e^{-\pi i/4}=\frac{1}{\sqrt{2}}(1-i)$
	\begin{align*}
	z^4+1=0 \lra 
	&z^4=-1 \lra
	\sqrt{z^4}=\pm\sqrt{-1} \lra
	z^2=\pm i \lra
	\sqrt{z^2}=\pm\sqrt{\pm i} \lra \\ \lra
	&z=\pm\frac{1}{\sqrt{2}}(1\pm i)=\frac{1}{\sqrt{2}}(\pm 1\pm i)
	\end{align*}
\end{task}

\begin{task}{b)}
	Låt $z=re^{\phi i}, \quad r \ge 0$
	\begin{align*}
		z^5=32 \lra
		&(re^{\phi i})^5=32 \lra
		r^5e^{5\phi i}=32e^{2k\pi i}, \quad k\in\mathbb{Z} \lra
		\begin{cases}
		5\phi=2k\pi, \quad k\in\mathbb{Z} \\
		r^5=32
		\end{cases} \lra \\ \lra
		&\begin{cases}
		\phi=\frac{2k\pi}{5}, \quad k\in\mathbb{Z} \\
		r=2
		\end{cases}
	\end{align*}
	$k=\{0,1,2,3,4\}$ ger alla unika lösningar.
	
	\ans $e^{2k\pi i/5} \qquad k=\{0,1,2,3,4\}$
\end{task}

\begin{task}{1.11}
	\begin{align*}
		e^{3ix}=
		(e^{ix})^3=
		(\cos x + i\sin x)^3=
		\cos^3x+i3\cos^2x\sin x-3\cos x\sin^2x-i\sin^3x
	\end{align*}
	\begin{align*}
	\cos 3x=
	&\Re e^{3ix}=
	\cos^3x-3\cos x\sin^2x=
	\cos^3x-3\cos x(1-\cos^2x)= \\ =
	&\cos^3x-3\cos x+3\cos^3x=
	4\cos^3x-3\cos x
	\end{align*}
	
	\ans $4\cos^3x-3\cos x$
\end{task}

\begin{task}{1.12 a)}
	Se formelblad.
	
	$C = b+ai$ där $a\cos\omega t + b\sin\omega t$
	\[\sqrt{3}\cos\omega t - \sin\omega t \lra
	\begin{cases}
	a=\sqrt{3} \\
	b=-1
	\end{cases} \lra
	C=-1+i\sqrt{3}\]
	
	\ans $C=-1+i\sqrt{3}$
\end{task}