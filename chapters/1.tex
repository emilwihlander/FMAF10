\chapter{1}{Svängningar och komplexa tal}
\begin{task}{1.1 a)}
Allmänna funktionen för odämpad harmonisk svängning är $u(t)=A\sin(\omega t + \alpha)$ där $\omega$ är vinkelfrekvensen.
\[u(t)=3\sin(2t-5) \ra \omega=2\]
\[T=\frac{2\pi}{\omega} \ra T = \frac{2\pi}{2} = \pi\]
\[f=\frac{1}{T} \ra f = \frac{1}{\pi}\]
\ans vinkelfrekvens: 2, period: $\pi$, frekvens: $\frac{1}{\pi}$
\end{task}

\begin{task}{b)}
	Allmänna funktionen för odämpad harmonisk svängning är $u(t)=A\sin(\omega t + \alpha)$ där $\omega$ är vinkelfrekvensen.
	\[u(t)=50\sin(100\pi t+1) \ra \omega=100\pi\]
	\[T=\frac{2\pi}{\omega} \ra T = \frac{2\pi}{100\pi} = \frac{1}{50}\]
	\[f=\frac{1}{T} \ra f = 50\]
	\ans vinkelfrekvens: $100\pi$, period: $\frac{1}{50}$, frekvens: 50
\end{task}

\begin{task}{1.2 a)}
\end{task}

\begin{task}{b)}
\end{task}

\begin{task}{c)}
\end{task}

\begin{task}{d)}
\end{task}

\begin{task}{e)}
\end{task}

\begin{task}{f)}
\end{task}

\begin{task}{1.3}
	Använd regeln $\sin(\alpha+\beta)=\sin\alpha\cos\beta+\cos\alpha\sin\beta$ från formelbladet.
	\begin{align*}
	u(t)= &
	6\sin(3t+\frac{\pi}{4})=
	6(\sin(3t)\cos(\frac{\pi}{4})+\cos(3t)\sin(\frac{\pi}{4}))= \\ =
	& 6\frac{1}{\sqrt{2}}\sin(3t)+6\frac{1}{\sqrt{2}}\cos(3t)=
	3\sqrt{2}\cos(3t)+3\sqrt{2}\sin(3t)
	\end{align*}
	\ans $a=b=3\sqrt{2}, \omega=3 \ra 3\sqrt{2}\cos(3t)+3\sqrt{2}\sin(3t)$
\end{task}

\pagebreak
\begin{task}{1.4 a)}
	låt $u(t)=A\sin(\omega t + \alpha)=A\sin\alpha\cos(\omega t)+A\cos\alpha\sin(\omega t)=\sqrt{3}\cos(\omega t)-\sin(\omega t)$ där $A$ är amplituden och $\alpha$ är fasförskjutningen.
	\begin{align*}
		\begin{linsys}{rr}
		A\sin\alpha=&\sqrt{3} \\
		A\cos\alpha=&-1
		\end{linsys} \lra &
		\sqrt{(A\sin\alpha)^2+(A\cos\alpha)^2}=\sqrt{(\sqrt{3})^2+(-1)^2} \lra \\ \lra
		& \sqrt{A^2}\sqrt{\sin\alpha^2+\cos\alpha^2}=\sqrt{4} \ra
		A\sqrt{1}=2 \lra
		A=2
	\end{align*}
	\begin{align*}
		\tan\alpha= &
		\frac{\sin\alpha}{\cos\alpha}=
		\frac{A\sin\alpha}{A\cos\alpha}=
		\frac{\sqrt{3}}{-1} \ra \\ \ra
		\alpha= &
		\arctan(-\frac{\sqrt{3}}{1})+\pi=
		-\frac{\pi}{6}+\pi=
		\frac{2\pi}{3}~~(+\pi \text{ ty } -4<0)
	\end{align*}
	eller:
	\begin{align*}
		u(t)= &
		\sqrt{3}\cos(\omega t)-\sin(\omega t)=
		2(\frac{\sqrt{3}}{2}\cos(\omega t)-\frac{1}{2}\sin(\omega t))= \\ =
		& 2(\sin\frac{2\pi}{3}\cos(\omega t)+\cos\frac{2\pi}{3}\sin(\omega t))=
		\sin(\omega t + \frac{2\pi}{3})
	\end{align*}
	\ans Amplitud: $2$ och fasförskjutning: $\frac{2\pi}{3}$
\end{task}

\begin{task}{1.4 b)}
	låt $u(t)=A\sin(\omega t + \alpha)=A\sin\alpha\cos(\omega t)+A\cos\alpha\sin(\omega t)=-2\cos(\omega t)-4\sin(\omega t)$ där $A$ är amplituden och $\alpha$ är fasförskjutningen.
	\begin{align*}
	\begin{linsys}{rr}
	A\sin\alpha=&-2 \\
	A\cos\alpha=&-4
	\end{linsys} \lra &
	\sqrt{(A\sin\alpha)^2+(A\cos\alpha)^2}=\sqrt{(-2)^2+(-4)^2} \lra \\ \lra
	& \sqrt{A^2}\sqrt{\sin\alpha^2+\cos\alpha^2}=\sqrt{4+16} \ra
	A\sqrt{1}=\sqrt{20} \lra
	A=2\sqrt{5}
	\end{align*}
	\begin{align*}
	\tan\alpha= &
	\frac{\sin\alpha}{\cos\alpha}=
	\frac{A\sin\alpha}{A\cos\alpha}=
	\frac{-2}{-4} \ra \\ \ra
	\alpha= &
	\arctan\frac{1}{2}+\pi~~(+\pi \text{ ty } -4<0)
	\end{align*}
	\ans Amplitud: $2\sqrt{5}$ och fasförskjutning: $\arctan\frac{1}{2}+\pi$
\end{task}